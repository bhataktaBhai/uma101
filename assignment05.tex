\documentclass[12pt]{article}
\usepackage[utf8]{inputenc}
\usepackage[a4paper, total={6in, 9in}]{geometry}
\usepackage{amsmath}
\usepackage{amssymb}
\usepackage{amsthm}
\usepackage{enumitem}
\usepackage{outlines}

\usepackage{paracol}

\usepackage[dvipsnames]{xcolor}
\colorlet{exercise}{red!80!black}
\colorlet{solved}{green!30!black}
\colorlet{self_proof}{blue!30!black}

\usepackage{mathrsfs}
\usepackage{parskip}

\usepackage{accents}
\newcommand{\ubar}[1]{\underaccent{\bar}{#1}}

\usepackage{hyperref}
\usepackage{cleveref}

\providecommand{\dd}{\,\mathrm{d}}

\usepackage{mathtools} % also contains \coloneqq
\DeclarePairedDelimiter{\oldparen}{(}{)}
\DeclarePairedDelimiter{\oldset}{\{}{\}}
\DeclarePairedDelimiter{\oldabs}{\lvert}{\rvert}
\DeclarePairedDelimiter{\oldnorm}{\lVert}{\rVert}
\DeclarePairedDelimiter{\oldfloor}{\lfloor}{\rfloor}
\DeclarePairedDelimiter{\oldceil}{\lceil}{\rceil}

\makeatletter
\def\paren{\@ifstar{\oldparen}{\oldparen*}}
\def\set{\@ifstar{\oldset}{\oldset*}}
\def\abs{\@ifstar{\oldabs}{\oldabs*}}
\def\norm{\@ifstar{\oldnorm}{\oldnorm*}}
\def\floor{\@ifstar{\oldfloor}{\oldfloor*}}
\def\ceil{\@ifstar{\oldceil}{\oldceil*}}

\g@addto@macro\normalsize{%
  \setlength\abovedisplayskip{7pt}%
  \setlength\belowdisplayskip{7pt}%
  \setlength\abovedisplayshortskip{1pt}%
  \setlength\belowdisplayshortskip{1pt}%
}
\makeatother

\newcommand\N{\ensuremath{\mathbb{N}}}
\newcommand\R{\ensuremath{\mathbb{R}}}
\newcommand\Z{\ensuremath{\mathbb{Z}}}
\renewcommand\O{\ensuremath{\varnothing}}
\renewcommand\P{\ensuremath{\mathbb{P}}}
\newcommand\Q{\ensuremath{\mathbb{Q}}}
\newcommand\C{\ensuremath{\mathbb{C}}}

% make implied by and implies arrows shorter
\let\implies\Rightarrow
\let\impliedby\Leftarrow
\let\iff\Leftrightarrow

% make \epsilon and \varepsilon the same
\let\epsilon\varepsilon

% <theorems>
% \theoremstyle{plain}
% \newtheorem{thm}{Theorem}[section]
% \newtheorem*{thm*}{Theorem}
% \newtheorem{prop}[thm]{Proposition}
% \newtheorem{lem}[thm]{Lemma}
% \newtheorem{cor}[thm]{Corollary}
% \newtheorem{axiom}{Axiom}[section]
% \newtheorem*{exer}{Exercise}
% 
% \theoremstyle{definition}
% \newtheorem{defn}[thm]{Definition}
% 
% \theoremstyle{remark}
% \newtheorem*{rem}{Remark} 
% \newtheorem*{example}{Example}
% </theorems>

% Gilles Castel's theorems
\usepackage[framemethod=Tikz]{mdframed}
\mdfsetup{skipabove=1em,skipbelow=0em}
\mdfdefinestyle{axiomstyle}{
    outerlinewidth = 1.5,
    roundcorner = 10,
    leftmargin = 15,
    rightmargin = 15,
    backgroundcolor = blue!10
}
\mdfdefinestyle{defnstyle}{
    outerlinewidth = 1.5,
    % roundcorner = 10,
    leftmargin = 7,
    rightmargin = 7,
    backgroundcolor = green!10
}
\mdfdefinestyle{thmstyle}{
    outerlinewidth = 1.5,
    roundcorner = 10,
    leftmargin = 7,
    rightmargin = 7,
    backgroundcolor = yellow!10
}
\mdfdefinestyle{lemmastyle}{
    outerlinewidth = 1.5,
    roundcorner = 10,
    leftmargin = 7,
    rightmargin = 7,
    backgroundcolor = yellow!10
}
\theoremstyle{definition}
\newmdtheoremenv[nobreak=true, style=axiomstyle]{axiom}{Axiom}[section]
\newmdtheoremenv[nobreak=true, style=thmstyle]{thm}{Theorem}[section]
\newmdtheoremenv[nobreak=true]{prop}[thm]{Proposition}
\newmdtheoremenv[nobreak=true]{lem}[thm]{Lemma}
\newmdtheoremenv[nobreak=true]{cor}[thm]{Corollary}
\newmdtheoremenv[nobreak=true, style=defnstyle]{defn}[thm]{Definition}

\newcounter{assignment}
\newmdtheoremenv[nobreak=true]{problem}{Problem}[assignment]

\theoremstyle{remark}
\newtheorem*{rem}{Remarks}
\newtheorem*{example}{Example}

% <cref>
\crefname{thm}{theorem}{theorems}
\crefname{prop}{proposition}{propositions}
\crefname{lem}{lemma}{lemmas}
\crefname{cor}{corollary}{corollaries}
\crefname{axiom}{axiom}{axioms}
\crefname{defn}{definition}{definitions}
% </cref>

% <hyperlinks>
\hypersetup{colorlinks,
    linkcolor={red!50!black},
    citecolor={blue!50!black},
    urlcolor={blue!80!black}}
% </hyperlinks>

\usepackage{xifthen}
\def\testdateparts#1{\dateparts#1\relax}
\def\dateparts#1 #2 #3 #4 #5\relax{
    \marginpar{\small\textsf{\mbox{#1 #2 #3 #5}}}
}

\def\@lecture{}%
\newcommand*{\lecture}[3]{
    \ifthenelse{\isempty{#3}}{%
        \def\@lecture{Lecture #1}%
    }{%
        \def\@lecture{Lecture #1: #3}%
    }%
    \subsection*{\@lecture}
    \marginpar{\small\textsf{\vbox{#2}}}
    \vskip 6pt
}
\newcommand*{\assignment}[2]{%
    \stepcounter{assignment}%
    \subsection*{Assignment #1}
    \marginpar{\small\vbox{due #2}}
    \vskip 6pt
}
\makeatletter
\newcommand*{\refifdef}[3]{%label,command,fallback
    \@ifundefined{r@#1}{#3}{#2{#1}}%
}
\makeatother

\DeclareMathOperator{\dom}{dom}
\DeclareMathOperator{\ran}{ran}
\DeclareMathOperator{\spann}{span}
\DeclareMathOperator{\dimn}{dim}

\newcommand{\psum}[2]{\ensuremath{\mathrm{sum}_{#1}({#2})}}
\newcommand{\pprod}[2]{\ensuremath{\mathrm{prod}_{#1}({#2})}}

\newlist{examples}{enumerate}{1}
\setlist[examples]{label*=(\alph*)~,ref=(\alph*)}
\makeatletter
\newcommand\myitem[1][]{%
  \if\relax\detokenize{#1}\relax
    \item\relax
  \else
    \protected@edef\@currentlabel{#1}%
    \item[(#1)~]
  \fi}
\makeatother


\title{Assignment 05}
\author{Naman Mishra}
\date{20 November 2022}

\begin{document}
\maketitle

You may freely use (without proof):
\begin{enumerate}[label=(\roman*)]
    \item $0 < \cos(x) < \abs{\dfrac{\sin x}{x}} < 1$ for all $0 < \abs{x} < \dfrac{\pi}{2}$.
    \item Any trigonometric identities that you have seen in school.
    \item Any limits computed in Lecture 12-14.
\end{enumerate}

\begin{problem}
    Prove the squeeze theorem.
\end{problem}
\begin{thm}[squeeze theorem] \label{thm:limit:squeeze}
    Let $f$, $g$, $h$ be functions defined on some neighborhood $N$ of $p$, except perhaps at $p$. Suppose $f \leq g \leq h$ on $N$, and \[
        \lim_{x \to p} f(x) = \lim_{x \to p} h(x) = L \in \R.
    \] Then \[
        \lim_{x \to p} g(x) = L.
    \]
\end{thm}
\begin{proof}
    For any $\varepsilon > 0, \;\exists\; \delta_{1}, \delta_{2}$ such that \begin{align*}
        \abs{f(x) - a} &< \varepsilon \quad\;\forall\; x \in N \cap N_{\delta_{1}}(p) \setminus \set{p} \\
        \abs{h(x) - a} &< \varepsilon \quad\;\forall\; x \in N \cap N_{\delta_{2}}(p) \setminus \set{p}
    \end{align*}

    Thus for all $x \in N \cap N_{\delta}(p) \setminus \set{p}$, where $\delta = \min\set{\delta_{1}, \delta_{2}}$ so that $N_{\delta} = N_{\delta_{1}} \cap N_{\delta_{2}}$,
    \begin{align*}
        a - \varepsilon < f(x) \leq g(x) &\leq h(x) < a + \varepsilon \\
        \implies \lim_{x \to p} g(x) &= a \qedhere
    \end{align*}
\end{proof}

\begin{problem}
    In each of the following cases, determine whether the limit exists or not, and compute the limit whenever it exists.
    You may use any of the theorems stated in class, but state what you are using.
    \begin{enumerate}[label=(\alph*)]
        \item $\lim\limits_{x \to 2} \dfrac{(3x + 1)^{2} - 49}{x - 2}$
        \item $\lim\limits_{x \to 0} \cos\paren{\dfrac{1}{x}}$
        \item $\lim\limits_{x \to 0} \dfrac{\sqrt{1 + x} - \sqrt{1 - x}}{x}$
        \item $\lim\limits_{x \to p} x^{n}$ (for fixed $n \in \N$ and $p \in \R$)
    \end{enumerate}
\end{problem}
\begin{proof}
    \begin{enumerate}[label=(\alph*)]
        \item For any $x \neq  2$, \[
            \frac{(3x + 1)^{2} - 49}{x - 2} = \frac{(3x + 8)(3x - 6)}{x - 2} = 3(3x + 8)
        \] By limit laws, \[
            \lim_{x \to 2} 3(3x + 8) = (\lim_{x \to 2} 3) ((\lim_{x \to 2} 3) \lim_{x \to 2} x + \lim_{x \to 2} 8) = 3(3 \cdot 2 + 8) = 42
        \]
    
        \item Suppose the function has a limit $L$.
        Take $\varepsilon = 1$.
        Then $\exists\; \delta > 0$ such that $\abs{\cos \paren{\frac{1}{x}} - L} < 1$ for all $0 < \abs{x - 0} < \delta$.
        By the Archimedean property, there exists $N$ such that $N 2\pi > \frac{1}{\delta} \implies 0 < \frac{1}{(2N + 1)\pi} < \frac{1}{2 N \pi} < \delta$.
        \begin{align*}
            \abs{1 - L} &< 1 \\
            \abs{-1 - L} &< 1 \\
            2 = \abs{1 - L + L + 1} &\leq \abs{1 - L} + \abs{-1 - L} < 1 + 1 = 2
        \end{align*}
        Contradiction.
    
        \item \[
            \frac{\sqrt{1 + x} - \sqrt{1 - x}}{x} = \frac{2}{\sqrt{1 + x} + \sqrt{1 - x}} 
        \] Since $\lim_{x \to 1} \sqrt{x} = 1$, we have $\lim_{x \to 0} \sqrt{1 + x} = \lim_{x \to 0} \sqrt{1 - x} = 1$. By limit laws, \[
            \lim_{x \to 0} \frac{\sqrt{1 + x} - \sqrt{1 - x}}{x} = \frac{2}{2} = 1
        \]
    
        \item We have $\lim_{x \to p} x^{0} = \lim_{x \to p} 1 = 1 = p^{0}$.
        Suppose $\lim_{x \to p} x^{n} = p^{n} \;\forall\; p \in \R$ for some $n$.
        Then $\lim_{x \to p} x^{n + 1} = \lim_{x \to p} x \cdot \lim_{x \to p} x^{n} = p \cdot p^{n} = p^{n + 1} \;\forall\; p \in \R$.
        Thus $\lim_{x \to p} x^{n} = p^{n} \;\forall\; p \in \R \;\forall\; n \in \N$. \qedhere
    \end{enumerate}
\end{proof}
\begin{problem}
    Let $f$ and $g$ be functions on $\R$ such that \[
        \lim_{x \to 0} f(x) = L \quad\text{and}\quad \lim_{y \to 0} g(y) = M,
    \] for some $L, M \in \R$.
    Is it true that \[
        \lim_{x \to 0} (g \circ f)(x) = M?
    \] If your answer is ``yes'', prove the above statement.
    If your answer is ``no'', provide a counterexample, and give a sufficient condition on $g$ that will make the above statement true.
\end{problem}
\begin{proof}
    \emph{NO.}
    Let $f(x) = 0$, $g(x) = \begin{cases} 0 & x \neq 0 \\ 1 & x = 0 \end{cases}$.
    Then $\lim_{x \to 0} f(x) = 0$ and $\lim_{x \to 0} g(x) = 0$, but $\lim_{x \to 0} g \circ f (x) = 1$.
    
    This arises from the fact that $\abs{f(x) - L} < \varepsilon$ does not imply $0 < \abs{f(x) - L}$, so it doesn't guarantee $0 < \abs{y - L} < \delta$.
    
    If we have $g$ continuous at $L$, then we know $g$ is defined at $L$ as well.
    
    For any $\varepsilon > 0$, $\exists\; \delta_{1} > 0$ such that $\abs{y - L} < \delta_{1} \implies g(y) \textrm{is defined and} \abs{g(y) - M} < \varepsilon$.
    Choose $\varepsilon_{2} = \delta_{1}$.
    Then there exists $\delta > 0$ such that $0 < \abs{x - 0} < \delta \implies f(x)$ is defined and $\abs{f(x) - L} < \varepsilon_{2} = \delta_{1} \implies \abs{g(f(x)) - M} < \varepsilon$.
\end{proof}
\begin{rem}
    We can also enforce a condition on $f$: if $f(x) \neq L \;\forall\; x \neq 0$, then $\lim_{x \to 0} g(f(x)) = \lim_{x \to L} g(x)$.
\end{rem}
This yields a theorem:
\begin{thm}[limit of composition] \label{thm:limit:composition}
    Suppose $f$ and $g$ are functions on $\R$ such that \[
        \lim_{x \to a} f(x) = L \;\textrm{and}\; \lim_{x \to L} g(x) = M.
    \] Then we have \[
        \lim_{x \to a} g(f(x)) = M
    \] if
    \begin{enumerate}[label=(\alph*)]
        \item $g$ is continuous, or
        \item $f(x) \neq L \;\forall\; x \in N_{\delta}(a) \setminus \set{a}$ for some $\delta > 0$.
    \end{enumerate}
\end{thm}

\begin{problem}
    Prove the sequential characterization of continuity.
\end{problem}
\begin{thm}[sequential characterization of continuity] \label{thm:cont:sequential_characterization}
    Let $f : A \to \R$ be a function and let $p \in A$.
    Let $P = \set{\set{a_{n}} \subseteq A : \lim\limits_{n \to \infty} a_{n} = p}$.
    Then $f$ is continuous at $p$ iff $\lim\limits_{n \to \infty} f(a_{n}) = f(p) \;\forall\; \set{a_{n}} \in P$.
\end{thm}
\begin{proof} \leavevmode
    \begin{enumerate}[label=(\alph*)]
        \item Let $f$ be continuous at $p$ and let $\set{a_{n}} \in P$.

        For every $\varepsilon > 0$ there exists $\delta$ such that $f(x) \in N_{\varepsilon}(f(p)) \;\forall\; x \in N_{\delta}(p) \cap A$.
        Morever, $\exists\; N \in \N$ such that $ a_{n} \in N_{\delta}(p) \cap A \;\forall\; n \geq N$.

        Thus for all $n \geq N$, $ f(a_{n}) \in N_{\varepsilon}(f(p))$.
        Thus $\lim_{n \to \infty} f(a_{n}) = f(p)$.
        \item Let $\lim_{n \to \infty} f(a_{n}) = f(p) \;\forall\; \set{a_{n}} \in P$.
        Suppose $f$ is not continuous at $p$.
        Then there exists an $\varepsilon > 0$ such that for all $\delta > 0$, there exists $a \in N_{\delta}(p) \cap A \setminus \set{p}$ such that $\abs{f(a) - f(p)} \geq \varepsilon$.

        Let $\set{\delta_{n}}_{n \in \P} = \set{\frac{1}{n}}_{n \in \P}$.
        Then corresponding to every $\delta_{n} \;\exists\; a_{n} \in A \cap N_{\delta_{n}}(p) \setminus \set{p}$ such that $\abs{f(a_{n}) - f(p)} \geq \varepsilon$.

        Thus $\set{a_{n}} \to p$ but $\lim\limits_{n \to \infty} f(a_{n}) \neq f(p)$.
        Contradiction. \qedhere

        (This proof uses the axiom of choice. Let me know if you have a proof without it.)
    \end{enumerate}
\end{proof}

\begin{problem}
    Complete the following steps to establish the continuity of the sine and cosine functions on \R. Recall (i) and (ii) given at the beginning of this assignment.
    \begin{enumerate}[label=(\alph*)]
        \item Show that $\lim\limits_{x \to 0} \sin(x) = 0$.
        \item Using (a) and a trigonometric identity relating sin and cos, show that $\lim\limits_{x \to 0} \cos(x) = 1$.
        \item Using (a) and (b), show that sin and cos are continuous at any $x \in \R$.
        \item Show that $\lim\limits_{x \to 0} \dfrac{\sin x}{x} = 1$.
    \end{enumerate}
\end{problem}
\begin{proof} \leavevmode
    \begin{enumerate}[label=(\alph*)]
        \item \begin{equation*}
            0 < \abs{\frac{\sin x}{x}} < 1 \implies -\abs{x} < \sin x < \abs{x} \qquad\forall\; 0 < \abs{x} < \frac{\pi}{2}
        \end{equation*}
        Since $\lim_{x \to 0} -\abs{x} = \lim_{x \to 0} \abs{x} = 0$, we have $\lim_{x \to 0} \sin x = 0$ by \nameref{thm:limit:squeeze}.
        \item Since $\cos^{2}(x) + \sin^{2}(x) = 1$, we have \[
            \lim_{x \to 0} \cos^{2}(x) = \lim_{x \to 0}(1 - \sin^{2}(x)) = 1 - \lim_{x \to 0} \sin^{2} x = 1 - \paren{\lim_{x \to 0} \sin x}^{2} =  1
        \] Since $\cos x > 0$ for $\abs{x - 0} < \frac{\pi}{2}$, we have $\sqrt{\cos^{2} (x)} - \cos x = 0$ for $\abs{x - 0} < \frac{\pi}{2}$.
        So $\lim_{x \to 0} (\sqrt{\cos^{2}x} - \cos x) = 0$. Thus \[
            \lim_{x \to 0} \cos x = \lim_{x \to 0} \sqrt{\cos^{2} x} = \sqrt{\lim_{x \to 0} \cos^{2} x} = 1
        \] by \nameref{thm:limit:composition} ($\sqrt{\cdot}$ is continuous).
        \item Now
        \begin{align*}
            \sin(x + h) &= \sin x \cos h & \cos(x + h) &= \cos x \cos h \\
            &+ \cos x \sin h &&- \sin x \sin h \\
            \lim_{h \to 0} \sin(x + h) &= \sin x & \lim_{h \to 0} \cos(x + h) &= \cos x \\
            \lim_{y \to x} \sin(y) &= \sin x & \lim_{y \to x} \cos(y) &= \cos x
        \end{align*} for all $x \in \R$. Thus $\sin$ and $\cos$ are continuous at any $x \in \R$
        \item Finally, we have \[
            \abs{\frac{\sin x}{x}} = \frac{\sin x}{x} \quad\;\forall\; 0 < \abs{x} < \frac{\pi}{2}
        \] so \[
            \cos x < \frac{\sin x}{x} < 1 \quad \forall\; 0 < \abs{x} < \frac{\pi}{2}.
        \] By the squeeze theorem, we get \[
            \lim_{x \to 0} \frac{\sin x}{x} = 1. \qedhere
        \]
    \end{enumerate}
\end{proof}
\end{document}
