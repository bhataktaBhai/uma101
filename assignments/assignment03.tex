\documentclass[12pt]{article}
\usepackage[utf8]{inputenc}
\usepackage[a4paper, total={6in, 9in}]{geometry}
\usepackage{amsmath}
\usepackage{amssymb}
\usepackage{amsthm}
\usepackage{enumitem}
\usepackage{outlines}

\usepackage{paracol}

\usepackage[dvipsnames]{xcolor}
\colorlet{exercise}{red!80!black}
\colorlet{solved}{green!30!black}
\colorlet{self_proof}{blue!30!black}

\usepackage{mathrsfs}
\usepackage{parskip}

\usepackage{accents}
\newcommand{\ubar}[1]{\underaccent{\bar}{#1}}

\usepackage{hyperref}
\usepackage{cleveref}

\providecommand{\dd}{\,\mathrm{d}}

\usepackage{mathtools} % also contains \coloneqq
\DeclarePairedDelimiter{\oldparen}{(}{)}
\DeclarePairedDelimiter{\oldset}{\{}{\}}
\DeclarePairedDelimiter{\oldabs}{\lvert}{\rvert}
\DeclarePairedDelimiter{\oldnorm}{\lVert}{\rVert}
\DeclarePairedDelimiter{\oldfloor}{\lfloor}{\rfloor}
\DeclarePairedDelimiter{\oldceil}{\lceil}{\rceil}

\makeatletter
\def\paren{\@ifstar{\oldparen}{\oldparen*}}
\def\set{\@ifstar{\oldset}{\oldset*}}
\def\abs{\@ifstar{\oldabs}{\oldabs*}}
\def\norm{\@ifstar{\oldnorm}{\oldnorm*}}
\def\floor{\@ifstar{\oldfloor}{\oldfloor*}}
\def\ceil{\@ifstar{\oldceil}{\oldceil*}}

\g@addto@macro\normalsize{%
  \setlength\abovedisplayskip{7pt}%
  \setlength\belowdisplayskip{7pt}%
  \setlength\abovedisplayshortskip{1pt}%
  \setlength\belowdisplayshortskip{1pt}%
}
\makeatother

\newcommand\N{\ensuremath{\mathbb{N}}}
\newcommand\R{\ensuremath{\mathbb{R}}}
\newcommand\Z{\ensuremath{\mathbb{Z}}}
\renewcommand\O{\ensuremath{\varnothing}}
\renewcommand\P{\ensuremath{\mathbb{P}}}
\newcommand\Q{\ensuremath{\mathbb{Q}}}
\newcommand\C{\ensuremath{\mathbb{C}}}

% make implied by and implies arrows shorter
\let\implies\Rightarrow
\let\impliedby\Leftarrow
\let\iff\Leftrightarrow

% make \epsilon and \varepsilon the same
\let\epsilon\varepsilon

% <theorems>
% \theoremstyle{plain}
% \newtheorem{thm}{Theorem}[section]
% \newtheorem*{thm*}{Theorem}
% \newtheorem{prop}[thm]{Proposition}
% \newtheorem{lem}[thm]{Lemma}
% \newtheorem{cor}[thm]{Corollary}
% \newtheorem{axiom}{Axiom}[section]
% \newtheorem*{exer}{Exercise}
% 
% \theoremstyle{definition}
% \newtheorem{defn}[thm]{Definition}
% 
% \theoremstyle{remark}
% \newtheorem*{rem}{Remark} 
% \newtheorem*{example}{Example}
% </theorems>

% Gilles Castel's theorems
\usepackage[framemethod=Tikz]{mdframed}
\mdfsetup{skipabove=1em,skipbelow=0em}
\mdfdefinestyle{axiomstyle}{
    outerlinewidth = 1.5,
    roundcorner = 10,
    leftmargin = 15,
    rightmargin = 15,
    backgroundcolor = blue!10
}
\mdfdefinestyle{defnstyle}{
    outerlinewidth = 1.5,
    % roundcorner = 10,
    leftmargin = 7,
    rightmargin = 7,
    backgroundcolor = green!10
}
\mdfdefinestyle{thmstyle}{
    outerlinewidth = 1.5,
    roundcorner = 10,
    leftmargin = 7,
    rightmargin = 7,
    backgroundcolor = yellow!10
}
\mdfdefinestyle{lemmastyle}{
    outerlinewidth = 1.5,
    roundcorner = 10,
    leftmargin = 7,
    rightmargin = 7,
    backgroundcolor = yellow!10
}
\theoremstyle{definition}
\newmdtheoremenv[nobreak=true, style=axiomstyle]{axiom}{Axiom}[section]
\newmdtheoremenv[nobreak=true, style=thmstyle]{thm}{Theorem}[section]
\newmdtheoremenv[nobreak=true]{prop}[thm]{Proposition}
\newmdtheoremenv[nobreak=true]{lem}[thm]{Lemma}
\newmdtheoremenv[nobreak=true]{cor}[thm]{Corollary}
\newmdtheoremenv[nobreak=true, style=defnstyle]{defn}[thm]{Definition}

\newcounter{assignment}
\newmdtheoremenv[nobreak=true]{problem}{Problem}[assignment]

\theoremstyle{remark}
\newtheorem*{rem}{Remarks}
\newtheorem*{example}{Example}
\newtheorem*{examplex}{Examples}

\newenvironment{examples}
{\begin{examplex}\leavevmode\begin{itemize}}{\end{itemize}\end{examplex}}
% <cref>
\crefname{thm}{theorem}{theorems}
\crefname{prop}{proposition}{propositions}
\crefname{lem}{lemma}{lemmas}
\crefname{cor}{corollary}{corollaries}
\crefname{axiom}{axiom}{axioms}
\crefname{defn}{definition}{definitions}
\crefname{problem}{problem}{problems}
% </cref>

% <hyperlinks>
\hypersetup{colorlinks,
    linkcolor={red!50!black},
    citecolor={blue!50!black},
    urlcolor={blue!80!black}}
% </hyperlinks>

\usepackage{xifthen}
\def\testdateparts#1{\dateparts#1\relax}
\def\dateparts#1 #2 #3 #4 #5\relax{
    \marginpar{\small\textsf{\mbox{#1 #2 #3 #5}}}
}

\def\@lecture{}%
\newcommand*{\lecture}[3]{
    \ifthenelse{\isempty{#3}}{%
        \def\@lecture{Lecture #1}%
    }{%
        \def\@lecture{Lecture #1: #3}%
    }%
    \subsection*{\@lecture}
    \marginpar{\raggedright\small\textsf{#2}}
    \vskip 6pt
}
\newcommand*{\assignment}[2]{%
    \stepcounter{assignment}%
    \subsection{Assignment #1}
    \marginpar{\raggedright\small{due #2}}
    \vskip 6pt
}
\makeatletter
\newcommand*{\refifdef}[3]{%label,command,fallback
    \@ifundefined{r@#1}{#3}{#2{#1}}%
}
\makeatother

\DeclareMathOperator{\dom}{dom}
\DeclareMathOperator{\ran}{ran}
\DeclareMathOperator{\spann}{span}
\DeclareMathOperator{\dimn}{dim}

\newcommand{\psum}[2]{\ensuremath{\mathrm{sum}_{#1}({#2})}}
\newcommand{\pprod}[2]{\ensuremath{\mathrm{prod}_{#1}({#2})}}


\title{Assignment 03}
\author{Naman Mishra}
\date{4 November 2022}

\begin{document}
\maketitle

\begin{problem}
    Let $x \in \R$ such that $0 \leq x < \delta$ for every $\delta > 0$.
    Show that $x$ must be 0. Explicitly state the field and order axioms that you are using.
\end{problem}
\begin{proof}
    By \labelcref{defn:order:trichotomy}, done.
\end{proof}
%\begin{proof}
%    We will first prove that inequalities can be added. \\
%    Suppose $a < b$ and $c < d$. Then by (O3),
%    \begin{align*}
%        a + c &< b + c  &  b + c < b + d
%    \end{align*}
%    Thus by (O2), $a + c < b + d$.
%
%    Now since $2 = 1 + 1$, $2x = 1x + 1x$ (distributivity) $= x + x$ (identity). \\
%    Thus $\frac{1}{2} + \frac{1}{2} = 2 \frac{1}{2} = 1$ (multiplicative inverse).
%
%    Now suppose for some $0 < x$, $\frac{1}{x} < 0$. \\
%    Then by (O4), $0 < x \cdot \paren{- \frac{1}{x}} \implies 0 < -\paren{x \cdot \frac{1}{x}} \implies 0 < -1$ which is false.
%\end{proof}

\begin{problem}
    Formulate  definitions  of  the  terms  ``bounded  below  set'',  ``lower  bound''  and ``greatest lower bound'' for subsets of $\R$.
    Show that $\Z$ is neither bounded above nor bounded below.
\end{problem}
\begin{defn} \label{defn:bounds:lower}
    A subset $S \subseteq \R$ is said to be \emph{bounded below} if there exists an element $b \in \R$ such that $\forall\; s \in S, b \leq s$.

    Here, $b$ is called a \emph{lower bound} of $S$.

    $b$ is said to be a (the) \emph{greatest lower bound} if $\forall\; b' > b$, $b'$ is not a lower bound of $S$, \textit{i.e.}, $\exists\; s \in S$ such that $s < b'$.
\end{defn}
\begin{proof}
    ($\Z$ is unbounded) Suppose $b \in \R$ is an upper bound of $\Z$.
    By the Archimedean property, there exists $n \in \P \implies n \in \Z$ such that $n\cdot 1 = n > b$.
    Hence $b$ is not an upper bound.
    
    Next suppose $b \in \R$ is a lower bound of $\Z$.
    By the Archimedean property, there exists $n \in \P \implies n \in \Z$ such that $n \cdot 1 = n > -b \implies -n < b$.
    Since the additive inverse of an integer is also an integer, $b$ is not a lower bound.

    Thus $\Z$ cannot have an upper bound, nor a lower bound.
\end{proof}

\begin{problem}
    If $x$ is an arbitrary real number, prove that there is exactly one integer $n$ which satisfies \[
        n \leq x < n + 1
    \] You may use Theorem 1.28 from Apostol (without proof), which says $\P$ is not bounded above.
    Other than the least upper bound property of $\R$, you need not specify which axioms you are using in your proof.
\end{problem}
\begin{lem} \label{thm:N:no_natural_between_0_and_1}
    $1$ is the smallest positive natural number (where positive means non-zero).
\end{lem}
\begin{proof}
    Let $n$ be a positive natural number.
    From \cref{prob:peano:successor_range}, we know that $n$ has a predecessor $m$.
    Thus $n = m + 1 \implies n \geq 1$.
\end{proof}

\begin{lem} \label{thm:N:difference_geq_1}
    The symmetric difference between any two distinct integers is at least $1$.
\end{lem}
\begin{proof}
    Say $m, n \in \Z$ with $m > n$.
\end{proof}
\begin{proof}
    By the Archimedean property, there exists an $m$ such that $m \cdot 1 = m > x$. Thus the set \[
        S = \set{n \in \Z : n \leq x}
    \] is bounded above.
    Since $\Z$ is unbounded below, $\exists\; n \in \Z : n < y \;\forall\; y \in \R$. Thus the set $S$ is non-empty. \\
    Therefore the set $S$ has a least upper bound in $\R$.
    Call this $z$.
    Since $z-1$ is not an upper bound, there exists an $m \in S$ such that \[
        z - 1 < m \leq z < m + 1.
    \] Since $z < m + 1$, $m + 1 \notin S$.
    Thus $m \leq x < m + 1$.

    Now suppose $m_{1}$ and $m_{2}$ both satisfy this property.
    $m_{1} \leq x < m_{2} + 1 \implies m_{1} < m_{2} + 1$.
    Similarly $m_{2} < m_{1} + 1$.
    Thus $\abs{m_{1} - m_{2}} < 1$.
    Since $m_{1}$ and $m_{2}$ are integers, they cannot be distinct.
\end{proof}

\begin{problem}
    Let $\set{a_{n}} \subset R$ be an arbitrary sequence.
    Among the statements listed below,exactly one implies that $\set{a_{n}}$ is convergent, exactly one implies that $\set{a_{n}}$ is divergent, and the remaining one does not say anything conclusive about the convergence of $\set{a_{n}}$.
    Determine which is which.
    For the conclusive statements, you must give proofs.
    For the inconclusive statement, you must provide two sequences which satisfy the given statement, but one converges and the other diverges.
    \begin{enumerate}[label=(\arabic*)]
        \item There exists an $L \in \R$ such that for every $\varepsilon > 0$, there exists an $N \in \N$ such that $\abs{a_{n} - L} < n \varepsilon$ for all $n \geq N$.
        \item There exists an $L \in \R$ such that for every $\varepsilon > 0$, there exists an $N \in \N$ such that $\abs{a_{n} - L} < \frac{\varepsilon}{n + 1}$ for all $n \geq N$.
        \item For every $R > 0$, there exists an $N \in \N$ such that $\abs{a_{N}} > R$.
    \end{enumerate}
\end{problem}
\begin{proof}
    \begin{enumerate}[label=(\arabic*)]
        \item (Inconclusive) Suppose $a_{n} = 1 \;\forall\; n \in \N$.
        This converges to $1$, and passes the condition using the Archimedean property, and converges to 1.
        Now suppose $a_{n} = \sqrt{n}$.
        Then $\abs{a_{n} - 0} = \abs{\sqrt{n}}$.
        For any $\varepsilon > 0$, choose $N > \frac{1}{\varepsilon^{2}}$.
        Thus $n \geq N \implies \frac{1}{\sqrt{n}} < \frac{1}{\sqrt{n}} < \varepsilon$.
        Also $\frac{1}{\sqrt{n}} > 0 > -\varepsilon$.
        Thus the sequence satisfies the given condition, but diverges.
        \item (Convergent) Since $n + 1 > 1$, $\frac{\varepsilon}{n + 1} < \varepsilon$.
        So for any $\varepsilon > 0$, $\abs{a_{n} - L} < \frac{\varepsilon}{n + 1} < \varepsilon \implies \abs{a_{n} - L} < \varepsilon \implies \set{a_{n}}$ is convergent.
        \item (Divergent) Suppose the sequence converges to a limit $L$.
        Then there exists for all $\varepsilon > 0$, an $N \in \N$ such that $\abs{a_{n} - L} < \varepsilon$ for all $n \geq  N$.
        Let $R = \max(\set{\abs{a_{n}}}_{n \in \N, n < N} \;\cup\; \set{\abs{L} + \varepsilon})$.
        $\exists\; m \in \N$ such that $\abs{a_{m}} > R \implies \abs{a_{m}} > \abs{L} + \varepsilon$.
        $\abs{a_{m} - L} \geq \abs{\abs{a_{m}} - \abs{L}} > \abs{a_{m}} - \abs{L} > \varepsilon$.
        
        $m \not< N$ since $R \geq \abs{a_{n}} \;\forall\; n \in \N, n < N$.
        Thus $\exists\; m > N$ such that $\abs{a_{m} - L} > \varepsilon$, which contradicts the assumption that the sequence was convergent.

        \textbf{FALSE} alternative: We know that $\abs{a_{n}}$ diverges to $+\infty$ (\textcolor{red}{not necessarily}).
        Suppose $a_{n}$ converges to $L$.
        Then $\abs{\abs{a_{n}} - \abs{L}} \leq \abs{a_{n} - L}$.
        $\exists\; n \in \N$ such that for all $\varepsilon > 0, \abs{a_{n} - L} < \varepsilon$ for all $n \geq N$.This implies $\abs{\abs{a_{n}} - \abs{L}} < \varepsilon \;\forall\; n \geq N$, \textit{i.e.}, $\abs{a_{n}}$ converges to $\abs{L}$.

        Since $\abs{a_{n}}$ diverges, $a_{n}$ must also diverge. \qedhere
    \end{enumerate}
\end{proof}

\begin{problem}
    Compute the limit of the following sequences.
    \begin{enumerate}[label=(\arabic*)]
        \item \[
            \frac{2-3n^{2}}{n^{2} + 2n + 1}
        \]
    \end{enumerate}
\end{problem}
\end{document}




