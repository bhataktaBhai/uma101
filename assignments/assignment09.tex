\documentclass[12pt]{article}
\usepackage[utf8]{inputenc}
\usepackage[a4paper, total={6in, 9in}]{geometry}
\usepackage{amsmath}
\usepackage{amssymb}
\usepackage{amsthm}
\usepackage{enumitem}
\usepackage{outlines}

\usepackage{paracol}

\usepackage[dvipsnames]{xcolor}
\colorlet{exercise}{red!80!black}
\colorlet{solved}{green!30!black}
\colorlet{self_proof}{blue!30!black}

\usepackage{mathrsfs}
\usepackage{parskip}

\usepackage{accents}
\newcommand{\ubar}[1]{\underaccent{\bar}{#1}}

\usepackage{hyperref}
\usepackage{cleveref}

\providecommand{\dd}{\,\mathrm{d}}

\usepackage{mathtools} % also contains \coloneqq
\DeclarePairedDelimiter{\oldparen}{(}{)}
\DeclarePairedDelimiter{\oldset}{\{}{\}}
\DeclarePairedDelimiter{\oldabs}{\lvert}{\rvert}
\DeclarePairedDelimiter{\oldnorm}{\lVert}{\rVert}
\DeclarePairedDelimiter{\oldfloor}{\lfloor}{\rfloor}
\DeclarePairedDelimiter{\oldceil}{\lceil}{\rceil}

\makeatletter
\def\paren{\@ifstar{\oldparen}{\oldparen*}}
\def\set{\@ifstar{\oldset}{\oldset*}}
\def\abs{\@ifstar{\oldabs}{\oldabs*}}
\def\norm{\@ifstar{\oldnorm}{\oldnorm*}}
\def\floor{\@ifstar{\oldfloor}{\oldfloor*}}
\def\ceil{\@ifstar{\oldceil}{\oldceil*}}

\g@addto@macro\normalsize{%
  \setlength\abovedisplayskip{7pt}%
  \setlength\belowdisplayskip{7pt}%
  \setlength\abovedisplayshortskip{1pt}%
  \setlength\belowdisplayshortskip{1pt}%
}
\makeatother

\newcommand\N{\ensuremath{\mathbb{N}}}
\newcommand\R{\ensuremath{\mathbb{R}}}
\newcommand\Z{\ensuremath{\mathbb{Z}}}
\renewcommand\O{\ensuremath{\varnothing}}
\renewcommand\P{\ensuremath{\mathbb{P}}}
\newcommand\Q{\ensuremath{\mathbb{Q}}}
\newcommand\C{\ensuremath{\mathbb{C}}}

% make implied by and implies arrows shorter
\let\implies\Rightarrow
\let\impliedby\Leftarrow
\let\iff\Leftrightarrow

% make \epsilon and \varepsilon the same
\let\epsilon\varepsilon

% <theorems>
% \theoremstyle{plain}
% \newtheorem{thm}{Theorem}[section]
% \newtheorem*{thm*}{Theorem}
% \newtheorem{prop}[thm]{Proposition}
% \newtheorem{lem}[thm]{Lemma}
% \newtheorem{cor}[thm]{Corollary}
% \newtheorem{axiom}{Axiom}[section]
% \newtheorem*{exer}{Exercise}
% 
% \theoremstyle{definition}
% \newtheorem{defn}[thm]{Definition}
% 
% \theoremstyle{remark}
% \newtheorem*{rem}{Remark} 
% \newtheorem*{example}{Example}
% </theorems>

% Gilles Castel's theorems
\usepackage[framemethod=Tikz]{mdframed}
\mdfsetup{skipabove=1em,skipbelow=0em}
\mdfdefinestyle{axiomstyle}{
    outerlinewidth = 1.5,
    roundcorner = 10,
    leftmargin = 15,
    rightmargin = 15,
    backgroundcolor = blue!10
}
\mdfdefinestyle{defnstyle}{
    outerlinewidth = 1.5,
    % roundcorner = 10,
    leftmargin = 7,
    rightmargin = 7,
    backgroundcolor = green!10
}
\mdfdefinestyle{thmstyle}{
    outerlinewidth = 1.5,
    roundcorner = 10,
    leftmargin = 7,
    rightmargin = 7,
    backgroundcolor = yellow!10
}
\mdfdefinestyle{lemmastyle}{
    outerlinewidth = 1.5,
    roundcorner = 10,
    leftmargin = 7,
    rightmargin = 7,
    backgroundcolor = yellow!10
}
\theoremstyle{definition}
\newmdtheoremenv[nobreak=true, style=axiomstyle]{axiom}{Axiom}[section]
\newmdtheoremenv[nobreak=true, style=thmstyle]{thm}{Theorem}[section]
\newmdtheoremenv[nobreak=true]{prop}[thm]{Proposition}
\newmdtheoremenv[nobreak=true]{lem}[thm]{Lemma}
\newmdtheoremenv[nobreak=true]{cor}[thm]{Corollary}
\newmdtheoremenv[nobreak=true, style=defnstyle]{defn}[thm]{Definition}

\newcounter{assignment}
\newmdtheoremenv[nobreak=true]{problem}{Problem}[assignment]

\theoremstyle{remark}
\newtheorem*{rem}{Remarks}
\newtheorem*{example}{Example}

% <cref>
\crefname{thm}{theorem}{theorems}
\crefname{prop}{proposition}{propositions}
\crefname{lem}{lemma}{lemmas}
\crefname{cor}{corollary}{corollaries}
\crefname{axiom}{axiom}{axioms}
\crefname{defn}{definition}{definitions}
% </cref>

% <hyperlinks>
\hypersetup{colorlinks,
    linkcolor={red!50!black},
    citecolor={blue!50!black},
    urlcolor={blue!80!black}}
% </hyperlinks>

\usepackage{xifthen}
\def\testdateparts#1{\dateparts#1\relax}
\def\dateparts#1 #2 #3 #4 #5\relax{
    \marginpar{\small\textsf{\mbox{#1 #2 #3 #5}}}
}

\def\@lecture{}%
\newcommand*{\lecture}[3]{
    \ifthenelse{\isempty{#3}}{%
        \def\@lecture{Lecture #1}%
    }{%
        \def\@lecture{Lecture #1: #3}%
    }%
    \subsection*{\@lecture}
    \marginpar{\small\textsf{\vbox{#2}}}
    \vskip 6pt
}
\newcommand*{\assignment}[2]{%
    \stepcounter{assignment}%
    \subsection*{Assignment #1}
    \marginpar{\small\vbox{due #2}}
    \vskip 6pt
}
\makeatletter
\newcommand*{\refifdef}[3]{%label,command,fallback
    \@ifundefined{r@#1}{#3}{#2{#1}}%
}
\makeatother

\DeclareMathOperator{\dom}{dom}
\DeclareMathOperator{\ran}{ran}
\DeclareMathOperator{\spann}{span}
\DeclareMathOperator{\dimn}{dim}

\newcommand{\psum}[2]{\ensuremath{\mathrm{sum}_{#1}({#2})}}
\newcommand{\pprod}[2]{\ensuremath{\mathrm{prod}_{#1}({#2})}}

\newlist{examples}{enumerate}{1}
\setlist[examples]{label*=(\alph*)~,ref=(\alph*)}
\makeatletter
\newcommand\myitem[1][]{%
  \if\relax\detokenize{#1}\relax
    \item\relax
  \else
    \protected@edef\@currentlabel{#1}%
    \item[(#1)~]
  \fi}
\makeatother


\title{Assignment 9}
\author{Naman Mishra}
\date{January 3, 2022}

\begin{document}
\maketitle

\section*{Problem 1}

Let $s$ be a step function on partition $\mathcal{P} = \set{x_{0} < x_{1} < \dots < x_{n}}$ of $[a, b]$.
Suppose $\mathcal{P}'$ is a refinement of $\mathcal{P}$.
Let $\mathcal{W} = \mathcal{P}' \setminus \mathcal{P} = \set{y_{1} < y_{2} < \dots < y_{m}}$.
Let $\mathcal{P}_{0} = \mathcal{P}$ and $\mathcal{P}_{k+1} = \mathcal{P}_{k} \cup \set{y_{k+1}}$
Then $\mathcal{P}_{m} = \mathcal{P} \cup \mathcal{W} = \mathcal{P}'$.

We know that $s$ is a step function on $\mathcal{P}$ with some integral $I$.
Suppose inductively that $s$ is a step function on $\mathcal{P}_{k} = \set{z_{0} < z_{1} < \dots < z_{n+k}}$ with integral $I$.
Since $z_{0} = a < y_{k+1} < b = z_{n}$ and $y_{k+1} \not\in \mathcal{P}_{k}$, we have $z_{j-1} < y_{k+1} < z_{j}$ for some $j \in \N \cap [1, n+k]$. \[
    \mathcal{P}_{k+1} = \set{z_{0} < z_{1} < \dots < z_{j-1} < y_{k+1} < z_{j} < \dots < z_{n+k}}.
\]
Since $s$ is a step function on $\mathcal{P}_{k}, \;\exists\; c_{i} \in \R$ such that $s(x) = c_{i} \;\forall\; x \in (z_{i-1}, z_{i})$. Thus $s(x) = c_{j} \;\forall\; x \in (z_{j-1}, y_{k+1})$ and $s(x) = c_{j} \;\forall\; x \in (y_{k+1}, z_{j})$.
Thus $s$ is constant on all $(z_{i-1}, z_{i}), i \neq j$ as well as $(z_{j-1}, y_{k+1})$ and $(y_{k+1}, z_{j}) \implies s$ is a step function on $\mathcal{P}_{k+1}$. 

Moreover, \begin{align*}
    \int_{\mathcal{P}_{k+1}} s(x) \dd x &= \sum_{i=1}^{j-1} c_{i} (z_{i} - z_{i-1}) + c_{j} (y_{k+1} - z_{j-1}) + c_{j} (z_{j} - y_{k+1}) + \sum_{i=j+1}^{n+k} c_{i} (z_{i} - z_{i-1}) \\
    &= \sum_{i=1}^{j-1} c_{i} (z_{i} - z_{i-1}) + c_{j} (z_{j} - z_{j-1}) + \sum_{i=j+1}^{n+k} c_{i} (z_{i} - z_{i-1}) \\
    &= \sum_{i=1}^{n+k} c_{i} (z_{i} - z_{i-1}) \\
    &= \int_{\mathcal{P}_{k}} s(x) \dd x \\
    &= I.
\end{align*}
Thus by induction, $s$ is a step function on $\mathcal{P}'$ with integral $I$.

Taking the common refinement $\mathcal{R}$ of $\mathcal{P}$ and $\mathcal{Q}$ yields $\int_{\mathcal{P}} s(x) \dd x = \int_{\mathcal{R}} s(x) \dd x = \int_{\mathcal{Q}} s(x) \dd x$.


\section*{Problem 2}
\begin{enumerate}[label=(\alph*)]
    \item \[
        \floor{x-\frac{1}{2}} + \floor{x} = 
        \begin{cases}
            -3 & x \in [-1, -\frac{1}{2}) \\
            -2 & x \in [-\frac{1}{2}, 0) \\
            -1 & x \in [0, \frac{1}{2}) \\
            0 & x \in [\frac{1}{2}, 1) \\
            1 & x \in [1, \frac{3}{2}) \\
            2 & x \in [\frac{3}{2}, 2) \\
            3 & x = 2
        \end{cases}
    \] Thus \[
        \int_{-1}^{2} \paren{\floor{x-\frac{1}{2}} + \floor{x}} \dd x = -3 \cdot \frac{1}{2} + (-2) \cdot \frac{1}{2} + \dots + 2 \cdot \frac{1}{2} = -\frac{3}{2}.
    \] \item \[
        \floor{\sqrt{x}} =
        \begin{cases}
            1 & x \in [1, 4) \\
            2 & x \in [4, 9) \\
            3 & x = 9
        \end{cases}
    \] Thus \[
        \int_{1}^{9} \floor{\sqrt{x}} \dd x = 1 \cdot 3 + 2 \cdot 5 = 13.
    \]
\end{enumerate}


\section*{Problem 3}
Given a step function $f$ on $[a, b]$, we have \[
    S_{f} = \set{\int_{a}^{b} s(x) \dd x : s \text{ is a step function and } s \leq f \text{ on } [a, b]}.
\] For any step function $s \leq f$, we have $\int_{a}^{b} s(x) \dd x \leq \int_{a}^{b} f(x) \dd x$ (defined as sum of $f_{j} (x_{j} - x_{j-1})$.

Thus $\int_{a}^{b} f(x) \dd x$ is an upper bound of $S_{f}$. Moreover, since $f$ is a step function and $f \leq f$ on $[a, b]$, $\int_{a}^{b} f(x) \dd x \in S_{f}$.
Therefore, $\sup S_{f} = \int_{a}^{b} f(x) \dd x$.

Similarly, $\inf T_{f} = \int_{a}^{b} f(x) \dd x$ and so the two definitions are concurrent.

Alternatively, \[
    \int_{a}^{b} s(x) \dd x \leq \sup S_{f} \leq \inf T_{f} \leq \int_{a}^{b} t(x) \dd x.
\] Since $f \leq f$ and $f \geq f$, we can let $s = f$ and $t = f$. So \[
    \int_{a}^{b} f(x) \dd x \leq \sup S_{f} \leq \inf T_{f} \leq \int_{a}^{b} f(x) \dd x.
\] Thus $\sup S_{f} = \inf T_{f} = \int_{a}^{b} f(x) \dd x$.


\section*{Problem 4}
Suppose $f$ is not Riemann integrable on $[c, d]$.
Then $\ubar{I}_{[c, d]} \neq \bar{I}_{[c, d]} \implies \;\exists\; \varepsilon > 0$ such that $\int_{c}^{d} t_{[c, d]}(x) \dd x - \int_{c}^{d} s_{[c, d]}(x) \dd x > \varepsilon$ for all step functions $s_{[c, d]}, t_{[c, d]} : [c, d] \to \R$ such that $s_{[c, d]} \leq f \leq t_{[c, d]}$ on $[c, d]$.

Now suppose $\hat{s}_{[a, b]}, \hat{t}_{[a, b]} : [a, b] \to \R$ are step functions such that $\hat{s}_{[a, b]} \leq f \leq \hat{t}_{[a, b]}$ on $[a, b]$.


\section*{Problem 5}
By the expansion property, \[
    \int_{-0}^{-a} f(-x) \dd x = -\int_{0}^{a} f(x) \dd x.
\]
If $f$ is even, then $f(-x) = f(x)$ and so \[
    \int_{-0}^{-a} f(x) \dd x = -\int_{0}^{a} f(x) \dd x.
\] So $\int_{-a}^{0} f(x) \dd x = \int_{0}^{a} f(x) \dd x$ and thus \[
    \int_{-a}^{a} f(x) \dd x = 2 \int_{0}^{a} f(x) \dd x.
\] Similarly if $f$ is odd, then $f(-x) = -f(x)$ and so \[
    \int_{-0}^{-a} f(x) \dd x = \int_{0}^{a} f(x) \dd x.
\] So $\int_{-a}^{0} f(x) \dd x = -\int_{0}^{a} f(x) \dd x$ and thus \[
    \int_{-a}^{a} f(x) \dd x = 0.
\]


\section*{Problem 6}
Suppose $f$ attains a value $y > 0$ at some $x_{0}$. Since $f$ is continuous, there exists a neighborhood $N_{\delta}(x_{0})$ with $0 < \delta < \min \set{b - x_{0}, x_{0} - a}$ of size $2 \delta$ on which $f > y/2$.
Thus $\int_{x_{0} - \delta}^{x_{0} + \delta} f(x) \dd x > y \delta > 0$.

Since $f \geq 0$, we have $\int_{a}^{x_{0} - \delta} f(x) \dd x \geq 0$ and $\int_{x_{0} + \delta}^{b} f(x) \dd x \geq 0$. Thus \[
    \int_{a}^{b} f(x) \dd x > 0.
\] This is a contradiction, and thus $f(x) = 0 \;\forall\; x \in [a, b]$.


\section*{Problem 7}
Suppose $f$ is Riemann integrable on $[a, b]$ per the lecture definition. Then $\sup S_{f} = \inf T_{f} = I$.
Since $I$ is the supremum of $S_{f}$, $I - \varepsilon$ is not an upper bound of $S_{f}$, and thus there exists an $s_{\varepsilon} \leq f$ such that $\int_{a}^{b} s_{\varepsilon}(x) \dd x > I - \varepsilon$.

Similarly there exists a $t_{\varepsilon} \geq f$ such that $\int_{a}^{b} t_{\varepsilon}(x) \dd x$. Thus $f$ is Riemann integrable as per the given definition.

Now suppose $f$ is Riemann integrable on $[a, b]$ per the given definition. Defining $S_{f}$ and $T_{f}$ as before, we have \[
    \int_{a}^{b} s(x) \dd x \leq \sup S_{f} \leq \inf T_{f} \leq \int_{a}^{b} t(x) \dd x.
\] Suppose $\int_{a}^{b} s(x) \dd x = I + \varepsilon_{0} > I$.
Then there exists a $t_{\varepsilon_{0}}$ such that $\int_{a}^{b} t_{\varepsilon_{0}}(x) \dd x < I + \varepsilon_{0} = \int_{a}^{b} s(x) \dd x$. Contradiction.
Thus $\int_{a}^{b} s(x) \dd x \leq I$.
Also, since $I - \varepsilon$ is not an upper bound of $S_{f}$ for any $\varepsilon > 0$, $I$ is the least upper bound of $S_{f}$.

Similarly, $I$ is the greatest lower bound of $T_{f}$.
Thus $f$ is Riemann integrable per the lecture definition.

\end{document}
