\documentclass[12pt]{article}
\usepackage[utf8]{inputenc}
\usepackage[a4paper, total={6in, 9in}]{geometry}
\usepackage{amsmath}
\usepackage{amssymb}
\usepackage{amsthm}
\usepackage{enumitem}
\usepackage{outlines}

\usepackage{paracol}

\usepackage[dvipsnames]{xcolor}
\colorlet{exercise}{red!80!black}
\colorlet{solved}{green!30!black}
\colorlet{self_proof}{blue!30!black}

\usepackage{mathrsfs}
\usepackage{parskip}

\usepackage{accents}
\newcommand{\ubar}[1]{\underaccent{\bar}{#1}}

\usepackage{hyperref}
\usepackage{cleveref}

\providecommand{\dd}{\,\mathrm{d}}

\usepackage{mathtools} % also contains \coloneqq
\DeclarePairedDelimiter{\oldparen}{(}{)}
\DeclarePairedDelimiter{\oldset}{\{}{\}}
\DeclarePairedDelimiter{\oldabs}{\lvert}{\rvert}
\DeclarePairedDelimiter{\oldnorm}{\lVert}{\rVert}
\DeclarePairedDelimiter{\oldfloor}{\lfloor}{\rfloor}
\DeclarePairedDelimiter{\oldceil}{\lceil}{\rceil}

\makeatletter
\def\paren{\@ifstar{\oldparen}{\oldparen*}}
\def\set{\@ifstar{\oldset}{\oldset*}}
\def\abs{\@ifstar{\oldabs}{\oldabs*}}
\def\norm{\@ifstar{\oldnorm}{\oldnorm*}}
\def\floor{\@ifstar{\oldfloor}{\oldfloor*}}
\def\ceil{\@ifstar{\oldceil}{\oldceil*}}

\g@addto@macro\normalsize{%
  \setlength\abovedisplayskip{7pt}%
  \setlength\belowdisplayskip{7pt}%
  \setlength\abovedisplayshortskip{1pt}%
  \setlength\belowdisplayshortskip{1pt}%
}
\makeatother

\newcommand\N{\ensuremath{\mathbb{N}}}
\newcommand\R{\ensuremath{\mathbb{R}}}
\newcommand\Z{\ensuremath{\mathbb{Z}}}
\renewcommand\O{\ensuremath{\varnothing}}
\renewcommand\P{\ensuremath{\mathbb{P}}}
\newcommand\Q{\ensuremath{\mathbb{Q}}}
\newcommand\C{\ensuremath{\mathbb{C}}}

% make implied by and implies arrows shorter
\let\implies\Rightarrow
\let\impliedby\Leftarrow
\let\iff\Leftrightarrow

% make \epsilon and \varepsilon the same
\let\epsilon\varepsilon

% <theorems>
% \theoremstyle{plain}
% \newtheorem{thm}{Theorem}[section]
% \newtheorem*{thm*}{Theorem}
% \newtheorem{prop}[thm]{Proposition}
% \newtheorem{lem}[thm]{Lemma}
% \newtheorem{cor}[thm]{Corollary}
% \newtheorem{axiom}{Axiom}[section]
% \newtheorem*{exer}{Exercise}
% 
% \theoremstyle{definition}
% \newtheorem{defn}[thm]{Definition}
% 
% \theoremstyle{remark}
% \newtheorem*{rem}{Remark} 
% \newtheorem*{example}{Example}
% </theorems>

% Gilles Castel's theorems
\usepackage[framemethod=Tikz]{mdframed}
\mdfsetup{skipabove=1em,skipbelow=0em}
\mdfdefinestyle{axiomstyle}{
    outerlinewidth = 1.5,
    roundcorner = 10,
    leftmargin = 15,
    rightmargin = 15,
    backgroundcolor = blue!10
}
\mdfdefinestyle{defnstyle}{
    outerlinewidth = 1.5,
    % roundcorner = 10,
    leftmargin = 7,
    rightmargin = 7,
    backgroundcolor = green!10
}
\mdfdefinestyle{thmstyle}{
    outerlinewidth = 1.5,
    roundcorner = 10,
    leftmargin = 7,
    rightmargin = 7,
    backgroundcolor = yellow!10
}
\mdfdefinestyle{lemmastyle}{
    outerlinewidth = 1.5,
    roundcorner = 10,
    leftmargin = 7,
    rightmargin = 7,
    backgroundcolor = yellow!10
}
\theoremstyle{definition}
\newmdtheoremenv[nobreak=true, style=axiomstyle]{axiom}{Axiom}[section]
\newmdtheoremenv[nobreak=true, style=thmstyle]{thm}{Theorem}[section]
\newmdtheoremenv[nobreak=true]{prop}[thm]{Proposition}
\newmdtheoremenv[nobreak=true]{lem}[thm]{Lemma}
\newmdtheoremenv[nobreak=true]{cor}[thm]{Corollary}
\newmdtheoremenv[nobreak=true, style=defnstyle]{defn}[thm]{Definition}

\newcounter{assignment}
\newmdtheoremenv[nobreak=true]{problem}{Problem}[assignment]

\theoremstyle{remark}
\newtheorem*{rem}{Remarks}
\newtheorem*{example}{Example}
\newtheorem*{examplex}{Examples}

\newenvironment{examples}
{\begin{examplex}\leavevmode\begin{itemize}}{\end{itemize}\end{examplex}}
% <cref>
\crefname{thm}{theorem}{theorems}
\crefname{prop}{proposition}{propositions}
\crefname{lem}{lemma}{lemmas}
\crefname{cor}{corollary}{corollaries}
\crefname{axiom}{axiom}{axioms}
\crefname{defn}{definition}{definitions}
\crefname{problem}{problem}{problems}
% </cref>

% <hyperlinks>
\hypersetup{colorlinks,
    linkcolor={red!50!black},
    citecolor={blue!50!black},
    urlcolor={blue!80!black}}
% </hyperlinks>

\usepackage{xifthen}
\def\testdateparts#1{\dateparts#1\relax}
\def\dateparts#1 #2 #3 #4 #5\relax{
    \marginpar{\small\textsf{\mbox{#1 #2 #3 #5}}}
}

\def\@lecture{}%
\newcommand*{\lecture}[3]{
    \ifthenelse{\isempty{#3}}{%
        \def\@lecture{Lecture #1}%
    }{%
        \def\@lecture{Lecture #1: #3}%
    }%
    \subsection*{\@lecture}
    \marginpar{\raggedright\small\textsf{#2}}
    \vskip 6pt
}
\newcommand*{\assignment}[2]{%
    \stepcounter{assignment}%
    \subsection{Assignment #1}
    \marginpar{\raggedright\small{due #2}}
    \vskip 6pt
}
\makeatletter
\newcommand*{\refifdef}[3]{%label,command,fallback
    \@ifundefined{r@#1}{#3}{#2{#1}}%
}
\makeatother

\DeclareMathOperator{\dom}{dom}
\DeclareMathOperator{\ran}{ran}
\DeclareMathOperator{\spann}{span}
\DeclareMathOperator{\dimn}{dim}

\newcommand{\psum}[2]{\ensuremath{\mathrm{sum}_{#1}({#2})}}
\newcommand{\pprod}[2]{\ensuremath{\mathrm{prod}_{#1}({#2})}}


\title{Assignment 04}
\author{Naman Mishra}
\date{16 November 2022}

\begin{document}
\maketitle

\begin{problem}
    Let $\set{a_{n}}$ and $\set{b_{n}}$ be sequences in $\R$ such that for some $N \in \N, 0 \leq a_{n} \leq b_{n}$.
    Convince yourself that if $\lim_{n \to \infty} b_{n} = 0$, then $\lim_{n \to \infty} a_{n} = 0$.
    Using this fact, prove the following statements (you are not allowed to use logarithms for these proofs).
    \begin{enumerate}[label=(\alph*)]
        \item For any $r > 0$, $\lim\limits_{n \to \infty} \sqrt[n]{r} = 1$.
        \item $\lim\limits_{n \to \infty} \sqrt[n]{n} = 1$.
    \end{enumerate}
\end{problem}

\begin{proof}
    \begin{enumerate}[label=(\alph*)]
        \item For $r > 1$: By the Archimedean property, there exists $n_{0} \in \P$ such that $n_{0} \varepsilon > r - 1$ for all $\varepsilon > 0$.
        Thus for all $n \geq n_{0}$, $n \varepsilon > r - 1 \implies r < 1 + n \varepsilon \leq (1 + \varepsilon)^{n} \implies r^{1/n} < 1 + \varepsilon$.
        $r > 1 \implies r^{1/n} > 1 > 1 - \varepsilon$.
        Thus $\abs{r^{1 / n} - 1} < \varepsilon \;\forall\; n \geq n_{0}$.

        For $r = 1$, $\abs{r^{1/n} - 1} = 0 < \varepsilon$ for all $n \geq  1, \varepsilon > 0$.

        For $r < 1$, \[
            r^{1/n} = \frac{1}{\paren{\frac{1}{r}}^{1/n}}
        \] Since $\frac{1}{r} > 1$, by limit laws for sequnces, $\lim\limits_{n \to \infty} r^{1/n} = \frac{1}{1} = 1$.
        
        \item By the Archimedean property, there exists an $N \in \P > \frac{2}{\varepsilon^{2}} + 1$. For $n \geq N$, we have
        \begin{align*}
            \frac{n - 1}{2} \cdot \varepsilon^{2} &> 1 \\
            \frac{n(n - 1)}{2} \cdot \varepsilon^{2} &> n \\
            (1 + \varepsilon)^{n} &> n \\
            \sqrt[n]{n} &< 1 + \varepsilon
        \end{align*}
        Also since $n \geq 1$, $n^{1 / n} \geq 1 \implies \sqrt[n]{n} > 1 - \varepsilon$. \qedhere
    \end{enumerate}
\end{proof}

\begin{problem}
    Show that the series $\sum\limits_{n=0}^{\infty} \frac{1}{n!}$ converges.
    The mathematical constant $e$ is defined as the sum of this series.
\end{problem}
\begin{proof}
    Ratio test.
\end{proof}

\begin{problem}
    Let $\set{a_{n} : n \in \P}$ be an arbitrary collection of non-negative real numbers such that $\sum\limits_{n=1}^{\infty} a_{n}$ converges.
    Determine which of the following series will necessarily converge (proof required), and which may either converge or diverge depending on the choice of the $a_{n}$'s (examples required).
    \begin{enumerate}[label=(\alph*)]
        \item $\sum\limits_{n=1}^{\infty} a_{n}^{2}$
        \item $\sum\limits_{n=1}^{\infty} \sqrt{a_{n}}$
        \item $\sum\limits_{n=1}^{\infty} \frac{\sqrt{a_{n}}}{n}$
    \end{enumerate}
\end{problem}

\begin{proof}
    \begin{enumerate}[label=(\alph*)]
        \item Since $\sum_{n=1}^{\infty} a_{n}$ converges, $\lim\limits_{n \to \infty} a_{n} = 0$. Thus there exists $N \in \P$ such that $a_{n} < 1 \;\forall\; n \geq N \implies a_{n}^{2} < a_{n} \;\forall\; n \geq N$. By the comparison test, $\sum_{n=1}^{\infty} a_{n}^{2}$ converges.

        \item $\sum \frac{1}{n^{2}}$ converges but $\sum \frac{1}{n}$ diverges. $\sum 0$ converges and so does $\sum 0$. Inconclusive.

        \item Let $b_{n} = a_{n} + \frac{1}{n^{2}}$. By the limit laws for series, $\sum b_{n}$ converges. \[
            \frac{\sqrt{a_{n}}}{n} \leq \frac{\sqrt{b_{n}}}{n} = \frac{b_{n}}{n \sqrt{b_{n}}} \leq \frac{b_{n}}{n \sqrt{\frac{1}{n^{2}}}} = b_{n}.
        \] Thus by the comparison test, $\sum\limits_{n=1}^{\infty} \frac{\sqrt{a_{n}}}{n}$ converges. \qedhere
    \end{enumerate}
\end{proof}

\begin{problem}
    Show that each of the following series converges, and determine its sum.
    \begin{enumerate}[label=(\alph*)]
        \item $\sum\limits_{n=1}^{\infty} \frac{4n^{2} - 1 + 3^{n-1}}{3^{n}(2n+1)(2n-1)}$
        \item $\sum\limits_{n=6}^{\infty} \frac{6}{n^{2} - 1}$
        \item $\sum\limits_{n=1}^{\infty} \frac{n}{(n + 1)(n + 2)(n + 3)}$
    \end{enumerate}
\end{problem}

\begin{proof} \leavevmode
    \begin{enumerate}[label=(\alph*)]
        \item
        \begin{align*}
            \frac{4n^{2} - 1 + 3^{n-1}}{3^{n} (2n + 1)(2n - 1)} &= \frac{1}{3^{n}} + \frac{1}{3} \cdot \frac{1}{(2n + 1)(2n - 1)} \\
            &= \frac{1}{3^{n}} + \frac{1}{6} \cdot \paren{\frac{1}{2n-1} - \frac{1}{2n+1}} \\
            \\
            \sum_{n=1}^{\infty} \frac{1}{3^{n}} = \frac{1}{3} \sum_{n=0}^{\infty} \frac{1}{3^{n}} &= \frac{1 / 3}{1-\frac{1}{3}} = \frac{1}{2} \\
            \sum_{n=1}^{\infty} \frac{1}{2n-1} - \frac{1}{2n+1} &= \frac{1}{2n-1} - \frac{1}{2(n+1) - 1} \\
            &= \frac{1}{1} - \frac{1}{3} + \frac{1}{3} - \frac{1}{5} + \frac{1}{5} - \dots \\
            &= 1
        \end{align*}
        Thus by the limit laws,
        \begin{align*}
            \sum_{n=1}^{\infty} \frac{4n^{2} - 1 + 3^{n-1}}{3^{n} (2n + 1)(2n - 1)} &= \frac{1}{2} + \frac{1}{6} \cdot 1 \\
            &= \frac{2}{3}
        \end{align*}

        \item
        \begin{align*}
            \frac{6}{n^{2} - 1} &= \frac{3 \cdot 2}{(n - 1)(n + 1)} \\
            &= 3 \paren{\frac{1}{n - 1} - \frac{1}{n + 1}} \\
            &= 3 \paren{\frac{1}{n - 1} - \frac{1}{(n + 2) - 1}} \\
        \end{align*}
        Let $\set{s_{n}}$ be the sops of $\frac{1}{n-1} - \frac{1}{n+1}$. For $n > 5$,
        \begin{align*}
            s_{n} &= \frac{1}{5} - \frac{1}{7} + \frac{1}{6} - \frac{1}{8} + \frac{1}{7} - \frac{1}{9} + \dots + \frac{1}{n-1} + \frac{1}{n + 1} \\
            &= \frac{1}{5} + \frac{1}{6} - \frac{1}{n} - \frac{1}{n+1} \\
            \lim_{n \to \infty} s_{n} &= \frac{1}{5} + \frac{1}{6} \\
            &= \frac{11}{30} \\
            \\
            \implies \sum_{n=6}^{\infty} \frac{6}{n^{2} - 1} &= \frac{11}{10}
        \end{align*}

        \item
        \begin{align*}
            \frac{n}{(n + 1)(n + 2)(n + 3)} &= \frac{1}{2}\frac{3(n + 1) - (n + 3)}{(n + 1)(n + 2)(n + 3)} \\
            &= \frac{3}{2} \frac{1}{(n + 2)(n + 3)} - \frac{1}{2} \frac{1}{(n + 1)(n + 2)}\\
            &= \frac{3}{2} \paren{\frac{1}{n + 2} - \frac{1}{n + 3}} - \frac{1}{2} \paren{\frac{1}{n + 1} - \frac{1}{n + 2}}
        \end{align*}
        Let $s_{n}$ and $t_{n}$ be the sops of $\set{\frac{1}{n+2} - \frac{1}{n+3}}$ and $\set{\frac{1}{n+1} - \frac{1}{n+2}}$.
        \begin{align*}
            s_{n} &= \frac{1}{3} - \frac{1}{4} + \frac{1}{4} - \frac{1}{5} + \dots + \frac{1}{n} - \frac{1}{n+1} + \frac{1}{n+1} - \frac{1}{n+2} \\
            s_{n} &= \frac{1}{3} - \frac{1}{n+2} \\
            \lim_{n \to \infty} s_{n} &= \frac{1}{3} \\
            t_{n} &= \frac{1}{2} - \frac{1}{3} + \frac{1}{3} -\frac{1}{4} + \dots + \frac{1}{n + 1} - \frac{1}{n + 2} + \frac{1}{n + 2} - \frac{1}{n + 3}\\
            t_{n} &= \frac{1}{2} - \frac{1}{n + 3} \\
            \lim_{n \to \infty} t_{n} &= \frac{1}{2} \\
            \\
            \sum_{n=1}^{\infty} \frac{n}{(n + 1)(n + 2)(n + 3)} &= \frac{3}{2} \cdot \frac{1}{3} - \frac{1}{2} \cdot \frac{1}{2} \\
            &= \frac{1}{4} \qedhere
        \end{align*}
    \end{enumerate}
\end{proof}

\begin{problem}
    For each of the series given below, determine whether it converges or diverges.
    You need not compute the sum in the case of convergence.
    \begin{enumerate}[label=(\arabic*)]
        \item $\sum\limits_{n=1}^{\infty} \frac{n \sin^{2}(n\pi/3)}{2^{n}}$
        \item $\sum\limits_{n=1}^{\infty} \paren{\frac{1}{n}}^{\frac{1}{n}}$
        \item $\sum\limits_{n=1}^{\infty} \frac{(-1)^{n}n^{25}}{(n+2)!}$
        \item $\sum\limits_{n=5}^{\infty} \frac{\sqrt{n}+1}{(n-1)(n+2)(n-4)}$
    \end{enumerate}
\end{problem}

\begin{proof}
    \begin{enumerate}[label=(\arabic*)]
        \item Let $b_{n} = \frac{n}{2^{n}}$. \[
            \frac{b_{n+1}}{b_{n}} = \frac{n+1}{2^{n+1}} \cdot \frac{2^{n}}{n} = \frac{1}{2} \cdot \frac{n+1}{n}
        \] \[
            \lim_{n \to \infty} \frac{b_{n+1}}{b_{n}} = \frac{1}{2}
        \] Thus by the ratio test, $\sum b_{n}$ converges. Since $a_{n} < b_{n}$, $\sum a_{n}$ converges by the comparison test.

        \item $n \geq 1 \implies \frac{1}{n} \leq 1 \land n^{\frac{1}{n}} \leq  n^{1} \implies \paren{\frac{1}{n}}^{\frac{1}{n}} \geq \frac{1}{n}$.
        By the comparison test, $\sum_{n=1}^{\infty}\paren{\frac{1}{n}}^{\frac{1}{n}}$ diverges.

        \item Let $a_{n} = \frac{n^{25}}{(n + 2)!}$.
        Clearly $a_{n} > 0$. \[
            0 < \frac{a_{n+1}}{a_{n}} = \paren{1 + \frac{1}{n}}^{25} \cdot \frac{1}{n + 3} \leq 2^{25} \cdot \frac{1}{n + 3}
        \] For any $\varepsilon > 0$, there exists (by the Archimedean property), $N > \frac{2^{25}}{\varepsilon} - 3$.
        For $n \geq N, -\varepsilon < 0 < \frac{a_{n+1}}{a_{n}} \leq \frac{2^{25}}{n + 3} \leq \frac{2^{25}}{N + 3} < \varepsilon$.
        Thus the ratio converges to 0.
        By the ratio test, $\sum_{n=1}^{\infty} a_{n}$ converges.

        Since the given series converges absolutely, it converges.

        \item For $n \geq 5$, 
        \begin{align*}
            0 < \frac{\sqrt{n} + 1}{(n - 1)(n + 2)(n - 4)} &= \frac{1}{(\sqrt{n} - 1)(n + 2)(n - 4)} \\
            &< \frac{1}{(n + 2)(n - 4)} \\
            &= \frac{1}{n^{2} - 2n - 8} \\
            &< \frac{1}{n^{2} - 4n} \\
            &= \frac{1}{\frac{1}{5}n^{2} + \frac{4}{5} n (n - 5)} \\
            &\leq \frac{5}{n^{2}}
        \end{align*}
        So by the comparison test, $\sum\limits_{n=5}^{\infty}\frac{\sqrt{n} + 1}{(n - 1)(n + 2)(n - 4)}$ converges. \qedhere
    \end{enumerate}
\end{proof}
\end{document}
