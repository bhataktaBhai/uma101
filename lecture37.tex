\lecture{37}{Wed 25 Jan '23}{}

\begin{defn}[Span of sets] \label{defn:vector:span:set}
    Let $S \subseteq V$ be a nonempty set. The \emph{span} of the $S$ is the set \begin{align*}
        \spann S = \{v \in V : \;\exists\; &a_1, \dots, a_n \in F \\
        \text{and distinct } &v_1, \dots, v_n \in S \\
        \text{such that } &v = a_1 v_1 + \cdots + a_n v_n\}.
    \end{align*}
    Or \[
        \spann S = \bigcup_{\O \neq \Lambda \subseteq ^{\mathrm{finite}} S} \spann \Lambda.
    \]
    $\spann \O$ is defined to be $\set{0}$.
\end{defn}

\begin{example}
    \begin{enumerate}[label=(\alph*)]
        \item $\spann(0) = \set{0}$.
        \item Let $V \in \R^{2}$.
        Let $(a, b) \neq (0, 0)$.
        Then $\spann (a, b) = \set{(x, y) \in \R^{2} : ay - bx = 0}$.

    \end{enumerate}
\end{example}

\begin{prop}[Spans are subspaces] \label{thm:vector:span:subspace}
    Let $S \subseteq V$.
    Then $\spann S$ is a subspace of $V$.
    If $W$ is a subspace of $V$ and $S \subseteq W$, then \[
        \spann S \subseteq W.
    \]
\end{prop}
\begin{proof}
    
\end{proof}

\begin{defn}[finite and infinite dimensions] \label{defn:vector:fdvs}
    Let $V$ be a vector space over $F$.
    \begin{enumerate}[label=(\alph*)]
        \item $V$ is \emph{finite-dimensional} if there exists a finite set $S \subseteq V$ such that $V = \spann S$.
        \item $V$ is \emph{infinite-dimensional} if it is not finite-dimensional.
    \end{enumerate}
\end{defn}
