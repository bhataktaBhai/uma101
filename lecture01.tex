\lecture{1}{mon 17 oct '22}{Introduction; Peano Sets}

\part{Analysis}
\section{Set theory \& the real number system}

\begin{defn} \label{defn:peano}
    The set $A$ along with a successor function $S$ is called a Peano set if it obeys the Peano axioms.
    \begin{enumerate}[label=(P\arabic*)]
        \item \label{defn:peano:zero}
            There is an element called 0 in $A$.
        \item \label{defn:peano:succ}
            For every $a \in A$, its successor $S(a)$ is also in $A$.
        \item \label{defn:peano:not_succ}
            $\forall \; a \in A, S(a) \neq 0$.
        \item \label{defn:peano:injective}
            For any $m, n \in A$, $S(m) = S(n)$ only if $m = n$.
        \item \label{defn:peano:induction}
            (principle of mathematical induction) For any set $B \subseteq A$, if $0 \in B$ and $a \in B \implies S(a) \in B$, then $B = A$.
    \end{enumerate}
\end{defn}
