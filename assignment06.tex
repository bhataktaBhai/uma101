\documentclass[12pt]{article}
\usepackage[utf8]{inputenc}
\usepackage[a4paper, total={6in, 9in}]{geometry}
\usepackage{amsmath}
\usepackage{amssymb}
\usepackage{amsthm}
\usepackage{enumitem}
\usepackage{outlines}

\usepackage{paracol}

\usepackage[dvipsnames]{xcolor}
\colorlet{exercise}{red!80!black}
\colorlet{solved}{green!30!black}
\colorlet{self_proof}{blue!30!black}

\usepackage{mathrsfs}
\usepackage{parskip}

\usepackage{accents}
\newcommand{\ubar}[1]{\underaccent{\bar}{#1}}

\usepackage{hyperref}
\usepackage{cleveref}

\providecommand{\dd}{\,\mathrm{d}}

\usepackage{mathtools} % also contains \coloneqq
\DeclarePairedDelimiter{\oldparen}{(}{)}
\DeclarePairedDelimiter{\oldset}{\{}{\}}
\DeclarePairedDelimiter{\oldabs}{\lvert}{\rvert}
\DeclarePairedDelimiter{\oldnorm}{\lVert}{\rVert}
\DeclarePairedDelimiter{\oldfloor}{\lfloor}{\rfloor}
\DeclarePairedDelimiter{\oldceil}{\lceil}{\rceil}

\makeatletter
\def\paren{\@ifstar{\oldparen}{\oldparen*}}
\def\set{\@ifstar{\oldset}{\oldset*}}
\def\abs{\@ifstar{\oldabs}{\oldabs*}}
\def\norm{\@ifstar{\oldnorm}{\oldnorm*}}
\def\floor{\@ifstar{\oldfloor}{\oldfloor*}}
\def\ceil{\@ifstar{\oldceil}{\oldceil*}}

\g@addto@macro\normalsize{%
  \setlength\abovedisplayskip{7pt}%
  \setlength\belowdisplayskip{7pt}%
  \setlength\abovedisplayshortskip{1pt}%
  \setlength\belowdisplayshortskip{1pt}%
}
\makeatother

\newcommand\N{\ensuremath{\mathbb{N}}}
\newcommand\R{\ensuremath{\mathbb{R}}}
\newcommand\Z{\ensuremath{\mathbb{Z}}}
\renewcommand\O{\ensuremath{\varnothing}}
\renewcommand\P{\ensuremath{\mathbb{P}}}
\newcommand\Q{\ensuremath{\mathbb{Q}}}
\newcommand\C{\ensuremath{\mathbb{C}}}

% make implied by and implies arrows shorter
\let\implies\Rightarrow
\let\impliedby\Leftarrow
\let\iff\Leftrightarrow

% make \epsilon and \varepsilon the same
\let\epsilon\varepsilon

% <theorems>
% \theoremstyle{plain}
% \newtheorem{thm}{Theorem}[section]
% \newtheorem*{thm*}{Theorem}
% \newtheorem{prop}[thm]{Proposition}
% \newtheorem{lem}[thm]{Lemma}
% \newtheorem{cor}[thm]{Corollary}
% \newtheorem{axiom}{Axiom}[section]
% \newtheorem*{exer}{Exercise}
% 
% \theoremstyle{definition}
% \newtheorem{defn}[thm]{Definition}
% 
% \theoremstyle{remark}
% \newtheorem*{rem}{Remark} 
% \newtheorem*{example}{Example}
% </theorems>

% Gilles Castel's theorems
\usepackage[framemethod=Tikz]{mdframed}
\mdfsetup{skipabove=1em,skipbelow=0em}
\mdfdefinestyle{axiomstyle}{
    outerlinewidth = 1.5,
    roundcorner = 10,
    leftmargin = 15,
    rightmargin = 15,
    backgroundcolor = blue!10
}
\mdfdefinestyle{defnstyle}{
    outerlinewidth = 1.5,
    % roundcorner = 10,
    leftmargin = 7,
    rightmargin = 7,
    backgroundcolor = green!10
}
\mdfdefinestyle{thmstyle}{
    outerlinewidth = 1.5,
    roundcorner = 10,
    leftmargin = 7,
    rightmargin = 7,
    backgroundcolor = yellow!10
}
\mdfdefinestyle{lemmastyle}{
    outerlinewidth = 1.5,
    roundcorner = 10,
    leftmargin = 7,
    rightmargin = 7,
    backgroundcolor = yellow!10
}
\theoremstyle{definition}
\newmdtheoremenv[nobreak=true, style=axiomstyle]{axiom}{Axiom}[section]
\newmdtheoremenv[nobreak=true, style=thmstyle]{thm}{Theorem}[section]
\newmdtheoremenv[nobreak=true]{prop}[thm]{Proposition}
\newmdtheoremenv[nobreak=true]{lem}[thm]{Lemma}
\newmdtheoremenv[nobreak=true]{cor}[thm]{Corollary}
\newmdtheoremenv[nobreak=true, style=defnstyle]{defn}[thm]{Definition}

\newcounter{assignment}
\newmdtheoremenv[nobreak=true]{problem}{Problem}[assignment]

\theoremstyle{remark}
\newtheorem*{rem}{Remarks}
\newtheorem*{example}{Example}

% <cref>
\crefname{thm}{theorem}{theorems}
\crefname{prop}{proposition}{propositions}
\crefname{lem}{lemma}{lemmas}
\crefname{cor}{corollary}{corollaries}
\crefname{axiom}{axiom}{axioms}
\crefname{defn}{definition}{definitions}
% </cref>

% <hyperlinks>
\hypersetup{colorlinks,
    linkcolor={red!50!black},
    citecolor={blue!50!black},
    urlcolor={blue!80!black}}
% </hyperlinks>

\usepackage{xifthen}
\def\testdateparts#1{\dateparts#1\relax}
\def\dateparts#1 #2 #3 #4 #5\relax{
    \marginpar{\small\textsf{\mbox{#1 #2 #3 #5}}}
}

\def\@lecture{}%
\newcommand*{\lecture}[3]{
    \ifthenelse{\isempty{#3}}{%
        \def\@lecture{Lecture #1}%
    }{%
        \def\@lecture{Lecture #1: #3}%
    }%
    \subsection*{\@lecture}
    \marginpar{\small\textsf{\vbox{#2}}}
    \vskip 6pt
}
\newcommand*{\assignment}[2]{%
    \stepcounter{assignment}%
    \subsection*{Assignment #1}
    \marginpar{\small\vbox{due #2}}
    \vskip 6pt
}
\makeatletter
\newcommand*{\refifdef}[3]{%label,command,fallback
    \@ifundefined{r@#1}{#3}{#2{#1}}%
}
\makeatother

\DeclareMathOperator{\dom}{dom}
\DeclareMathOperator{\ran}{ran}
\DeclareMathOperator{\spann}{span}
\DeclareMathOperator{\dimn}{dim}

\newcommand{\psum}[2]{\ensuremath{\mathrm{sum}_{#1}({#2})}}
\newcommand{\pprod}[2]{\ensuremath{\mathrm{prod}_{#1}({#2})}}

\newlist{examples}{enumerate}{1}
\setlist[examples]{label*=(\alph*)~,ref=(\alph*)}
\makeatletter
\newcommand\myitem[1][]{%
  \if\relax\detokenize{#1}\relax
    \item\relax
  \else
    \protected@edef\@currentlabel{#1}%
    \item[(#1)~]
  \fi}
\makeatother


\title{Assignment 06}
\author{Naman Mishra}
\date{24 November 2022}

\begin{document}
\maketitle

% Problem 1
\begin{problem}
    Give an example each of
    \begin{enumerate}[label=(\alph*)]
        \item a bounded function $f : [-1, 1] \to \R$ that does not achieve either its minimum or its maximum on $[-1, 1]$;
        \item a bounded continuous function $f : (-1, 1) \to \R$ that achieves its minimum but not its maximum on $(-1, 1)$.
    \end{enumerate}
\end{problem}
\begin{enumerate}[label=(\alph*)]
    \item $f(x) =
        \begin{cases}
            x & x \not\in \Z \\
            0 & x \in \Z
        \end{cases}$
    \item $f(x) = x^{2}$ \qedhere
\end{enumerate}

% Problem 2
\begin{problem}
    Let $f$ be a continuous function on $[a, b]$ such that $f(x) > 0$ for all $x \in [a, b]$.
    Show that there is a $c > 0$ such that $f(x) \geq c$ for all $x \in [a, b]$.
\end{problem}
\begin{proof}
    Since $f$ is continuous and non-zero on $[a, b]$, $1 / f$ is continuous on $[a, b]$.
    Since $1 / f$ is continuous, it is bounded on $[a, b]$ (\cref{thm:cont:compact->bounded}).
    Thus $f(x) \geq c = \frac{1}{M} \;\forall\; x \in [a, b]$ where $M$ is the supremum of $1 / f$ on $[a, b]$.
\end{proof}

\textbf{Problem 3.}
Let the polynomial be $f(x) = a_{0} + a_{1}x + a_{2}x^{2} + \dots + a_{n}x^{n}$, where $n$ is odd. Define \[
    g(x) = \frac{f(x)}{x^{n}} = \frac{a_{0}}{x^{n}} + \frac{a_{1}}{x^{n-1}} + \dots + a_{n}.
\] Define the sequence $\set{g(m)}_{m \in \N}$. We have \[
    \lim_{m \to \infty} g(m) = a_{n}.
\] Thus we have $N \in \N$ such that $\abs{ g(m) - a_{n} } < a_{n} \;\forall\; m \geq N \implies g(N) > 0 \implies f(N) = N^{n} g(N) > 0$. 

Define the sequence $\set{g(-m)}_{m \in \N}$. We have
\begin{align*}
    g(-m) &= -\frac{a_{0}}{m^{n}} + \frac{a_{1}}{m^{n-1}} - \dots + a_{n} \\
    \implies \lim_{m \to \infty} g(-m) &= a_{n}
\end{align*}
Thus we have $N' \in \N$ such that $\abs{ g(-m) - a_{n} } < a_{n} \;\forall\; m \geq N' \implies g(-N') > 0 \implies f(-N') = (-N')^{n} g(-N) = - (N')^{n} g(-N) < 0$. Also $-N' < 0 < N$. By IVT we have \[
    \exists\; c \in [-N', N] : f(c) = 0.
\]

\textbf{Problem 4.}
Consider $g(x) = \cos x - x^{2}$. Then $g'(x) = - \sin x - 2x < 0$. So $g$ is decreasing. $g(0) = 1 > 0$. $g(\frac{\pi}{2}) = -\frac{\pi^{2}}{4} < 0$. Therefore $g(x) = 0$ at exactly one $x = c$ in $[0, \frac{\pi}{2}]$. Thus $g(x) > 0$ for $0 \leq x < c$ and $g(x) < 0$ for $c < x \leq \frac{\pi}{2}$.

For $0 \leq x < c$, $f(x) = \cos x$. For $c < x \leq \frac{\pi}{2}$, $f(x) = x^{2}$.

Since $\cos x$ is decreasing in $[0, c] \subseteq [0, \frac{\pi}{2}]$, we have $f(x) > f(c) \;\forall\; x \in [0, c)$. Since $x^{2}$ is increasing in $[c, \frac{\pi}{2}] \subseteq [0, \frac{\pi}{2}]$, we have $f(x) > f(c) \;\forall\; x \in (c, \frac{\pi}{2}]$

Thus $f$ attains a global minimum at $c$, where $\cos c = c^{2}$.

\textbf{Problem 5.}
\begin{enumerate}[label=(\alph*)]
    \item \[
        h \circ g (x) = \abs{ x }^{3} =
        \begin{cases}
            x^{3} & x \geq 0 \\
            -x^{3} & x < 0
        \end{cases}
    \] Since all polynomials are continuous and differentiable at every point, we only need to check at 0.
    \begin{align*}
        \lim_{h \to 0} \frac{f(0 + h) - f(0)}{h} &= \lim_{h \to 0} \frac{\abs{ h }^{3}}{h} \\
        &= \lim_{h \to 0} \frac{h^{2}\abs{ h }}{h} \\
        &= \lim_{h \to 0} h \abs{ h } \\
        &= 0
    \end{align*}
    Thus $h \circ g$ is differentiable everywhere, so it is continuous everywhere.

    \item We first show that $\cos(\frac{1}{x})$ is differentiable at $x \neq 0$.
    \begin{align*}
        \lim_{x \to p} \frac{\cos(\frac{1}{x}) - \cos(\frac{1}{p})}{x - p} &= \lim_{x \to p} \frac{2 \sin(\frac{1}{2x} + \frac{1}{2p})\sin(\frac{1}{2p} - \frac{1}{2x})}{x - p} \\
        &= 2 \sin(\frac{1}{p}) \lim_{x \to p} \frac{\sin(\frac{x - p}{2xp})}{x - p} \\
        &= 2 \sin(\frac{1}{p}) \lim_{x \to p} \frac{\sin(\frac{x - p}{2xp})}{\frac{x - p}{2xp}} \frac{1}{2xp} \\
        &= \frac{1}{p^{2}} \sin(\frac{1}{p})
    \end{align*}
    The limit exists, so $\cos(\frac{1}{x})$ is differentiable everywhere in its domain.

    By algebra laws we have $x^{2} \cos(\frac{1}{x})$ also differentiable, as is $\abs{ x } = -x \;\forall\; x < 0$. So we only need to worry about 0.

    In the neighbourhood $(-1, 1)$ about 0, we have $-x^{2} \leq f(x) \leq \abs{ x }$. By the squeeze theorem, $f(x) + x^{2}$ tends to 0. By the limit laws, $\lim_{x \to 0} f(x) = 0$. Thus $f$ is continuous everywhere.

    \begin{align*}
        f(0 + h) - f(0) &= 
        \begin{cases}
            -h & h < 0 \\
            h^{2} \cos(\frac{1}{h}) & h > 0
        \end{cases} \\
        \frac{f(0 + h) - f(0)}{h} &=
        \begin{cases}
            -1 & h < 0 \\
            h \cos(\frac{1}{h}) & h > 0
        \end{cases} \\
    \end{align*}
    Let $\varepsilon = \frac{1}{4}$. For any $\delta > 0$, choose $k = \min \set{\frac{1}{2}, \frac{\delta}{2}}$. $-k \leq f(k) \leq k \implies f(k) > -k > -\frac{1}{2}$ and $\abs{ k - 0 } < \delta$. Also $f(-k) = -1$. For any $L$, $\abs{ f(k) - L } + \abs{ f(-k) - L } \geq \abs{ f(k) - f(-k) } = \abs{ f(k) + 1 } \geq \frac{1}{2} = 2 \varepsilon$. Thus the limit does not exist and so the function is not differentiable at 0.

    \item $\abs{ \sin x } = \abs{ \sin \abs{ x } }$. So \[
        f(x) = 
        \begin{cases}
            \frac{\abs{ \sin \abs{ x } }}{\sin \abs{ x }} & x \neq n \pi \\
            0 & x = n \pi
        \end{cases}
    \] or \[
        f(x) =
        \begin{cases}
            1 & \sin \abs{ x } > 0 \\
            0 & \sin \abs{ x } = 0 \\
            -1 & \sin \abs{ x } < 0
        \end{cases}
    \] Since constant functions are continuous and differentiable, $f(x)$ is differentiable in any region where $\sin \abs{ x }$ is constant in sign. Thus $f(x)$ is continuous and differentiable in all intervals $(n\pi, (n+1)\pi), n \in \Z$. This leaves only the points $n\pi$, where the function is neither continuous nor differentiable, as \[
        \lim_{x \to n\pi} f(x) \text{ does not exist.}
    \] Suppose limit exists and is equal to $L$. Then 
\end{enumerate}

% Problem 4, 5(b)
\end{document}




