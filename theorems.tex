\documentclass[12pt]{article}
\usepackage[utf8]{inputenc}
\usepackage[a4paper, total={6in, 9in}]{geometry}
\usepackage{amsmath}
\usepackage{amssymb}
\usepackage{amsthm}
\usepackage{enumitem}
\usepackage{outlines}

\usepackage{paracol}

\usepackage[dvipsnames]{xcolor}
\colorlet{exercise}{red!80!black}
\colorlet{solved}{green!30!black}
\colorlet{self_proof}{blue!30!black}

\usepackage{mathrsfs}
\usepackage{parskip}

\usepackage{accents}
\newcommand{\ubar}[1]{\underaccent{\bar}{#1}}

\usepackage{hyperref}
\usepackage{cleveref}

\providecommand{\dd}{\,\mathrm{d}}

\usepackage{mathtools} % also contains \coloneqq
\DeclarePairedDelimiter{\oldparen}{(}{)}
\DeclarePairedDelimiter{\oldset}{\{}{\}}
\DeclarePairedDelimiter{\oldabs}{\lvert}{\rvert}
\DeclarePairedDelimiter{\oldnorm}{\lVert}{\rVert}
\DeclarePairedDelimiter{\oldfloor}{\lfloor}{\rfloor}
\DeclarePairedDelimiter{\oldceil}{\lceil}{\rceil}

\makeatletter
\def\paren{\@ifstar{\oldparen}{\oldparen*}}
\def\set{\@ifstar{\oldset}{\oldset*}}
\def\abs{\@ifstar{\oldabs}{\oldabs*}}
\def\norm{\@ifstar{\oldnorm}{\oldnorm*}}
\def\floor{\@ifstar{\oldfloor}{\oldfloor*}}
\def\ceil{\@ifstar{\oldceil}{\oldceil*}}

\g@addto@macro\normalsize{%
  \setlength\abovedisplayskip{7pt}%
  \setlength\belowdisplayskip{7pt}%
  \setlength\abovedisplayshortskip{1pt}%
  \setlength\belowdisplayshortskip{1pt}%
}
\makeatother

\newcommand\N{\ensuremath{\mathbb{N}}}
\newcommand\R{\ensuremath{\mathbb{R}}}
\newcommand\Z{\ensuremath{\mathbb{Z}}}
\renewcommand\O{\ensuremath{\varnothing}}
\renewcommand\P{\ensuremath{\mathbb{P}}}
\newcommand\Q{\ensuremath{\mathbb{Q}}}
\newcommand\C{\ensuremath{\mathbb{C}}}

% make implied by and implies arrows shorter
\let\implies\Rightarrow
\let\impliedby\Leftarrow
\let\iff\Leftrightarrow

% make \epsilon and \varepsilon the same
\let\epsilon\varepsilon

% <theorems>
% \theoremstyle{plain}
% \newtheorem{thm}{Theorem}[section]
% \newtheorem*{thm*}{Theorem}
% \newtheorem{prop}[thm]{Proposition}
% \newtheorem{lem}[thm]{Lemma}
% \newtheorem{cor}[thm]{Corollary}
% \newtheorem{axiom}{Axiom}[section]
% \newtheorem*{exer}{Exercise}
% 
% \theoremstyle{definition}
% \newtheorem{defn}[thm]{Definition}
% 
% \theoremstyle{remark}
% \newtheorem*{rem}{Remark} 
% \newtheorem*{example}{Example}
% </theorems>

% Gilles Castel's theorems
\usepackage[framemethod=Tikz]{mdframed}
\mdfsetup{skipabove=1em,skipbelow=0em}
\mdfdefinestyle{axiomstyle}{
    outerlinewidth = 1.5,
    roundcorner = 10,
    leftmargin = 15,
    rightmargin = 15,
    backgroundcolor = blue!10
}
\mdfdefinestyle{defnstyle}{
    outerlinewidth = 1.5,
    % roundcorner = 10,
    leftmargin = 7,
    rightmargin = 7,
    backgroundcolor = green!10
}
\mdfdefinestyle{thmstyle}{
    outerlinewidth = 1.5,
    roundcorner = 10,
    leftmargin = 7,
    rightmargin = 7,
    backgroundcolor = yellow!10
}
\mdfdefinestyle{lemmastyle}{
    outerlinewidth = 1.5,
    roundcorner = 10,
    leftmargin = 7,
    rightmargin = 7,
    backgroundcolor = yellow!10
}
\theoremstyle{definition}
\newmdtheoremenv[nobreak=true, style=axiomstyle]{axiom}{Axiom}[section]
\newmdtheoremenv[nobreak=true, style=thmstyle]{thm}{Theorem}[section]
\newmdtheoremenv[nobreak=true]{prop}[thm]{Proposition}
\newmdtheoremenv[nobreak=true]{lem}[thm]{Lemma}
\newmdtheoremenv[nobreak=true]{cor}[thm]{Corollary}
\newmdtheoremenv[nobreak=true, style=defnstyle]{defn}[thm]{Definition}

\newcounter{assignment}
\newmdtheoremenv[nobreak=true]{problem}{Problem}[assignment]

\theoremstyle{remark}
\newtheorem*{rem}{Remarks}
\newtheorem*{example}{Example}
\newtheorem*{examplex}{Examples}

\newenvironment{examples}
{\begin{examplex}\leavevmode\begin{itemize}}{\end{itemize}\end{examplex}}
% <cref>
\crefname{thm}{theorem}{theorems}
\crefname{prop}{proposition}{propositions}
\crefname{lem}{lemma}{lemmas}
\crefname{cor}{corollary}{corollaries}
\crefname{axiom}{axiom}{axioms}
\crefname{defn}{definition}{definitions}
\crefname{problem}{problem}{problems}
% </cref>

% <hyperlinks>
\hypersetup{colorlinks,
    linkcolor={red!50!black},
    citecolor={blue!50!black},
    urlcolor={blue!80!black}}
% </hyperlinks>

\usepackage{xifthen}
\def\testdateparts#1{\dateparts#1\relax}
\def\dateparts#1 #2 #3 #4 #5\relax{
    \marginpar{\small\textsf{\mbox{#1 #2 #3 #5}}}
}

\def\@lecture{}%
\newcommand*{\lecture}[3]{
    \ifthenelse{\isempty{#3}}{%
        \def\@lecture{Lecture #1}%
    }{%
        \def\@lecture{Lecture #1: #3}%
    }%
    \subsection*{\@lecture}
    \marginpar{\raggedright\small\textsf{#2}}
    \vskip 6pt
}
\newcommand*{\assignment}[2]{%
    \stepcounter{assignment}%
    \subsection{Assignment #1}
    \marginpar{\raggedright\small{due #2}}
    \vskip 6pt
}
\makeatletter
\newcommand*{\refifdef}[3]{%label,command,fallback
    \@ifundefined{r@#1}{#3}{#2{#1}}%
}
\makeatother

\DeclareMathOperator{\dom}{dom}
\DeclareMathOperator{\ran}{ran}
\DeclareMathOperator{\spann}{span}
\DeclareMathOperator{\dimn}{dim}

\newcommand{\psum}[2]{\ensuremath{\mathrm{sum}_{#1}({#2})}}
\newcommand{\pprod}[2]{\ensuremath{\mathrm{prod}_{#1}({#2})}}


\title{UM101: Short Notes}
\author{Naman Mishra}

\begin{document}
\maketitle

\section{Set theory \& the real number system}

\begin{defn} \label{defn:peano}
    The set $A$ along with a successor function $S$ is called a Peano set if it obeys the Peano axioms.
    \begin{enumerate}[label=(P\arabic*)]
        \item \label{defn:peano:zero}
            There is an element called 0 in $A$.
        \item \label{defn:peano:succ}
            For every $a \in A$, its successor $S(a)$ is also in $A$.
        \item \label{defn:peano:not_succ}
            $\forall \; a \in A, S(a) \neq 0$.
        \item \label{defn:peano:injective}
            For any $m, n \in A$, $S(m) = S(n)$ only if $m = n$.
        \item \label{defn:peano:induction}
            (principle of mathematical induction) For any set $B \subseteq A$, if $0 \in B$ and $a \in B \implies S(a) \in B$, then $B = A$.
    \end{enumerate}
\end{defn}

\subsection{The ZFC Axioms}

\begin{defn} \label{defn:set}
    A \textbf{set} is a well-defined collection of (mathematical) objects, called the \emph{elements} of that set.
    To say that $a$ is an element of set $A$, we write $a \in A$.
    Otherwise, we write $a \notin A$.

    Given two sets $A$ and $B$, we say that:
    \begin{enumerate}[wide]
        \item[($A \subseteq B$)] $A$ is a subset of $B$, \textit{i.e.}, every element of $A$ is an element of $B$.
        \item[($A \not\subseteq B$)] $A$ is not a subset of $B$, \textit{i.e.}, there is some element in $A$ which is not an element of $B$.
        \item[($A \subsetneq B$)] $A$ is a proper subset of $B$, \textit{i.e.}, $A \subseteq B$ but $\exists\; b \in B$ such that $b \notin A$.
    \end{enumerate}
\end{defn}

\begin{axiom}[the basic axiom] \label{zfc:basic}
    Every object is a set.
\end{axiom}

\begin{axiom}[axiom of extension] \label{zfc:extension}
    Two sets $A, B$ are equal if they have exactly the same elements.
    In other words, $A = B \iff A \subseteq B$ and $B \subseteq A$
\end{axiom}

\begin{axiom}[axiom of existence] \label{zfc:existence}
    There is a set with no elements called the empty set, denoted by the symbol $\varnothing$.
\end{axiom}

\begin{axiom}[axiom of specification] \label{zfc:specification}
    Let $A$ be a set. Let $P(a)$ denote a property that applies to every element in $A$, i.e., for each $a \in A$, either $P(a)$ is true or it is false. Then there exists a subset
    \[
        B = \set{a \in A: P(a) \text{ is true}}
    \]
\end{axiom}

\begin{axiom}[axiom of pairing]\label{zfc:pairing}
    Given two sets $A, B$, there exists a set which contains precisely $A, B$ as its elements, which we denote by $\set{A, B}$.
\end{axiom}

\begin{axiom}[axiom of unions] \label{zfc:unions}
    Given a set $\mathscr{F}$ of sets, there exists a set called the union of the sets in $\mathscr{F}$, denoted by $\bigcup_{A \in \mathscr{F}} A$, whose elements are precisely the elements of the elements of $\mathscr{F}$.
    \[
        a \in \bigcup_{A \in \mathscr{F}} A \iff a \in A \text{ for some } A \in \mathscr{F}
    \]
\end{axiom}

\begin{axiom}[axiom of powers] \label{zfc:powers}
    Given a set $A$, there exists a set called power set of $A$ denoted $\mathscr{P}(A)$, whose elements are precisely all the subsets of $A$.   
\end{axiom}

\begin{defn} \label{defn:relation}
    A relation from set $A$ to set $B$ is a subset $R$ of $A \times B$. For any $a \in A, b \in B$ we say $a\mathrel{R}b$ iff $(a, b) \in R$.
    \begin{outline}
        \1 The \emph{domain} of $R$ is the set
        \[
            \text{dom}(R) = \set{ a \in A : (a, b) \in R \text{ for some } b \in B}
        \]
        \1 The \emph{range} of $R$ is the set
        \[
            \text{ran}(R) = \set{ b \in B : (a, b) \in R \text{ for some } a \in A}
        \]
        \1 $R$ is called a \emph{function} from $A$ to $B$, denoted as $R: A \to B$ iff
            \2 dom$(R) = A$
            \2 for each $a \in A$ there is (at most) one $b \in B$ such that $(a, b) \in R$.
    \end{outline}
\end{defn}

\begin{axiom}[axiom of regularity] \label{zfc:regularity}
    Read up
\end{axiom}

\begin{axiom}[axiom of replacement] \label{zfc:replacement}
    Read up
\end{axiom}

\begin{axiom}[axiom of choice] \label{zfc:choice}
    Read up
\end{axiom}

\begin{defn} \label{defn:zfc:inductive}
    Given a set $A$, its \emph{successor} is the set \[
        A^{+} = A \cup \set{A}.
    \] A set $A$ is said to be \emph{inductive} if $\varnothing \in A$ and for every $a \in A$, we have $a^{+} \in A$.
\end{defn}

\begin{axiom}[axiom of infinity] \label{zfc:infinity}
    There exists an inductive set.    
\end{axiom}

\begin{lem} \label{thm:zfc:inductive_intersection}
    Let $\mathscr{F}$ be a nonempty set of inductive sets. (This exists by \nameref{zfc:infinity} and \nameref{zfc:pairing}). Then \[
        \bigcap_{A \in \mathscr{F}}{A} \text{ is inductive.}
    \]
\end{lem}

\begin{thm} \label{thm:zfc:omega}
    There exists a \emph{unique}, \emph{minimal} inductive set $\omega$, \textit{i.e.}, for any inductive set $S$, \[
        \omega \subseteq S
    \] and if $\omega'$ is any other inductive set satisfying this property, \[
        \omega = \omega'
    \]
\end{thm}

\begin{thm} \label{thm:zfc:omega_is_peano}
    The $\omega$ in \cref{thm:zfc:omega} is a Peano set with successor function $a \mapsto a^{+}$.
\end{thm}

\begin{thm}[principle of recursion] \label{thm:zfc:recursion}
    Let $A$ be a set, and $f: A \to A$ be a function. Let $a \in A$. Then, there exists a function $F: \omega \to A$ such that
    \begin{enumerate}[label=(\alph*)]
        \item $F(\varnothing) = a$
        \item For some $b \in \omega$, we have $F(b^{+}) = f(F(b))$
    \end{enumerate}
\end{thm}

\subsection{Natural Numbers}

\begin{defn}[Peano addition] \label{defn:N:addition}
    Given a fixed $m \in \N$, the \nameref{thm:zfc:recursion} gives a unique function \[
        \mathrm{sum}_{m} : \N \to \N
    \]
    \begin{enumerate}[label=(A\arabic*)]
        \item\label{defn:N:addition:zero} $\psum{m}{0} = m$ 
        \item\label{defn:N:addition:recursion} $\psum{m}{n^{+}} = (\psum{m}{n})^{+}$.
    \end{enumerate}
    Define \[
        m + n := \psum{m}{n}
    \]
\end{defn}

\begin{prop} \label{thm:N:2+3=5}
    $2 + 3 = 5$
\end{prop}

\begin{defn}[Peano multiplication] \label{defn:N:multiplication}
    Let $m \in \N$. By the recursion principle, $\exists$ a unique function \[
        \mathrm{prod}_{m} : \N \to \N
    \]
    such that
    \begin{enumerate}[label=(\alph*)]
        \item\label{defn:N:multiplication:zero} $\pprod{m}{0} = 0$
        \item\label{defn:N:multiplication:recursion} $\pprod{m}{n^{+}} = m + \pprod{m}{n}$.
    \end{enumerate}
\end{defn}

\begin{thm} \label{thm:N:properties}
    The following hold:
    \begin{enumerate}[label=(\alph*)]
        \item\label{thm:N:properties:commutativity} (Commutativity)
        \begin{align*}
            m + n &= n + m \\
            m \cdot n &= n \cdot m
        \end{align*}
        for all natural numbers $m$ and $n$.

        \item\label{thm:N:properties:associativity} (Associativity)
        \begin{align*}
            m + (n + k) &= (m + n) + k \\
            m \cdot (n \cdot k) &= (m \cdot n) \cdot k
        \end{align*}
        for all natural numbers $m, n, k$.

        \item\label{thm:N:distributivity} (Distributivity) \[
            m \cdot (n + k) = (m \cdot n) + (m \cdot k)
        \] for all natural numbers $m, n, k$.
        \item\label{thm:N:m+n=0=>m=n=0} $m + n = 0 \iff m = n = 0$ for any $m, n \in \N$

        \item\label{thm:N:m.n=0=>m=0|n=0} $m \cdot n = 0 \iff m = 0$ or $n = 0$ for any $m, n \in \N$

        \item\label{thm:N:cancellation} (Cancellation) $m + k = n + k \iff m = n$ for any $m, n, k \in \N$ and if $m \cdot k = n \cdot k$ and $k \neq 0$, then $m = n$.
    \end{enumerate}
\end{thm}

\subsubsection{Tao}

\begin{lem} \label{thm:N:Tao:n+0=n}
For any natural number $n$, $n+0=n$.
\end{lem}

\begin{lem} \label{thm:N:Tao:n+S(m)=S(n+m)}
For any natural numbers $n$ and $m$, $n+m_{++}=(n+m)_{++}$
\end{lem}

\begin{cor} \label{thm:N:Tao:S(n)=n+1}
    $n_{++} = n + 1$.
\end{cor}

\begin{prop} \label{thm:N:Tao:addition_is_commutative}
\emph{(Addition is commutative)} For any natural numbers $n$ and $m$, $n + m = m + n$
\end{prop}

\subsection{Fields, Ordered Sets and Ordered Fields}

\begin{defn} \label{defn:field}
    A field is a set $F$ with 2 operations $+ : F \times F \to F$ and $\cdot : F \times F \to F$ such that
    \begin{enumerate}[label=(F\arabic*)]
        \item \label{defn:field:commutativity}
            $+$ \& $\cdot$ are commutative on $F$.
        \item \label{defn:field:associativity}
            $+$ \& $\cdot$ are associative on $F$.
        \item \label{defn:field:distributivity}
            $+$ \& $\cdot$ satisfy distributivity on $F$, \textit{i.e.}, $a \cdot (b + c) = a \cdot b + a \cdot c$ for all $a, b, c \in F$. 
        \item \label{defn:field:identity}
            There exist 2 \emph{distinct} elements, called 0 (additive identity) and 1 (multiplicative identity) such that
            \begin{align*}
                x + 0 &= x \\
                x \cdot 1 &= x
            \end{align*}
            for all $x \in F$
        \item \label{defn:field:negative}
            For every $x \in F, \,\exists\; y \in F$ such that \[
                x + y = 0
            \]
        \item \label{defn:field:reciprocal}
            For every $x \in F \setminus \set{0}, \,\exists\; z \in F$ such that \[
                x \cdot z = 1
            \]
    \end{enumerate}
\end{defn}

\begin{thm} \label{thm:field:0x=0}
    $(F, +, \cdot)$ is a field. Then for all $x$, \[
        0 \cdot x = x \cdot 0 = 0
    \]
\end{thm}

\begin{defn} \label{defn:order}
    A set $A$ with a relation $<$ is called an \emph{ordered set} if
    \begin{enumerate}[label=(O\arabic*)]
        \newcounter{temp}
        \item \label{defn:order:trichotomy}
            (Trichotomy) For every $x, y \in A$, exactly one of the following holds. \[
                x < y, \quad x = y, \quad y < x
            \]
        \item \label{defn:order:transitivity}
            (Transitivity) If $x < y$ and $y < z$, then $x < z$.
        \setcounter{temp}{\value{enumi}}
    \end{enumerate}
    \textbf{Notation:} $x < y$ is read as ``x is less than y'' \\
    $x \leq y$ means $x < y$ or $x = y$, read as ``x is less that or equal to y''. \\
    $x > y$ is read as ``$x$ is greater that $y$'' and equivalent to $y < x$.
\end{defn}

\begin{defn} \label{defn:ordered field}
    An \emph{ordered field} is a set that admits two operations $+$ and $\cdot$ and relation $<$ so that $(F, +, \cdot)$ is a field and $(F, <)$ is an ordered set and:
    \begin{enumerate}[label=(O\arabic*)]
        \setcounter{enumi}{\value{temp}}
        \item \label{defn:order:sum}
            For $x, y, z \in F$, if $x < y$ then $x + z < y + z$.
        \item \label{defn:order:product}
            For $x, y \in F$, if $0 < x$ and $0 < y$ then $0 < x \cdot y$.
    \end{enumerate}
\end{defn}

\begin{lem} \label{thm:field:unique_inverses}
    Given a field $(F, +, \cdot)$: For any element $a$ in a field $F$, there exists ony one $b$ such that $a + b = 0$. We will denote this $b$ as $-a$. Similarly for any $a$ in $F \setminus \set{0}$ there exists only one $b \in F$ such that $ab = 1$. We will denote this $b$ as $\frac{1}{a}$ or $a^{-1}$.
\end{lem}

\begin{lem} \label{thm:field:inverse_involution}
    $-(-a) = a = (a^{-1})^{-1}$
\end{lem}

\begin{lem} \label{thm:field:(-a)b=-(ab)}
    For any field $(F, +, \cdot)$, $(-a)b = -(ab)$ and $(-a)(-b) = ab$.
\end{lem}

\begin{thm} \label{thm:field:0<1}
    For any field $(F, +, \cdot)$, $0 < 1$.
\end{thm}

\subsection{Upper bounds \& least upper bounds}

\begin{defn} \label{defn:upper_bound}
    A non-empty subset $S \subseteq F$ is said to be \emph{bounded above} in $F$ if there exists a $b \in F$ such that \[
        a \leq b \;\forall\; a \in S
    \]
    Here, $b$ is called an \emph{upper bound} of $S$. If $b \in S$, then $b$ is a \emph{maximum} of $S$.
\end{defn}

\begin{defn} \label{defn:upper_bound:supremum}
    Let $S \subseteq F$ be bounded above. An element $b \in F$ is said to be a \emph{least upper bound} of $S$ or a \emph{supremum} of $S$ if:
    \begin{enumerate}[label=(\alph*)]
        \item $b$ is an upper bound of $S$.
        \item If for $c \in F$, $c < b$, then $c$ is not an upper bound of $S$. In other words, for any $c < b, \;\exists\; s_{c} \in S$ such that $c < s_{c}$. \\
        Contrapositive: If $c$ is an upper bound of $S$, then $c$ is not less than $b$, \textit{i.e.}, $b \leq c$.
    \end{enumerate}
\end{defn}

\subsection{The Real Numbers}

\begin{thm}[Archimedean property of $\R$] \label{thm:R:archimedean}
    Let $x, y \in \R$ and $x > 0$, then $\exists\; n \in \N$ such that \[
        n \cdot x > y.
    \]
\end{thm}

\section{Sequences \& Series}

\subsection{Sequences}

\begin{defn} \label{defn:sequence}
    A sequence in $\R$ is a function $f: \N \to \R$. We denote this sequence by $\set{a_{n}}_{n \in \N}$, where \[
        a_{n} = f(n) \quad \forall\; n \in \N
    \] and $a_{n}$ is called the $n^{th}$ term of $\set{a_{n}}_{n \in \N}$.
\end{defn}

\begin{defn} \label{defn:sequence:convergence}
    We say that a sequence $\set{a_{n}} \subseteq \R$ is \emph{convergent} (in $\R$) if $\exists\; L \in \R$ such that for each $\varepsilon > 0, \,\exists\; N_{\varepsilon,L} \in \N$ such that \[
        \abs{a_{n} - L} < \varepsilon \quad \forall\; n \geq N_{\varepsilon, L}
    \] 
    We will call $L$ \emph{a} limit of $\set{a_{n}}$ and we write: \[
        a_{n} \to L \text{ as } n \to \infty
    \]
    A sequence $\set{a_{n}}$ is said to be \emph{divergent} if it is not convergent, \textit{i.e.}, $\forall\; L \in \R$ and $N_{L} \in \N$, $\exists\; \varepsilon > 0$ and $N \geq N_{L}$ such that \[
        \abs{a_{N} - L} > \varepsilon
    \]
\end{defn}

\begin{thm}[Uniqueness of limits] \label{thm:sequence:unique_limit}
    Suppose $L_{1}$ and $L_{2}$ are limits of a (convergent) sequence $\set{a_{n}} \in \R$. Then $L_{1} = L_{2}$.
\end{thm}

\begin{defn} \label{defn:sequence:bounded}
    A sequence $\set{a_{n}}_{n \in \N}$ is said to be \emph{bounded} if $\exists\; M > 0$ such that $\abs{a_{n}} < M \,\forall\; n \in \N$.
\end{defn}

\begin{thm} \label{thm:sequence:convergent=>bounded}
    Every convergent sequence is bounded.
\end{thm}

\begin{defn} \label{defn:sequence:monotone}
    A sequence $\set{a_{n}} \subseteq \R$ is said to be \emph{monotonically increasing} if $a_{n} \leq a_{n+1} \,\forall\; n \in \N$.

    A sequence $\set{a_{n}} \subseteq \R$ is said to be \emph{monotonically decreasing} if $a_{n} \geq a_{n+1} \,\forall\; n \in \N$.

    A sequence $\set{a_{n}} \subseteq \R$ is said to be \emph{monotone} if it is either monotonically increasing or monotonically decreasing.
\end{defn}

\begin{thm}[Monotone convergence theorem] \label{thm:sequence:MCT}
    A monotone sequence is convergent iff it is bounded.
\end{thm}

\begin{defn} \label{defn:sequence:diverging_to_infinity}
    We say that a sequence diverges to $+\infty$ if $\forall\; R \in \R$, $\exists\; N_{R} \in \N$ such that $a_{n} > R \;\forall\; n \geq N_{R}$. \\
    We say that a sequence diverges to $-\infty$ if $\forall\; R \in \R$, $\exists\; N_{R} \in \N$ such that $a_{n} < R \;\forall\; n \geq N_{R}$. \\
    We write $\lim_{n \to \infty} a_{n} = +\infty$ or $\lim_{n \to \infty} a_{n} = -\infty$, but this is purely notational and does not mean ``$\set{a_{n}}$ has a limit''.
\end{defn}

\begin{thm}[Tao Theorem 6.1.19] \label{thm:sequence:reciprocal_limit}
    Suppose $\set{b_{n}}$ converges to $b \neq 0$ (and $\exists\; M \in \N$ such that $b_{n} \neq 0 \;\forall\; n \geq M$.) \\
    Then $\set{\frac{1}{b}}_{n \geq M} \to \frac{1}{b}$ as $n \to \infty$.
\end{thm}

\subsection{Infinite series}

\begin{defn} \label{defn:series}
    An infinite series is a \emph{formal expression} of the form \[
        a_{0} + a_{1} + a_{2} + \dots, \text{ or}, \sum_{n=0}^{\infty} a_{n}
    \]
    Given $\sum_{n=0}^{\infty} a_{n}$, its sequence of partial sums (sops) is $\set{s_{n}}_{n=0}^{\infty}$ where
    \begin{align*}
        s_{0} &= a_{0} \\
        s_{1} &= a_{0} + a_{1} \\
        &\;\;\vdots \\
        s_{n} &= a_{0} + a_{1} + \dots a_{n}
    \end{align*}
    We say that $\sum a_{n}$ is \emph{convergent} with sum $s$ if $\lim_{n \to \infty} s_{n} = s$. Otherwise, we say that $\sum a_{n}$ is divergent.
\end{defn}

\begin{thm} \label{thm:series:convergent=>limit=0}
    Suppose $\sum a_{n}$ is convergent. Then \[
        \lim_{n \to \infty} a_{n} = 0
    \]
\end{thm}

\begin{thm}[Comparison test] \label{thm:series:comparison}
    Suppose there exist constants $M \in \N$ and $0 < C$ such that \[
        0 \leq a_{n} \leq C b_{n} \quad\forall\; n \geq M
    \] If $\sum b_{n}$ converges, then $\sum a_{n}$ converges. In other words, If $\sum a_{n}$ diverges, $\sum b_{n}$ diverges.
\end{thm}

\begin{thm}[Ratio test] \label{thm:series:ratio_test}
    Let $\sum a_{n}$ be a series of positive terms. Suppose \[
        \lim_{n \to \infty} \frac{a_{n+1}}{a_{n}} = L \in \R
    \] Then,
    \begin{enumerate}[label=(\alph*)]
        \item If $L < 1$, the series converges.
        \item If $L > 1$, the series diverges.
        \item If $L = 1$, the test is inconclusive.
    \end{enumerate}
\end{thm}

\begin{thm} \label{thm:series:limit_laws}
    Suppose $\sum a_{n}$ and $\sum b_{n}$ converge with sums $a$ and $b$ respectively.
    Then, for constants $l$ and $m$, $\sum la_{n} + mb_{n}$ converges to $la + mb$.
    Suppose $\sum \abs{a_{n}}$ and $\sum \abs{b_{n}}$ converge.
    Then, so does $\sum \abs{la_{n} + mb_{n}}$ for any choice of $l$ and $m$ in $\R$.
\end{thm}

\begin{cor} \label{thm:series:limit_laws:divergence}
    Suppose $\sum a_{n}$ converges and $\sum b_{n}$ diverges. Let $m \in \R \setminus \set{0}$. Then, $\sum(a_{n} + b_{n})$ diverges, and $\sum mb_{n}$ diverges.
\end{cor}

\begin{defn} \label{defn:series:absolute_convergence}
    A series $\sum a_{n}$ of real numbers is said to \emph{converge absolutely} if $\sum \abs{a_{n}}$ converges. A series $\sum a_{n}$ of real numbers is said to \emph{converge conditionally} if $\sum \abs{a_{n}}$ diverges but $\sum a_{n}$ converges.
\end{defn}

\begin{thm} \label{thm:series:absolute=>conditional}
    If $\sum a_{n}$ converges absolutely, it must converge. Moreover, $\abs{\sum a_{n}} \leq \sum \abs{a_{n}}$.
\end{thm}

\begin{thm}[Alternating series test] \label{thm:series:AST}
    Suppose $\set{a_{n}}_{n \in \N}$ is a decreasing sequence of positive numbers going to 0.
    Then, $\sum (-1)^{n} a_{n}$ converges.
    Denoting the sum by $S$, we have that \[
        0 < (-1)^{n}(S - s_{n}) < a_{n+1}.
    \] Also called the Leibniz test.
\end{thm}

\section{Limits \& Continuity}

\subsection{Limit of a function}

\begin{defn}[Neighborhood] \label{defn:limit:neighborhood}
    Given a real number $p$ and an $\varepsilon > 0$, the \emph{$\varepsilon$-neighborhood of $p$} is the open interval \[
        N_{\varepsilon}(p) = \paren{p - \varepsilon, p + \varepsilon} = \set{x \in \R : \abs{ x - p} < \varepsilon}.
    \] 
\end{defn}

\begin{defn}[Limit of a function] \label{defn:limit:function}
    Given a function $f$ that is defined on some $I = (a, p) \cup (p, b)$ with $a < b$, we say that $f$ has a \emph{limit} $L$ as it approaches $p$ iff for every $\varepsilon > 0 \;\exists\; \delta > 0$ such that
    \begin{enumerate}[label=(\alph*)]
        \item $0 < \abs{ x - p } < \delta \implies \abs{ f(x) - L } < \varepsilon$ OR
        \item $x \in N_{\delta}(p) \setminus \set{p} \implies f(x) \in N_{\varepsilon}(L)$.
    \end{enumerate}
    This is denoted as \[
        \lim_{x \to p} f(x) = L.
    \]
\end{defn}

\begin{thm}[Limit laws for functions] \label{thm:limit:laws}
    Suppose $f$ and $g$ are functions such that \[
        \lim_{x \to p} f(x) = a, \qquad \lim_{x \to p} g(x) = b.
    \] Then,
    \begin{align}
        \lim_{x \to p} (f \pm g)(x) &= a \pm b \\
        \lim_{x \to p} (f \cdot g)(x) &= a \cdot b \\
        \lim_{x \to p} (f/g)(x) &= a/b
    \end{align}
\end{thm}

\subsection{Continuity}

\begin{defn} \label{defn:cont}
    Let $S \subseteq \R$ be a (nonempty) subset, $f: S \to \R$ and $p \in S$. We say that $f$ is continuous at $p$ iff: \\
    for every $\varepsilon > 0, \;\exists\; \delta_{\varepsilon} > 0$ such that \[
        \abs{x - p} < \delta_{\varepsilon} \;\land\; x \in S \implies \abs{f(x) - f(p)} < \varepsilon
    \] We say that $f$ is continuous on $S$ iff $f$ is continuous at each $p \in S$.
\end{defn}

\begin{thm}[Algebraic laws for continuity] \label{thm:cont:laws}
    Suppose $f$ and $g$ are continuous at $p \in S$. Then so are $f \pm g, fg$ and if $g(p) \neq 0, f/g$.
\end{thm}

\begin{thm} \label{thm:cont:composition}
    Let $f: A \to \R$ and $g: B \to \R$ be continuous functions such that $f(A) \coloneqq \mathop{range}(f) \subseteq B$. Then, \[
        g \circ f : x \in A \mapsto g(f(x)) \in \R
    \] is continuous.
\end{thm}

\begin{thm}[intermediate value theorem] \label{thm:cont:IVT}
    Let $f: [a, b] \to \R$ be a continuous function. Suppose $y \in \R$ is a number between $f(a)$ and $f(b)$, \textit{i.e.}, $y \in [f(a), f(b)]$. Then $\exists\; c \in [a, b]$ such that \[
        f(c) = y
    \]
\end{thm}

\begin{cor}[Bolzano's theorem] \label{thm:cont:Bolzano}
    Let $f : [a, b] \to \R$ be a continuous function such that $f(a)$ and $f(b)$ take opposite signs. Then $\exists\; c \in (a, b)$ such that $f(c) = 0$.
\end{cor}

\begin{thm}[the Borsuk-Ulam theorem] \label{thm:cont:Borsuk-Ulam}
    Let $S^{n}$ be the unit $n$-sphere, \textit{i.e.}, $S^{n} = \set{x \in \R^{n+1} : \norm{x} = 1}$. Let $f: S^{n} \to \R^{n}$ be a continuous function.
    Then $f$ maps some pair of antipodal points to the same point. \[
        \exists\; x \text{ such that } f(x) = f(-x)
    \]
\end{thm}

\begin{lem} \label{thm:sequence:comparison}
    Let $a_{n}, b_{n}$ be convergent sequences such that $a_{n} \leq b_{n}$ for all $n$ (large enough). Then \[
        \lim_{n \to \infty} a_{n} \leq \lim_{n \to \infty} b_{n}
    \]
\end{lem}

\begin{defn} \label{defn:cont:bounded}
    A function $f : S \to \R$ is said to be \emph{bounded above} on $S$ if $\exists\; U \in \R$ such that $f(x) \leq U \;\forall\; x \in S$.

    $f$ is said to be \emph{bounded} if $\exists\; M > 0$ such that $\abs{f(x)} < M \;\forall\; x \in S$.
\end{defn}

\begin{thm}[Continuous functions on compact intervals are bounded] \label{thm:cont:compact_to_bounded}
    Let $f : [a, b] \to \R$ be continuous on $[a, b]$. Then $f$ is a bounded function.
\end{thm}

\begin{defn} \label{defn:cont:global_extrema}
    A function $f : S \to \R$ is said to have a \emph{global maximum} on $S$ at a point $p \in S$ if $f(x) \leq f(p) \;\forall\; x \in S$.

    A function $f : S \to \R$ is said to have a \emph{global minimum} on $S$ at a point $p \in S$ if $f(x) \geq f(p) \;\forall\; x \in S$.
\end{defn}

\begin{thm}[Extreme value theorem] \label{thm:cont:EVT}
    Let $f : [a, b] \to \R$ be a continuous function. Then $f$ attains both a global maximum and a global minimum in $[a, b]$.
\end{thm}

\begin{cor} \label{thm:cont:EVT:closed->closed}
    Let $f : [a, b] \to \R$ be continuous.
    Then (using IVT), \[
        f([a, b]) = [\min_{[a, b]} f, \max_{[a, b]} f].
    \]
\end{cor}

\section{Differentiation}

\begin{defn} \label{defn:diff}
    Let $f : (a, b) \to \R$ be a function and $p \in (a, b)$. We say that $f$ is differentiable in $(a, b)$ if \[
        \lim_{h \to 0} \frac{f(p + h) - f(p)}{h} = \lim_{x \to p} \frac{f(x) - f(p)}{x - p}
    \] exists, and the limit is called the derivative of $f$ at $p$, denoted $f'(p)$.

    If $f$ is differentiable on each $p$ in $(a, b)$, it is said to be differentiable on $(a, b)$ and $f' : (a, b) \to \R$ is called the derivative of $f$ on $(a, b)$.

    We define two more functions:
    \begin{enumerate}[label=(\alph*)]
        \item For any function $f : (a, b) \to \R$ and any $p \in (a, b)$, define \[
            f^{p}_{\delta} : h \in (a - p, b - p) \setminus \set{0} \mapsto \frac{f(p + h) - f(p)}{h} \in \R.
        \]
        \item For any differentiable function $f : (a, b) \to \R$ and any $p \in (a, b)$, define \[
            f^{p}_{\Delta} : h \in (a - p, b - p) \mapsto \begin{cases}
                \dfrac{f(p + h) - f(p)}{h} & h \neq 0 \\
                f'(p) & h = 0
            \end{cases} \in \R.
        \]
    \end{enumerate}
\end{defn}

\begin{thm}[Differentiability $\implies$ continuity] \label{thm:diff:continuity}
    Let $f : (a, b) \to \R$ be differentiable at $p \in (a, b)$. Then $f$ is continuous at $p$.
\end{thm}

\begin{thm}[Algebra of derivatives] \label{thm:diff:laws}
    Let $f, g : (a, b) \to \R$ be differentiable at $p \in (a, b)$. Then
    \begin{enumerate}[label=(\alph*)]
        \item $f + g$ is differentiable at $p$ and $(f + g)' = f' + g'$.
        \item $f - g$ is differentiable at $p$ and $(f - g)' = f' - g'$.
        \item $f \cdot g$ is differentiable at $p$ and $(f \cdot g)' = f' \cdot g + f \cdot g'$.
        \item $f / g$ is differentiable at $p$ if $g \neq 0$ and $(f / g)' = \frac{f' \cdot g - f \cdot g'}{g^2}$.
    \end{enumerate}
\end{thm}

\begin{defn}[Inverse function] \label{defn:diff:inverse}
    Let $f : A \to B$ be bijective. Then for any $y \in B$, there exists (unique) $x_{y} \in A$ such that $f(x_{y}) = y$. We define the inverse function $f^{-1} : B \to A$ as \[
        f^{-1}(y) = x_{y}.
    \] and say that $f$ is invertible on $A$.

    Note that $(f \circ f^{-1})$ and $(f^{-1} \circ f)$ are the identity functions on $B$ and $A$ respectively.
    
    For example, the function $f(x) = x^{2}$ is invertible on $\R^{+}$ and its inverse is $f^{-1}(x) = \sqrt{x}$.
\end{defn}

\begin{thm}[inverse function properties] \label{thm:diff:inverse}
    Let $f : [a, b] \to \R$ be an invertible function on $[a, b]$ with range $J$.
    \begin{enumerate}[label=(\roman*)]
        \item If $f$ is (strictly) increasing, then so is $f^{-1}$.
        \item If $f$ is continuous, then $f : [a, b] \to J$ is strictly monotone and $f^{-1} : J \to [a, b]$ is continuous.
        \item If $f$ is differentiable at $p \in (a, b)$ with $f'(p) \neq 0$ \emph{and continuous in some neighborhood around $p$}, then $f^{-1}$ is differentiable at $f(p) = q \in J$ and $(f^{-1})'(q) = \dfrac{1}{f'(p)}$.
    \end{enumerate}
\end{thm}

\begin{thm}[chain rule] \label{thm:composition dv}
    Let $f: (a, b) \to \R$ and $g : (c, d) \to \R$ with $f((a, b)) \subseteq (c, d)$ and $f$ differentiable in $(a, b)$. Let $g$ be differentiable at $f(p) := q$. Then $g \circ f : (a, b) \to \R$ is differentiable at $p$ and $(g \circ f)' = g' \circ f \cdot f'$ at $p$.
\end{thm}

\subsection{Local Extrema}

\begin{defn}[Local Extrema] \label{defn:local extrema}
    Let $f : A \to \R$. We say that $f$ attains a local maximum at $a \in A$ iff $\exists\; \delta > 0$ such that \[
        f(x) \leq f(a) \;\forall\; x \in N_{\delta}(a) \cap A.
    \] We say that $f$ attains a local minimum at $a \in A$ iff $\exists\; \delta > 0$ such that \[
        f(x) \geq f(a) \;\forall\; x \in N_{\delta}(a) \cap A.
    \]
\end{defn}

\begin{thm}[Extremum $\implies$ Stationary] \label{thm:stationary extrema}
    Let $f : (a, b) \to \R$.
    Let $c \in (a, b)$ such that $f$ is differentiable at $c$.
    If $f$ attains a local extremum at $c$, then $f'(c) = 0$.
    Points at which the derivative vanishes are called `stationary points' and sometimes `critical points'.
\end{thm}

\begin{thm}[Mean Value Theorem] \label{thm:mvt}
    Let $f : [a, b] \to \R$ be continuous on $[a, b]$ and differentiable on $(a, b)$. Then there exists a $c \in (a, b)$ such that $f'(c) = \dfrac{f(b) - f(a)}{b - a}$.
\end{thm}

\begin{defn}[Taylor Polynomial] \label{defn:taylor}
    Let $f : (a, b) \to \R$ be $k$ times differentiable at some $x_{0} \in (a, b)$. The $k^{th}$ Taylor polynomial at $x_{0}$ is defined as \[
        P_{k}^{x_{0}}(x) = f(x_{0}) + \frac{f'(x_{0})}{1!}(x - x_{0}) + \frac{f''(x_{0})}{2!}(x - x_{0})^{2} + \cdots + \frac{f^{(k)}(x_{0})}{k!}(x - x_{0})^{k}.
    \]
\end{defn}

\begin{thm}[Taylor's Theorem] \label{thm:taylor}
    Let $f : (a, b) \to \R$ be an $(n + 1)$ times differentiable function on $(a, b)$.
    Note that this implies $f, f', f'', \dots f^{(n)}$ are continuous.
    Let $x_{0} \in (a, b)$.
    Then $\forall\; x \in (a, b) \;\exists\; c_{x}$ between $x$ and $x_{0}$ such that \[
        f(x) = P_{n}^{x_{0}}(x) + f^{(n+1)}(c_{x}) \frac{(x - x_{0})^{n+1}}{(n+1)!}.
    \]
\end{thm}

\section{Integration}

\begin{defn}[Partition] \label{defn:partition}
    A \emph{partition} of $[a, b]$ is a finite subset \[
        P = \set{x_{0}, x_{1}, \dots x_{n}} \subseteq [a, b]
    \] such that $a = x_{0} < x_{1} < \dots < x_{n-1} < x_{n} = b$.
    We write \[
        P = \set{x_{0} < x_{1} < \dots < x_{n}}.
    \]
\end{defn}

\begin{defn}[Refinement] \label{defn:refinement}
    Given two partitions $P$ and $Q$ of $[a, b]$, $Q$ is said to be a \emph{refinement} of $P$ if $P \subseteq Q$.
    
\end{defn}

\begin{defn}[Common refinement] \label{defn:common refinement}
    Given two partitions $P$ and $Q$ of $[a, b]$, the \emph{common refinement} of $P$ and $Q$ is the smallest refinement of both $P$ and $Q$ simultaneously. Thus, \[
        R = P \cup Q
    \] is the common refinement of $P$ and $Q$.
\end{defn}

\begin{defn}[Step function] \label{defn:step fn}
    Given an interval $[a, b]$, a function $S : [a, b] \to \R$ is called a \emph{step function} is there is some partition $P = \set{x_{0} < x_{1} < \dots < x_{n}}$ of $[a, b]$ such that for each $j \in [1 .. n], \;\exists\; s_{j} \in \R$ such that \[
        s(x) = s_{j} \quad \;\forall\; x \in (x_{j-1}, x_{j}).
    \]
\end{defn}

\begin{defn}[Step Integration] \label{defn:step int}
    Given a step function $s : [a, b] \to \R$ corresponding to $P = \set{x_{0} < x_{1} < \dots < x_{n}}$, define \[
        \int_{a}^{b} s(x) \dd x = \sum_{j=1}^{n} s_{j} (x_{j} - x_{j-1}).
\] We also define \[
    \int_{b}^{a} s(x) \dd x = - \int_{a}^{b} s(x) \dd x.
\]
\end{defn}

\begin{thm}[Properties] \label{thm:int properties} \hfill
    \begin{enumerate}[label=(\alph*)]
        \item $\int_{a}^{b} (c_{1}s(x) + c_{2}t(x)) \dd x = c_{1} \int_{a}^{b} s(x) \dd x + c_{2} \int_{a}^{b} t(x) \dd x$.
        \item If $s \leq t$ on $[a, b]$, then $\int_{a}^{b} s(x) \dd x \leq \int_{a}^{b} t(x) \dd x$.
        \item $\int_{ka}^{kb} s(x/k) \dd x = k \int_{a}^{b} s(x) \dd x$.
    \end{enumerate}
\end{thm}

\begin{defn}[] \label{defn:}
    Let $f : [a, b] \to \R$ be a bounded function. Let \[
        S_{f} = \set{s : [a, b] \to \R : s \text{ is a step function and } s \leq f \text{ on } [a, b]}
    \] and \[
        T_{f} = \set{t : [a, b] \to \R : t \text{ is a step function and } t \geq f \text{ on } [a, b]}.
    \]
\end{defn}

\begin{lem}[] \label{lem:}
    Let $f : [a, b] \to \R$ be bounded, \textit{i.e.}, $\exists\; M > 0$ such that \[
        -M \leq f(x) \leq M \;\forall\; x \in [a, b].
    \] Then $\sup s_{f}$ and $\inf t_{f}$ exist and \[
        -M (b - a) \leq \sup s_{f} \leq \inf t_{f} \leq M (b - a)
    \] where $s_{f} = \set{\int_{a}^{b}s(x) \dd x : s \in S_{f}}$ and $t_{f} = \set{\int_{a}^{b} t(x) \dd x : t \in T_{f}}$.
\end{lem}

\begin{defn}[] \label{defn:}
    Given a bounded function $f : [a, b] \to \R$, its \emph{lower integral} is \[
        \ubar{I}(f) = \sup \set{\int_{a}^{b} s(x) \dd x : s \in S_{f}}
    \] and its \emph{upper integral} is \[
        \bar{I}(f) = \inf \set{\int_{a}^{b} s(x) \dd x : s \in T_{f}}.
    \]
    A bounded function $f : [a, b] \to \R$ is said to be \emph{Riemann integrable} (not really) if $\ubar{I}(f) = \bar{I}(f)$ and we call this quantity the integral of $f$ over $[a, b]$, denoted by \[
        \int_{a}^{b} f(x) \dd x.
    \] We also define \[
        \int_{b}^{a} f(x) \dd x = -\int_{a}^{b} f(x) \dd x.
    \]
\end{defn}

\begin{thm}[Monotone Integrable] \label{thm:monotone integral}
    Every bounded monotone function on $[a, b]$ is Riemann integrable on $[a, b]$.
\end{thm}

\begin{defn}[Uniform Continuity] \label{defn:uniformly continuous}
    A function $f : A \to \R$ is said to be \emph{uniformly continuous} if for every $\varepsilon > 0$, there exists a $\delta_{\varepsilon} > 0$ such that whenever $x, y \in A$ and $\abs{x - y} < \delta_{\varepsilon}$, then $\abs{f(x) - f(y)} < \varepsilon$.
\end{defn}

\begin{thm}[Closed continuous $\implies$ uniformly continuous] \label{thm:integration:uniform_continuity:bounded}
    Every continuous function on a closed, bounded interval is uniformly continuous on $[a, b]$.
\end{thm}

\begin{thm}[Continuity $\implies$ Riemann Integrability] \label{thm:continuous RI}
    Let $f$ be a continuous function on $[a, b]$. Then $f$ is Riemann integrable on $[a, b]$.
\end{thm}

\begin{thm}[Mean Value -- Integrals] \label{thm:mvt int}
    Let $f$ be a continuous function on $[a,b]$.  Then there exists a
    number $c$ in $[a,b]$ such that \[
        f(c) = \frac{1}{b - a} \int_{a}^{b} f(x) \dd x.
    \]
\end{thm}

\begin{thm}[The First Fundamental Theorem of Calculus] \label{thm:integration:IFTOC}
    Let $f : [a, b] \to \R$ be Riemann integrable. Let \[
        F(x) = \int_{a}^{x} f(t) \dd t \qquad \forall\; x \in [a, b].
    \] Then $F$ is continuous on $[a, b]$. Moreover, if $f$ is continuous at some $p \in (a, b)$, then $F$ is differentiable at $p$ with $F'(p) = f(p)$.
\end{thm}

\begin{thm}[Integral Triangle Inequality] \label{thm:integral triangle}
    Let $f : [a, b] \to \R$ be Riemann integrable. Then $\abs{f} : [a, b] \to \R$ is Riemann integrable and \[
        \abs{\int_{a}^{b} f(x) \dd x} \leq \int_{a}^{b} \abs{f(x)} \dd x.
    \]
\end{thm}

\begin{defn}[Primitive] \label{defn:primitive}
    Given a function $f : (a, b) \to \R$, with $a < b$ and $a, b \in \R \cup \set{-\infty, +\infty}$, a \emph{primitive} or \emph{antiderivative} of $f$ on $(a, b)$ is a differentiable function $F : (a, b) \to \R$ such that \[
        F'(x) = f(x) \;\forall\; x \in (a, b).
    \]
\end{defn}

\begin{thm}[The Second Fundamental Theorem of Calculus] \label{thm:second ftoc}
    Let $f : (c, d) \to \R$ be a function such that $f_{[a, b]}$ ($[a, b] \subset (c, d)$) is Riemann integrable. Let $F$ be a primitive of $f$ on $(c, d)$. Then \[
        \int_{a}^{b} f(x) \dd x = F(b) - F(a).
    \]
\end{thm}

\subsection{Logarithms \& Exponentiation}

\begin{defn}[Natural Logarithm] \label{defn:ln}
    Let $x > 0$. The \emph{natural logarithm} of $x$ is the quantity \[
        \ln(x) = \int_{1}^{x} \frac{1}{t} \dd t.
    \]
\end{defn}

\begin{thm}[] \label{thm:}
    The function $\ln : \R^{+} \to \R$ has the following properties:
    \begin{enumerate}[label=(\alph*)]
        \item $\ln(1) = 0$.
        \item $\ln(x) + \ln(y) = \ln(xy) \;\forall\; x, y \in \R^{+}$.
        \item $\ln$ is continuous and strictly increasing.
        \item $\ln$ is differentiable and \[
            \ln'(x) = \frac{1}{x}.
        \]
        \item (Leibniz) \[
            \int \frac{1}{t} \dd t = \ln \abs{t} + C.
        \]
        \item (Leibniz) \[
            \int \ln x \dd x = x \ln x - x + C.
        \]
        \item $\ln$ is bijective.
    \end{enumerate}
\end{thm}

\begin{defn}[e \& Exponentiation] \label{defn:e}
    Let $e$ be the unique number that satisfies \[
        \ln(e) = 1.
    \] Given any $x \in \R$, let $\exp(x)$ be the unique positive $y$ such that \[
        \ln(y) = x.
    \] That is, $\exp$ is the inverse function of $\ln$.
\end{defn}

\begin{thm}[] \label{thm:}
    $\exp : \R \to \R^{+}$ has the following properties:
    \begin{enumerate}[label=(\alph*)]
        \item $\exp(0) = 1$.
        \item $\exp(x + y) = \exp(x) \exp(y)$.
        \item $\exp$ is continuous and strictly increasing.
        \item $\exp$ is differentiable and \[
            \exp'(x) = \exp(x) \;\forall\; x \in \R.
        \]
        \item \[
            \int \exp(x) \dd x = \exp(x) + C.
        \]
        \item $\exp$ is bijective.
        \item $\exp(r) = e^{r} \;\forall\; r \in \Q$.
    \end{enumerate}
\end{thm}

\section{Vector Spaces}

\begin{defn}[] \label{defn:}
    Let $(F, \oplus, \odot)$ be a field. A \emph{vector space over $F$} is a set $V$ such that:
    \begin{enumerate}[label=(\alph*)]
        \item Given any two elements $v, w \in V$, there exists a unique element $v + w \in V$ called its sum.
        (This $+$ may not be the same as $\oplus$).
        \item Given an $a \in F$ and a $v \in V$, there is a unique element $av = a \cdot v \in V$ called the scalar product of $a$ and $v$.
    \end{enumerate}
    satisfying the following axioms:
    \begin{enumerate}[label=(V\arabic*)]
        \item $v + w = w + v$ for all $v, w \in V$.
        \item $(v + w) + u = v + (w + u)$ for all $v, w, u \in V$.
        \item There is an element $0 \in V$ such that $v + 0 = v$ for all $v \in V$.
        \item For all $v \in V$, there is a unique element $-v \in V$ called the additive inverse of $v$ such that $v + (-v) = 0$.
        \item For all $a, b \in F$ and $v \in V$, we have \[
            (a \odot b) \cdot v = a \cdot (b \cdot v)
        \] Note that this implies $a \cdot (b \cdot v) = b \cdot (a \cdot v)$ by the commutativity of $\odot$.
        \item Let $1_{F}$ be the multiplicative identity of $F$. Then, \[
            1_{F} \cdot v = v \quad\text{for all}\quad v \in V
        \]
        \item For all $a, b \in F$ and $v \in V$, we have \[
            (a \oplus b) \cdot v = a \cdot v + b \cdot v
        \]
        \item For all $a \in F$ and $v, w \in V$, we have \[
            a \cdot (v + w) = a \cdot v + a \cdot w
        \]
    \end{enumerate}
    We call the elements of $F$ \emph{scalars} and the elements of $V$ \emph{vectors}.
\end{defn}

\begin{prop}[Vector properties] \label{prop:vector:properties}
    Let $V$ be a vector space over $F$. Then the following hold:
    \begin{enumerate}[label=(\alph*)]
        \item $V$ has a unique additive identity.
        \item $0_{F} v = 0_{V}$ for all $v \in V$.
        \item $a 0_{V} = 0_{V}$ for all $a \in F$.
        \item Each $v \in V$ has a unique additive inverse given by $(-1_{F}) v$.
        \item If $a v = a w$ for some $a \in F \setminus\set{0}$ and $v, w \in V$, then $v = w$.
    \end{enumerate}
\end{prop}

\begin{defn}[Subspace] \label{defn:vector:subspace}
    Let $V$ be a vector space over some field $F$.
    A subset $S \subseteq V$ is a (linear) \emph{subspace} of $V$ if the following hold:
    \begin{enumerate}[label=(\alph*)]
        \item $0_{V} \in S$.
        \item If $v, w \in S$, then $v + w \in S$.
        \item If $a \in F$ and $v \in S$, then $a v \in S$.
    \end{enumerate}
    These properties together imply that $S$ is also a vector space over $F$.

    $S$ is said to be a \emph{proper} subspace of $V$ if $S \neq V$ but also $S \neq \set{0_{V}}$.
\end{defn}

\begin{defn}[Span of finite sequences] \label{defn:vector:span:finite_sequence}
    Let $v_{1}, v_{2}, \dots, v_{m} \in V$ be a finite sequence of vectors.
    A linear combination of $v_{1}, v_{2}, \dots, v_{m}$ is any vector of the form \[
        v = a_{1} v_{1} + a_{2} v_{2} + \cdots + a_{m} v_{m},
    \] where $a_{1}, a_{2}, \dots, a_{m} \in F$.
    The \emph{span} of the finite sequence $v_{1}, v_{2}, \dots, v_{m}$ is the set of all linear combinations of $v_{1}, v_{2}, \dots, v_{m}$.
    That is, \[
        \spann(v_{1}, v_{2}, \dots, v_{m}) = \set{\sum_{j = 1}^{m} a_{j} v_{j} : a_{j} \in F}.
    \]
\end{defn}

\begin{defn}[Span of sets] \label{defn:vector:span:set}
    Let $S \subseteq V$ be a nonempty set. The \emph{span} of the $S$ is the set \begin{align*}
        \spann S = \{v \in V : \;\exists\; &a_1, \dots, a_n \in F \\
        \text{and distinct } &v_1, \dots, v_n \in S \\
        \text{such that } &v = a_1 v_1 + \cdots + a_n v_n\}.
    \end{align*}
    Or \[
        \spann S = \bigcup_{\O \neq \Lambda \subseteq ^{\mathrm{finite}} S} \spann \Lambda.
    \]
    $\spann \O$ is defined to be $\set{0}$.
\end{defn}

\begin{defn}[Basis] \label{defn:basis}
    Given a vector space $V$ over $F$, a \emph{basis} is a subset $B \subseteq V$ such that
    \begin{enumerate}[label=(\alph*)]
        \item $B$ is a spanning set, \textit{i.e.}, $V = \spann(B)$.
        \item $B$ is linearly independent.
    \end{enumerate}
\end{defn}

\begin{cor}[] \label{cor:}
    Let $V$ be a finite dimensional vector space over a field $F$.
    Let $S$ be a finite spanning set of $V$.
    Then $S$ contains as a subset a basis of $V$.
\end{cor}

\begin{cor}[] \label{cor:}
    Every finite dimensional vector space has a basis.
\end{cor}

\begin{prop}[] \label{prop:}
    Let $L \subseteq V$ be linearly independent.
    Then for some $v \in V$, $L \cup \set{v}$ is linearly independent iff $v \notin \spann(L)$.
\end{prop}

\begin{cor}[] \label{cor:}
    Let $V$ be a finite dimensional vector space.
    Let $L \subseteq V$ be a finite linearly independent set.
    Then there exist finitely many vectors $w_{1}, \dots, w_{m} \in V$ such that $L \cup \set{w_{1}, \dots, w_{m}}$ is a basis.
\end{cor}

\begin{thm}[] \label{thm:}
    Let $V$ be a finite dimensional vector space. Let $S, L \subseteq V$ be such that $S$ is a spanning set and $L$ is linearly independent. Then \[
        \#L \leq \#S.
    \]
\end{thm}

\begin{cor}[] \label{cor:}
    Every (finite) basis of a finite dimensional vector space has the same size.
\end{cor}

\begin{cor}[] \label{cor:}
    Let $S$, $L$, $V$ be as in the previous theorem.
    If $\#L = \#S$, then both are bases of $V$.
\end{cor}

\begin{cor}[Finite Basis of FDVS] \label{cor:finite basis}
    Every basis of a finite dimensional vector space is finite.
\end{cor}

\begin{defn}[Dimension] \label{defn:dimension}
    Let $V$ be a finite dimensional vector space.
    We define the length of any of its bases to be the \emph{dimension} of $V$.
\end{defn}

\begin{prop}[] \label{prop:}
    Let $V$ be a finite dimensional vector space.
    Let $W$ be a subspace of $V$.
    Then $W$ is finite dimensional and $\dim(W) \leq \dim(V)$.
\end{prop}

\begin{prop}[] \label{prop:}
    Let $T \in \mathscr{L}(V, W)$. Then $N(T) = \set{0}$ iff $T$ is an injective transformation.
\end{prop}

\begin{thm}[Rank-Nullity Theorem] \label{thm:rank nullity}
    Let $T \in \mathscr{L}(V, W)$, where $V$ is a finite-dimensional vector space. Then \[
        \dimn(N(T)) + \dimn(R(T)) = \dimn(V).
    \]
\end{thm}

\begin{cor}[] \label{cor:}
    If $\dimn W < \dimn V$, then there is no injective linear transformation from $V$ to $W$.
\end{cor}

\begin{cor}[] \label{cor:}
    Let $T \in \mathscr{L}(V, W)$ where $\dimn V = \dimn W$. Then the following are equivalent:
    \begin{enumerate}[label=(\alph*)]
        \item $T$ is surjective.
        \item $T$ is injective.
        \item $T$ is invertible \emph{as a linear transformation}. That is, there exists $T^{-1} \in \mathscr{L}(W, V)$ such that $T^{-1}T = I_{W}$ and $TT^{-1} = I_{V}$.
    \end{enumerate}
\end{cor}

\end{document}

