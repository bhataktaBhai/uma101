\lecture{4}{wed 26 oct 2022}{Natural Numbers}
\subsection{Natural Numbers}

The ZFC axioms give us the existence of $\N = \set{0, 1, 2, \dots}$ with definitions as follows:
\begin{align*}
    0 &:= \varnothing \\
    1 &:= 0^{+} = \varnothing \cup \set{\varnothing} = \set{\varnothing} = \set{0} \\
    2 &:= 1^{+} = 1 \cup \set{1} = \set{0, 1} \\
    3 &:= 2^{+} = 2 \cup \set{2} = \set{0, 1, 2}
\end{align*}
This $\N$ is the minimal inductive set $\omega$.

\begin{defn}[Peano addition] \label{defn:N:addition}
    Given a fixed $m \in \N$, the \nameref{thm:zfc:recursion} gives a unique function \[
        \mathrm{sum}_{m} : \N \to \N
    \]
    \begin{enumerate}[label=(A\arabic*)]
        \item\label{defn:N:addition:zero} $\psum{m}{0} = m$ 
        \item\label{defn:N:addition:recursion} $\psum{m}{n^{+}} = (\psum{m}{n})^{+}$.
    \end{enumerate}
    Define \[
        m + n := \psum{m}{n}
    \]
\end{defn}

\begin{prop} \label{thm:N:2+3=5}
    $2 + 3 = 5$
\end{prop}
\begin{proof}
    \begin{align*}
        2 + 3 &= \psum{2}{3} \\
        &= \psum{2}{2^{+}} \\
        &= (\psum{2}{2})^{+} \tag{\labelcref{defn:N:addition:recursion}}\\
        &= (\psum{2}{1^{+}})^{+} \\
        &= ((\psum{2}{1})^{+})^{+} \tag{\labelcref{defn:N:addition:recursion}}\\
        &= ((\psum{2}{0^{+}})^{+})^{+} \\
        &= (((\psum{2}{0})^{+})^{+})^{+} \tag{\labelcref{defn:N:addition:recursion}}\\
        &= ((2^{+})^{+})^{+} \tag{\labelcref{defn:N:addition:zero}}\\
        &= (3^{+})^{+} \\
        &= 4^{+} \\
        &= 5 \qedhere
    \end{align*}
\end{proof}

\begin{rem}
    Note that $m^{+} =$ sum$_{m}(0)^{+} = $ sum$_{m}(0^{+}) = $ sum$_{m}(1) = m + 1$. \\
    So we will now denote $m^{+}$ as $m + 1$.
\end{rem}

\begin{defn}[Peano multiplication] \label{defn:N:multiplication}
    Let $m \in \N$. By the recursion principle, $\exists$ a unique function \[
        \mathrm{prod}_{m} : \N \to \N
    \]
    such that
    \begin{enumerate}[label=(\alph*)]
        \item\label{defn:N:multiplication:zero} $\pprod{m}{0} = 0$
        \item\label{defn:N:multiplication:recursion} $\pprod{m}{n^{+}} = m + \pprod{m}{n}$.
    \end{enumerate}
\end{defn}

\begin{thm} \label{thm:N:properties}
    The following hold:
    \begin{enumerate}[label=(\alph*)]
        \item\label{thm:N:properties:commutativity} (Commutativity)
        \begin{align*}
            m + n &= n + m \\
            m \cdot n &= n \cdot m
        \end{align*}
        for all natural numbers $m$ and $n$.

        \item\label{thm:N:properties:associativity} (Associativity)
        \begin{align*}
            m + (n + k) &= (m + n) + k \\
            m \cdot (n \cdot k) &= (m \cdot n) \cdot k
        \end{align*}
        for all natural numbers $m, n, k$.

        \item\label{thm:N:distributivity} (Distributivity) \[
            m \cdot (n + k) = (m \cdot n) + (m \cdot k)
        \] for all natural numbers $m, n, k$.
        \item\label{thm:N:m+n=0=>m=n=0} $m + n = 0 \iff m = n = 0$ for any $m, n \in \N$

        \item\label{thm:N:m.n=0=>m=0|n=0} $m \cdot n = 0 \iff m = 0$ or $n = 0$ for any $m, n \in \N$

        \item\label{thm:N:cancellation} (Cancellation) $m + k = n + k \iff m = n$ for any $m, n, k \in \N$ and if $m \cdot k = n \cdot k$ and $k \neq 0$, then $m = n$.
    \end{enumerate}
\end{thm}

\begin{proof}
    \begin{enumerate}[label=(\alph*)]
        \item[\labelcref{thm:N:cancellation}] We will prove $P(k)$ is true $\forall\; k \in \N$, where $P(k)$ is the property that for some fixed $m, n$, we have $m + k = n + k \iff m = n$ for all $m, n \in \N$ \\
        $P(0)$ is true as $m + 0 = m$, $n + 0 = n$, so $m + 0 = n + 0 \iff m + n$. \\
        Suppose $P(k)$ holds. Then $m + k^{+} = n + k^{+} \iff (m + k)^{+} = (n + k)^{+} \iff m + k = n + k$ (P3) which implies $m = n$ by $P(k)$. \\
        Since $m, n$ were arbitrary, $P(k)$ holds for any value of $m, n$.
        \qedhere
    \end{enumerate}
\end{proof}

\subsubsection{Tao} \vskip 0.5cm
\begin{lem} \label{thm:N:Tao:n+0=n}
For any natural number $n$, $n+0=n$.
\end{lem}

\begin{proof}
    Let $P(n)$ be the property that $n+0=n$. From \labelcref{defn:N:addition:zero}, we have $0+0=0$, \emph{i.e.}, $P(0)$ is true. Suppose $P(n)$ is true for some natural number $n$. Then $n_{++}+0=(n+0)_{++}=n_{++}$, \emph{i.e.}, $P(n_{++})$ is true. By P5, we have $P(n)$ true for every natural number $n$.
\end{proof}

\begin{lem} \label{thm:N:Tao:n+S(m)=S(n+m)}
For any natural numbers $n$ and $m$, $n+m_{++}=(n+m)_{++}$
\end{lem}

\begin{proof}
    Let P(n, m) be the property
    \begin{equation*}
        n+m_{++}=(n+m)_{++}
    \end{equation*}
    $P(0, m)$ is clearly true. \\
    Suppose P(n, m) is true for some n. Then
    \begin{align*}
        n_{++}+m_{++} &= (n + m_{++})_{++}  \tag{\labelcref{defn:N:addition:recursion}} \\
                      &= ((n+m)_{++})_{++}  \tag{induction hypothesis} \\
                      &= (n_{++}+m)_{++}    \tag{\labelcref{defn:N:addition:recursion}}
    \end{align*}
    Therefore $P(n_{++}, m)$ holds. Since we made no assumptions on m, $P(n, m)$ holds for all $n, m$ by \labelcref{defn:peano:induction}
\end{proof}

\begin{cor} \label{thm:N:Tao:S(n)=n+1}
    $n_{++} = n + 1$.
\end{cor}

\begin{proof}
    From \cref{thm:N:Tao:n+0=n,thm:N:Tao:n+S(m)=S(n+m)}, $n + 1 = n + 0_{++} = (n + 0)_{++} = n_{++}$.
\end{proof}

\begin{prop} \label{thm:N:Tao:addition_is_commutative}
\emph{(Addition is commutative)} For any natural numbers $n$ and $m$, $n + m = m + n$
\end{prop}

\begin{proof}
    Let $P(n, m) \iff n + m = m + n$
    \begin{align*}
        n_{++} + m &= (n + m)_{++} \tag{\labelcref{defn:N:addition:recursion}} \\
                   &= (m + n)_{++} \tag{induction hypothesis} \\
                   &= m + n_{++}   \tag{\cref{thm:N:Tao:n+S(m)=S(n+m)}}
    \end{align*}
    Thus $P(n, m) \implies P(n_{++}, m)$ and we know $P(0, m)$ to be true by \labelcref{defn:N:addition:zero}.

    Since there were no assumptions on $m$, $P(n, m)$ is true for all $n$, $m$ by \labelcref{defn:peano:induction}.
\end{proof}

