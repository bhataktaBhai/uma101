\documentclass[12pt]{article}
\usepackage[utf8]{inputenc}
\usepackage[a4paper, total={6in, 9in}]{geometry}
\usepackage{amsmath}
\usepackage{amssymb}
\usepackage{amsthm}
\usepackage{enumitem}
\usepackage{outlines}

\usepackage{paracol}

\usepackage[dvipsnames]{xcolor}
\colorlet{exercise}{red!80!black}
\colorlet{solved}{green!30!black}
\colorlet{self_proof}{blue!30!black}

\usepackage{mathrsfs}
\usepackage{parskip}

\usepackage{accents}
\newcommand{\ubar}[1]{\underaccent{\bar}{#1}}

\usepackage{hyperref}
\usepackage{cleveref}

\providecommand{\dd}{\,\mathrm{d}}

\usepackage{mathtools} % also contains \coloneqq
\DeclarePairedDelimiter{\oldparen}{(}{)}
\DeclarePairedDelimiter{\oldset}{\{}{\}}
\DeclarePairedDelimiter{\oldabs}{\lvert}{\rvert}
\DeclarePairedDelimiter{\oldnorm}{\lVert}{\rVert}
\DeclarePairedDelimiter{\oldfloor}{\lfloor}{\rfloor}
\DeclarePairedDelimiter{\oldceil}{\lceil}{\rceil}

\makeatletter
\def\paren{\@ifstar{\oldparen}{\oldparen*}}
\def\set{\@ifstar{\oldset}{\oldset*}}
\def\abs{\@ifstar{\oldabs}{\oldabs*}}
\def\norm{\@ifstar{\oldnorm}{\oldnorm*}}
\def\floor{\@ifstar{\oldfloor}{\oldfloor*}}
\def\ceil{\@ifstar{\oldceil}{\oldceil*}}

\g@addto@macro\normalsize{%
  \setlength\abovedisplayskip{7pt}%
  \setlength\belowdisplayskip{7pt}%
  \setlength\abovedisplayshortskip{1pt}%
  \setlength\belowdisplayshortskip{1pt}%
}
\makeatother

\newcommand\N{\ensuremath{\mathbb{N}}}
\newcommand\R{\ensuremath{\mathbb{R}}}
\newcommand\Z{\ensuremath{\mathbb{Z}}}
\renewcommand\O{\ensuremath{\varnothing}}
\renewcommand\P{\ensuremath{\mathbb{P}}}
\newcommand\Q{\ensuremath{\mathbb{Q}}}
\newcommand\C{\ensuremath{\mathbb{C}}}

% make implied by and implies arrows shorter
\let\implies\Rightarrow
\let\impliedby\Leftarrow
\let\iff\Leftrightarrow

% make \epsilon and \varepsilon the same
\let\epsilon\varepsilon

% <theorems>
% \theoremstyle{plain}
% \newtheorem{thm}{Theorem}[section]
% \newtheorem*{thm*}{Theorem}
% \newtheorem{prop}[thm]{Proposition}
% \newtheorem{lem}[thm]{Lemma}
% \newtheorem{cor}[thm]{Corollary}
% \newtheorem{axiom}{Axiom}[section]
% \newtheorem*{exer}{Exercise}
% 
% \theoremstyle{definition}
% \newtheorem{defn}[thm]{Definition}
% 
% \theoremstyle{remark}
% \newtheorem*{rem}{Remark} 
% \newtheorem*{example}{Example}
% </theorems>

% Gilles Castel's theorems
\usepackage[framemethod=Tikz]{mdframed}
\mdfsetup{skipabove=1em,skipbelow=0em}
\mdfdefinestyle{axiomstyle}{
    outerlinewidth = 1.5,
    roundcorner = 10,
    leftmargin = 15,
    rightmargin = 15,
    backgroundcolor = blue!10
}
\mdfdefinestyle{defnstyle}{
    outerlinewidth = 1.5,
    % roundcorner = 10,
    leftmargin = 7,
    rightmargin = 7,
    backgroundcolor = green!10
}
\mdfdefinestyle{thmstyle}{
    outerlinewidth = 1.5,
    roundcorner = 10,
    leftmargin = 7,
    rightmargin = 7,
    backgroundcolor = yellow!10
}
\mdfdefinestyle{lemmastyle}{
    outerlinewidth = 1.5,
    roundcorner = 10,
    leftmargin = 7,
    rightmargin = 7,
    backgroundcolor = yellow!10
}
\theoremstyle{definition}
\newmdtheoremenv[nobreak=true, style=axiomstyle]{axiom}{Axiom}[section]
\newmdtheoremenv[nobreak=true, style=thmstyle]{thm}{Theorem}[section]
\newmdtheoremenv[nobreak=true]{prop}[thm]{Proposition}
\newmdtheoremenv[nobreak=true]{lem}[thm]{Lemma}
\newmdtheoremenv[nobreak=true]{cor}[thm]{Corollary}
\newmdtheoremenv[nobreak=true, style=defnstyle]{defn}[thm]{Definition}

\newcounter{assignment}
\newmdtheoremenv[nobreak=true]{problem}{Problem}[assignment]

\theoremstyle{remark}
\newtheorem*{rem}{Remarks}
\newtheorem*{example}{Example}
\newtheorem*{examplex}{Examples}

\newenvironment{examples}
{\begin{examplex}\leavevmode\begin{itemize}}{\end{itemize}\end{examplex}}
% <cref>
\crefname{thm}{theorem}{theorems}
\crefname{prop}{proposition}{propositions}
\crefname{lem}{lemma}{lemmas}
\crefname{cor}{corollary}{corollaries}
\crefname{axiom}{axiom}{axioms}
\crefname{defn}{definition}{definitions}
\crefname{problem}{problem}{problems}
% </cref>

% <hyperlinks>
\hypersetup{colorlinks,
    linkcolor={red!50!black},
    citecolor={blue!50!black},
    urlcolor={blue!80!black}}
% </hyperlinks>

\usepackage{xifthen}
\def\testdateparts#1{\dateparts#1\relax}
\def\dateparts#1 #2 #3 #4 #5\relax{
    \marginpar{\small\textsf{\mbox{#1 #2 #3 #5}}}
}

\def\@lecture{}%
\newcommand*{\lecture}[3]{
    \ifthenelse{\isempty{#3}}{%
        \def\@lecture{Lecture #1}%
    }{%
        \def\@lecture{Lecture #1: #3}%
    }%
    \subsection*{\@lecture}
    \marginpar{\raggedright\small\textsf{#2}}
    \vskip 6pt
}
\newcommand*{\assignment}[2]{%
    \stepcounter{assignment}%
    \subsection{Assignment #1}
    \marginpar{\raggedright\small{due #2}}
    \vskip 6pt
}
\makeatletter
\newcommand*{\refifdef}[3]{%label,command,fallback
    \@ifundefined{r@#1}{#3}{#2{#1}}%
}
\makeatother

\DeclareMathOperator{\dom}{dom}
\DeclareMathOperator{\ran}{ran}
\DeclareMathOperator{\spann}{span}
\DeclareMathOperator{\dimn}{dim}

\newcommand{\psum}[2]{\ensuremath{\mathrm{sum}_{#1}({#2})}}
\newcommand{\pprod}[2]{\ensuremath{\mathrm{prod}_{#1}({#2})}}


\title{Assignment 01}
\author{Naman Mishra}
\date{22 October 2022}

\begin{document}
\maketitle

\begin{problem}
    Let $A$ be a Peano set and $S$ be the successor function on $A$ (as defined in the first lecture).
    Show, using only the axioms of Peano, that the range of $S$ is $A \setminus \set{0}$.
    For this question, please interpret the words ``function'' and ``range'' in the way you did in school, and not in the set-theoretic way introduced in class.
\end{problem}
\begin{proof}
    Let $P(n)$ be the property that $n$ is in the range of $S$, or $n = 0$.
    $P(0)$ is trivially true.
    If $P(n)$ is true, then $P(S(n))$ is trivially true, as $S(n)$ is the successor of $n$.

    Thus by the principle of mathematical induction, $P(n)$ is true for all natural $n$.
    
    This means that for all $n \neq 0 \in A$, $n$ is in the range of $S$.
    Since $0$ is not the successor of any natural number, it is not in the range of $S$.

    Thus, the range of $S$ is precisely $A \setminus \set{0}$.
\end{proof}

\begin{problem}
    We mentioned in class that when listing the ZFC axioms, we do not need to addadditional axioms for the existence of the intersection or the set-difference of two sets.
    Using the ZFC axioms, prove the following statements.
    \begin{enumerate}[label=(\alph*)]
        \item Given two sets $A$ and $B$, show that $A \cap B$ exists as a set.
        \item Given two sets $A$ and $B$, show that $A \setminus B$ exists as a set.
    \end{enumerate}
\end{problem}

\begin{proof}
    Using the axiom of specification, we have:
    \begin{enumerate}[label=(\alph*)]
        \item \[
            A \cap B = \set{a \in A : a \in B}
        \]
        \item \[
            A \setminus B = \set{a \in A : a \not\in B} \qedhere
        \]
    \end{enumerate}
\end{proof}

\begin{problem}
    Given two objects $a, b$, let $(a, b)$ denote the set $\set{\set{a}, \set{a, b}}$.
    First argue why the ZFC axioms guarantee the existence of this set.
    Then show that $(a, b) = (c, d)$ (as sets) if and only if $a = c$ and $b = d$.
\end{problem}

\begin{proof}
    By the basic axiom, $a$ and $b$ are sets. \\
    By the axiom of pairing, set $\set{a, a} = \set{a}$ (by axiom of equality) exists. \\
    By the axiom of pairing, set $\set{a, b}$ exists. \\
    Applying the axiom of pairing on these two sets, the set $(a, b) = \set{\{a}, \set{a, b}\}$ exists.

    If $a = c$ and $b = d$, $(a, b) = (c, d)$. \textcolor{exercise}{Do I even need to compare the two sets manually? Invoke any of ZFC? Doesn't this follow directly from the notion of ``equality''?}
    
    If $(a, b) = (c, d)$, then $\set{\{a}, \set{a, b}\} = \set{\{c}, \set{c, d}\}$. By the axiom of equality, $\set{a} \in (c, d) \implies \set{a} = \set{c}$ or $\set{a} = \set{c, d}$. In either case, the axiom of equality again implies that $c \in \set{a} \implies c = a$.

    By the axiom of equality again, $\set{c, d} = \set{a}$ implying $d = a$, or $\set{c, d} = \set{a, b}$ implying $d \in \set{a, b}$. Thus $d$ is either equal to $a$ or to $b$. Similarly $b$ is equal to either $c$ or $d$.

    If $b \neq d$, we must have $d = a$ and $b = c = a = d$, a contradiction. \\
    Thus $d$ is necessarily equal to $b$.
\end{proof}

\begin{problem}
    Prove \refifdef{thm:zfc:inductive_intersection}{\cref}{lemma 1.5 (probably)}, \textit{i.e.}, show that if $\mathscr{F}$ is a non-empty set of inductive sets, then \[
        \bigcap_{A \in \mathscr{F}} A
    \]
    is an inductive set.
\end{problem}

\begin{proof}
    I shall assume $I = \bigcap_{A \in \mathscr{F}}A$ to be defined as
    \[
        \bigcap_{A \in \mathscr{F}}A  = \set{a \in \bigcup_{A \in \mathscr{F}}A : a \in A \;\forall\; A \in \mathscr{F}}
    \]
    Since $\O \in A$ for every $A \in \mathscr{F}$, and $\O$ is also in the union of the sets contained in $\mathscr{F}$, $\O \in I$

    If an element $a \in I$, then $a$ is in every $A \in \mathscr{F}$. Since $A \in \mathscr{F}$, $A$ is an inductive set, and so $a \in A \implies a^{+} \in A$. Thus $a^{+} \in A$ for every $A \in \mathscr{F}$, which implies $a^{+} \in I$.

    These conditions together imply that $I$ is an inductive set.
\end{proof}

\begin{problem}
    Let $A, B, C, D$ be sets.
    Some of the following statements are always true, and the others are sometimes wrong.
    Decide which is which.
    For the ones you declare ``always true'', provide a proof.
    For the others, provide one counterexample each.
    \begin{enumerate}[label=(\alph*)]
        \item $A \times (B \cup C) = (A \times B) \cup (A \times C)$
        \item $(A \times B) \setminus (C \times D) = (A \setminus C) \times (B \setminus D)$
        \item $C \cap (A \setminus B) = A \cap (C \setminus B)$
        \item $C \cup (A \setminus B) = A \cup (C \setminus B)$
    \end{enumerate}
\end{problem}

\begin{proof}
    \begin{enumerate}[label=(\alph*)]
        \item \emph{True.}
        Call the LHS set $P$ and the RHS set $Q$.
        If $(a, x) \in P$, then $a \in A$ and $x \in B \cup C \iff x \in B$ or $x \in C$.
        Thus $a \in A$ and $x \in B$, or $a \in A$ and $x \in C$.
        Thus $(a, x) \in A \times B$ or $(a, x) \in A \times C$.
        Thus $(a, x) \in (A \times B) \cup (A \times C) = Q$.

        If $(a, x) \in Q$, then $(a, x) \in A \times B$ or $(a, x) \in A \times C$.
        Thus $a \in A$ and $x \in B$, or $a \in A$ and $x \in C$.
        Thus $a \in A$, and $x \in B$ or $x \in C$.
        Thus $a \in A$ and $x \in B \cup C$.
        Thus $(a, x) \in A \times (B \cup C) = P$.

        Therefore by the axiom of equality, $P = Q$.

        \item \emph{False.}
        Consider the sets $A = B = C = \set{\O}, D = \O$.
        Then $A \times B = \set{(\O, \O)}$ and $C \times D = \O$.
        Also, $A \setminus C = \O$ and $B \setminus D = \set{\O}$.

        LHS $= \set{(\O, \O)}$ \\
        RHS $= \O \neq$ LHS

        \item \emph{True.}
        Call the LHS set $P$ and the RHS set $Q$.
        \begin{align*}
            P &= \set{c \in C : c \in (A \setminus B)} \\
            &= \set{c \in C : c \in \{a \in A : a \notin B} \} \\
            &= \set{c \in C : c \in A, c \notin B}
        \end{align*}

        Similarly $Q = \set{a \in A : a \in C, a \notin B}$.

        Clearly, every element of $P$ is in $Q$ and vice versa. Thus the two sets are equal.

        \item \emph{False.} Consider the sets $A = \O, B = C = \set{\O}$. Then LHS $= C \cup \O = C = \set{\O}$, but RHS $= \O \cup \O = \O \neq$ LHS. \qedhere
    \end{enumerate}
\end{proof}

\begin{problem}
    Let $A$ be a set. Define a relation \textbf{R} such that for any subsets $B$ and $C$ of $A$,
    \[
        B \text{\textbf{R}} C \iff B \subseteq C
    \]
    Remember that a relation \textbf{R} is a subset of a Cartesian product of sets.
    Is the relation that you’ve defined a function?
\end{problem}

\begin{proof}
    Since $B$ and $C$ are subsets of $A$, they can be precisely the elements of the power set of $A$. So \textbf{R} is a subset of the $P(A) \times P(A)$.

    If $A = \O$, the only subset of $A$ is $\O$.
    So $B$ and $C$ can only take one value, $\O$, and indeed $\O \subseteq \O \iff \O$\textbf{R}$\O$.

    Thus, in this case, \textbf{R} is a function if it is taken to be a relation from $\set{\O}$ to $\set{\O}$.
    However, \textbf{R} could also be taken to be a relation from $\set{\O, \{\O}\}$ to $\set{\O}$, in which case it is no longer a function as $\set{\O}$\textbf{R}$x$ is false for all $x \in \set{\O}$.

    If $A \neq \O$, we have $\O \in P(A)$ and $A \in P(A)$.
    Thus $\O$\textbf{R}$\O$ and $\O$\textbf{R}$A$, with $\O \neq A$.
    Thus for $\O \in P(A)$ there are at least two $x$ such that $(\O, x) \in$ \textbf{R}.
    Therefore \textbf{R} is not a function.
\end{proof}

\end{document}
