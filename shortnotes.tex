\documentclass[12pt]{article}
\usepackage[utf8]{inputenc}
\usepackage[a4paper, total={6in, 9in}]{geometry}
\usepackage{amsmath}
\usepackage{amssymb}
\usepackage{amsthm}
\usepackage{enumitem}
\usepackage{outlines}

\usepackage{paracol}

\usepackage[dvipsnames]{xcolor}
\colorlet{exercise}{red!80!black}
\colorlet{solved}{green!30!black}
\colorlet{self_proof}{blue!30!black}

\usepackage{mathrsfs}
\usepackage{parskip}

\usepackage{accents}
\newcommand{\ubar}[1]{\underaccent{\bar}{#1}}

\usepackage{hyperref}
\usepackage{cleveref}

\providecommand{\dd}{\,\mathrm{d}}

\usepackage{mathtools} % also contains \coloneqq
\DeclarePairedDelimiter{\oldparen}{(}{)}
\DeclarePairedDelimiter{\oldset}{\{}{\}}
\DeclarePairedDelimiter{\oldabs}{\lvert}{\rvert}
\DeclarePairedDelimiter{\oldnorm}{\lVert}{\rVert}
\DeclarePairedDelimiter{\oldfloor}{\lfloor}{\rfloor}
\DeclarePairedDelimiter{\oldceil}{\lceil}{\rceil}

\makeatletter
\def\paren{\@ifstar{\oldparen}{\oldparen*}}
\def\set{\@ifstar{\oldset}{\oldset*}}
\def\abs{\@ifstar{\oldabs}{\oldabs*}}
\def\norm{\@ifstar{\oldnorm}{\oldnorm*}}
\def\floor{\@ifstar{\oldfloor}{\oldfloor*}}
\def\ceil{\@ifstar{\oldceil}{\oldceil*}}

\g@addto@macro\normalsize{%
  \setlength\abovedisplayskip{7pt}%
  \setlength\belowdisplayskip{7pt}%
  \setlength\abovedisplayshortskip{1pt}%
  \setlength\belowdisplayshortskip{1pt}%
}
\makeatother

\newcommand\N{\ensuremath{\mathbb{N}}}
\newcommand\R{\ensuremath{\mathbb{R}}}
\newcommand\Z{\ensuremath{\mathbb{Z}}}
\renewcommand\O{\ensuremath{\varnothing}}
\renewcommand\P{\ensuremath{\mathbb{P}}}
\newcommand\Q{\ensuremath{\mathbb{Q}}}
\newcommand\C{\ensuremath{\mathbb{C}}}

% make implied by and implies arrows shorter
\let\implies\Rightarrow
\let\impliedby\Leftarrow
\let\iff\Leftrightarrow

% make \epsilon and \varepsilon the same
\let\epsilon\varepsilon

% <theorems>
% \theoremstyle{plain}
% \newtheorem{thm}{Theorem}[section]
% \newtheorem*{thm*}{Theorem}
% \newtheorem{prop}[thm]{Proposition}
% \newtheorem{lem}[thm]{Lemma}
% \newtheorem{cor}[thm]{Corollary}
% \newtheorem{axiom}{Axiom}[section]
% \newtheorem*{exer}{Exercise}
% 
% \theoremstyle{definition}
% \newtheorem{defn}[thm]{Definition}
% 
% \theoremstyle{remark}
% \newtheorem*{rem}{Remark} 
% \newtheorem*{example}{Example}
% </theorems>

% Gilles Castel's theorems
\usepackage[framemethod=Tikz]{mdframed}
\mdfsetup{skipabove=1em,skipbelow=0em}
\mdfdefinestyle{axiomstyle}{
    outerlinewidth = 1.5,
    roundcorner = 10,
    leftmargin = 15,
    rightmargin = 15,
    backgroundcolor = blue!10
}
\mdfdefinestyle{defnstyle}{
    outerlinewidth = 1.5,
    % roundcorner = 10,
    leftmargin = 7,
    rightmargin = 7,
    backgroundcolor = green!10
}
\mdfdefinestyle{thmstyle}{
    outerlinewidth = 1.5,
    roundcorner = 10,
    leftmargin = 7,
    rightmargin = 7,
    backgroundcolor = yellow!10
}
\mdfdefinestyle{lemmastyle}{
    outerlinewidth = 1.5,
    roundcorner = 10,
    leftmargin = 7,
    rightmargin = 7,
    backgroundcolor = yellow!10
}
\theoremstyle{definition}
\newmdtheoremenv[nobreak=true, style=axiomstyle]{axiom}{Axiom}[section]
\newmdtheoremenv[nobreak=true, style=thmstyle]{thm}{Theorem}[section]
\newmdtheoremenv[nobreak=true]{prop}[thm]{Proposition}
\newmdtheoremenv[nobreak=true]{lem}[thm]{Lemma}
\newmdtheoremenv[nobreak=true]{cor}[thm]{Corollary}
\newmdtheoremenv[nobreak=true, style=defnstyle]{defn}[thm]{Definition}

\newcounter{assignment}
\newmdtheoremenv[nobreak=true]{problem}{Problem}[assignment]

\theoremstyle{remark}
\newtheorem*{rem}{Remarks}
\newtheorem*{example}{Example}
\newtheorem*{examplex}{Examples}

\newenvironment{examples}
{\begin{examplex}\leavevmode\begin{itemize}}{\end{itemize}\end{examplex}}
% <cref>
\crefname{thm}{theorem}{theorems}
\crefname{prop}{proposition}{propositions}
\crefname{lem}{lemma}{lemmas}
\crefname{cor}{corollary}{corollaries}
\crefname{axiom}{axiom}{axioms}
\crefname{defn}{definition}{definitions}
\crefname{problem}{problem}{problems}
% </cref>

% <hyperlinks>
\hypersetup{colorlinks,
    linkcolor={red!50!black},
    citecolor={blue!50!black},
    urlcolor={blue!80!black}}
% </hyperlinks>

\usepackage{xifthen}
\def\testdateparts#1{\dateparts#1\relax}
\def\dateparts#1 #2 #3 #4 #5\relax{
    \marginpar{\small\textsf{\mbox{#1 #2 #3 #5}}}
}

\def\@lecture{}%
\newcommand*{\lecture}[3]{
    \ifthenelse{\isempty{#3}}{%
        \def\@lecture{Lecture #1}%
    }{%
        \def\@lecture{Lecture #1: #3}%
    }%
    \subsection*{\@lecture}
    \marginpar{\raggedright\small\textsf{#2}}
    \vskip 6pt
}
\newcommand*{\assignment}[2]{%
    \stepcounter{assignment}%
    \subsection{Assignment #1}
    \marginpar{\raggedright\small{due #2}}
    \vskip 6pt
}
\makeatletter
\newcommand*{\refifdef}[3]{%label,command,fallback
    \@ifundefined{r@#1}{#3}{#2{#1}}%
}
\makeatother

\DeclareMathOperator{\dom}{dom}
\DeclareMathOperator{\ran}{ran}
\DeclareMathOperator{\spann}{span}
\DeclareMathOperator{\dimn}{dim}

\newcommand{\psum}[2]{\ensuremath{\mathrm{sum}_{#1}({#2})}}
\newcommand{\pprod}[2]{\ensuremath{\mathrm{prod}_{#1}({#2})}}


\title{UM101: Short Notes}
\author{Naman Mishra}

\begin{document}
\maketitle
\tableofcontents

\section{Set theory \& the real number system}

\begin{defn} \label{defn:peano set}
    The set $A$ along with a successor function $S$ is called a Peano set if it obeys the Peano axioms.
    \begin{enumerate}[label=(P\arabic*)]
        \item \label{peano:zero}
            There is an element called 0 in $A$.
        \item \label{peano:succ}
            For every $a \in A$, its successor $S(a)$ is also in $A$.
        \item \label{peano:not succ}
            $\forall \; a \in A, S(a) \neq 0$.
        \item \label{peano:injective}
            For any $m, n \in A$, $S(m) = S(n)$ only if $m = n$.
        \item \label{peano:induction}
            (principle of mathematical induction) For any set $B \subseteq A$, if $0 \in B$ and $a \in B \implies S(a) \in B$, then $B = A$.
    \end{enumerate}
\end{defn}

\subsection{The ZFC Axioms}

\begin{defn} \label{defn:set}
    A \textbf{set} is a well-defined collection of (mathematical) objects, called the \emph{elements} of that set. To say that $a$ is an element of set $A$, we write $a \in A$. Otherwise, we write $a \notin A$. \\
    Given two sets $A$ and $B$, we say that:
    \begin{enumerate}
        \item[($A \subseteq B$)] $A$ is a subset of $B$, \textit{i.e.}, every element of $A$ is an element of $B$.
        \item[($A \not\subseteq B$)] $A$ is not a subset of $B$, \textit{i.e.}, there is some element in $A$ which is not an element of $B$.
        \item[($A \subsetneq B$)] $A$ is a proper subset of $B$, \textit{i.e.}, $A \subseteq B$ but $\exists\; b \in B$ such that $b \notin A$.
    \end{enumerate}
\end{defn}

\begin{rem}
    We need ZFC axioms because not any collection can be called a set. Read up on Russell's paradox.
\end{rem}

\begin{axiom}[the basic axiom] \label{zfc:basic}
    Every object is a set.
\end{axiom}

\begin{axiom}[axiom of extension] \label{zfc:extension}
    Two sets $A, B$ are equal if they have exactly the same elements. In other words, $A = B \iff A \subseteq B$ and $B \subseteq A$
\end{axiom}

\begin{rem}
    As a consequnce, it doesn't matter whether a set contains multiple copies of an element.
    \begin{align*}
        A &= \set{ 1 } \\
        B &= \set{ 1, 1, 1 }
    \end{align*}
    Clearly $A \subseteq B$ and $B \subseteq A$, implying $A = B$.
\end{rem}

\begin{axiom}[axiom of existence] \label{zfc:existence}
    There is a set with no elements called the empty set, denoted by the symbol $\varnothing$.
\end{axiom}

\begin{axiom}[axiom of specification] \label{zfc:specification}
    Let $A$ be a set. Let $P(a)$ denote a property that applies to every element in $A$, i.e., for each $a \in A$, either $P(a)$ is true or it is false. Then there exists a subset
    \[
        B = \set{a \in A: P(a) \text{ is true}}
    \]
\end{axiom}

\begin{rem}
    We are forced to create sets only as subsets of other sets because of Russell's paradox. \textcolor{red!85!black}{\emph{From MathGarden:} A somewhat surprising result is that the axiom of specification implies for each set $A$ the existence of an element (a set) $x$ such that $x \not\in A$. In other words, there is no set containing all sets of our mathematical universe.}
\end{rem}

\begin{axiom}[axiom of pairing]\label{zfc:pairing}
    Given two sets $A, B$, there exists a set which contains precisely $A, B$ as its elements, which we denote by \set{A, B}.
\end{axiom}

\begin{rem}
    In particular, by letting $A = B$, we get a set containing only $A$, i.e., \set{A}. For example, we can have $\set{\varnothing}$, and $\set{\varnothing, \{\varnothing}\}$, etc.
\end{rem}

\begin{axiom}[axiom of unions] \label{zfc:unions}
    Given a set $\mathscr{F}$ of sets, there exists a set called the union of the sets in $\mathscr{F}$, denoted by $\bigcup_{A \in \mathscr{F}} A$, whose elements are precisely the elements of the elements of $\mathscr{F}$.
    \[
        a \in \bigcup_{A \in \mathscr{F}} A \iff a \in A \text{ for some } A \in \mathscr{F}
    \]
\end{axiom}

\begin{rem}
    Intersection of a nonempty set of two or more sets and difference between two sets need not be defined as they follow from the previous axioms. (Exercise)
\end{rem}

\begin{axiom}[axiom of powers] \label{zfc:powers}
    Given a set $A$, there exists a set called power set of $A$ denoted $\mathscr{P}(A)$, whose elements are precisely all the subsets of $A$.   
\end{axiom}

\begin{rem}
    This axiom allows us to define ordered pairs as sets (assignment) (\textcolor{red!85!black}{Isn't pairing sufficient?}) and thus direct products, relations and functions. \[
        A \times B = \set{(a, b) : a \in A, b \in B}
    \] \textcolor{red!85!black}{How does this set exist?}
\end{rem}

\begin{defn} \label{defn:relation}
    A relation from set $A$ to set $B$ is a subset $R$ of $A \times B$. For any $a \in A, b \in B$ we say $a\mathrel{R}b$ iff $(a, b) \in R$.
    \begin{outline}
        \1 The \emph{domain} of $R$ is the set
        \[
            \text{dom}(R) = \set{ a \in A : (a, b) \in R \text{ for some } b \in B}
        \]
        \1 The \emph{range} of $R$ is the set
        \[
            \text{ran}(R) = \set{ b \in B : (a, b) \in R \text{ for some } a \in A}
        \]
        \1 $R$ is called a \emph{function} from $A$ to $B$, denoted as $R: A \to B$ iff
            \2 dom$(R) = A$
            \2 for each $a \in A$ there is (at most) one $b \in B$ such that $(a, b) \in R$.
    \end{outline}
\end{defn}

\begin{rem}
    A \emph{bijective} function from $A$ to $B$ is an injective as well as surjective fuction from $A$ to $B$.
\end{rem}

\begin{axiom}[axiom of regularity] \label{zfc:regularity}
    Read up
\end{axiom}

\begin{axiom}[axiom of replacement] \label{zfc:replacement}
    Read up
\end{axiom}

\begin{axiom}[axiom of choice] \label{zfc:choice}
    Read up
\end{axiom}

\begin{defn} \label{defn:inductive}
    Given a set $A$, its \emph{successor} is the set \[
        A^{+} = A \cup \set{A}.
    \] A set $A$ is said to be \emph{inductive} if $\varnothing \in A$ and for every $a \in A$, we have $a^{+} \in A$.
\end{defn}

\begin{rem}
    The successor of $A$ is guaranteed to exist by \nameref{zfc:pairing} and \nameref{zfc:unions}. \\
    $\set{A}$ exists by \nameref{zfc:pairing} by letting $B = A$. \\
    $A \cup \set{A}$ exists by applying \nameref{zfc:unions} on the set $\set{A, \{A}\}$ formed using \nameref{zfc:pairing} again.

    $\set{A}$ can also be created as a subset (\nameref{zfc:specification}) of the power set (\nameref{zfc:powers}) of $A$.
\end{rem}

\begin{rem}
    The definition of an inductive set is very similar to the principle of mathematical induction in the Peano axioms.
\end{rem}

\begin{axiom}[axiom of infinity] \label{zfc:infinity}
    There exists an inductive set.    
\end{axiom}

\begin{lem} \label{lem:inductive intersection}
    Let $\mathscr{F}$ be a nonempty set of inductive sets. (This exists by \nameref{zfc:infinity} and \nameref{zfc:pairing}). Then \[
        \bigcap_{A \in \mathscr{F}}{A} \text{ is inductive.}
    \]
\end{lem}

\begin{thm} \label{thm:omega}
    There exists a \emph{unique}, \emph{minimal} inductive set $\omega$, \textit{i.e.}, for any inductive set $S$, \[
        \omega \subseteq S
    \] and if $\omega'$ is any other inductive set satisfying this property, \[
        \omega = \omega'
    \]
\end{thm}

\begin{thm} \label{thm:omega peano}
    The $\omega$ in \cref{thm:omega} is a Peano set with successor function $a \mapsto a^{+}$.
\end{thm}

\begin{thm}[principle of recursion] \label{thm:recursion}
    Let $A$ be a set, and $f: A \to A$ be a function. Let $a \in A$. Then, there exists a function $F: \omega \to A$ such that
    \begin{enumerate}[label=(\alph*)]
        \item $F(\varnothing) = a$
        \item For some $b \in \omega$, we have $F(b^{+}) = f(F(b))$
    \end{enumerate}
\end{thm}

\subsection{Natural Numbers}

\begin{defn}[Peano addition] \label{defn:addn}
    Given a fixed $m \in \N$, the \nameref{thm:recursion} gives a unique function $\text{sum}_{m} : \N \to \N$
    \begin{enumerate}[label=(\alph*)]
        \item sum$_{m} (0) = m$ 
        \item sum$_{m} (n^{+}) = ($sum$_{m}(n))^{+}$
    \end{enumerate}
    Define \[
        m + n := \text{sum}_{m}(n)
    \]
\end{defn}

\begin{prop} \label{prop:2+3=5}
    $2 + 3 = 5$
\end{prop}

\begin{rem}
    Note that $m^{+} =$ sum$_{m}(0)^{+} = $ sum$_{m}(0^{+}) = $ sum$_{m}(1) = m + 1$. \\
    So we will now denote $m^{+}$ as $m + 1$.
\end{rem}

\begin{defn}[Peano multiplication] \label{defn:mult}
    Let $m \in \N$. By the recursion principle, $\exists$ a unique function
    \[
        \text{prod}_{m} : \N \to \N
    \]
    such that
    \begin{enumerate}[label=(\alph*)]
        \item prod$_{m}(0) = 0$
        \item prod$_{m}(n^{+}) = m + $ prod$_{m}(n)$
    \end{enumerate}
\end{defn}

\begin{thm} \label{thm:properties}
    The following hold:
    \begin{enumerate}[label=(\alph*)]
        \item\label{thm:comm} (Commutativity)
        \[
            m + n = n + m
        \]
        \[
            m \cdot n = n \cdot m
        \]
        for all natural numbers $m$ and $n$.

        \item\label{thm:asso} (Associativity)
        \[
            m + (n + k) = (m + n) + k
        \] \[
            m \cdot (n \cdot k) = (m \cdot n) \cdot k
        \]
        for all natural numbers $m, n, k$.

        \item\label{thm:dist} (Distributivity)
        \[
            m \cdot (n + k) = (m \cdot n) + (m \cdot k)
        \]
        
        \item\label{thm:m+n=0} $m + n = 0 \iff m = n = 0$ for any $m, n \in \N$

        \item $m \cdot n = 0 \iff m = 0$ or $n = 0$ for any $m, n \in \N$

        \item (Cancellation) $m + k = n + k \iff m = n$ for any $m, n, k \in \N$ and if $m \cdot k = n \cdot k$ and $k \neq 0$, then $m = n$.
    \end{enumerate}
\end{thm}

\subsubsection{Tao}

\begin{lem} \label{lem:n+0=n}
For any natural number $n$, $n+0=n$
\end{lem}

\begin{lem} \label{lem:n+succ(m)=succ(n+m)}
For any natural numbers $n$ and $m$, $n+m_{++}=(n+m)_{++}$
\end{lem}

\begin{cor} \label{cor:succ(n)=n+1}
    $n_{++} = n + 1$
\end{cor}

\begin{prop} \label{prop:addn is commutative}
\emph{(Addition is commutative)} For any natural numbers $n$ and $m$, $n + m = m + n$
\end{prop}

\subsection{Fields, Ordered Sets and Ordered Fields}

\begin{defn} \label{defn:field}
    A field is a set $F$ with 2 operations $+ : F \times F \to F$ and $\cdot : F \times F \to F$ such that
    \begin{enumerate}[label=(F\arabic*)]
        \item \label{field:comm}
            $+$ \& $\cdot$ are commutative on $F$.
        \item \label{field:asso}
            $+$ \& $\cdot$ are associative on $F$.
        \item \label{field:dist}
            $+$ \& $\cdot$ satisfy distributivity on $F$, \textit{i.e.}, $a \cdot (b + c) = a \cdot b + a \cdot c$ for all $a, b, c \in F$. 
        \item \label{field:iden}
            There exist 2 \emph{distinct} elements, called 0 (additive identity) and 1 (multiplicative identity) such that
            \begin{align*}
                x + 0 &= x \\
                x \cdot 1 &= x
            \end{align*}
            for all $x \in F$
        \item \label{field:negative}
            For every $x \in F, \,\exists\; y \in F$ such that \[
                x + y = 0
            \]
        \item \label{field:reciprocal}
            For every $x \in F \setminus \set{0}, \,\exists\; z \in F$ such that \[
                x \cdot z = 1
            \]
    \end{enumerate}
\end{defn}

\begin{rem}
    We are tempted to call $y$ in \labelcref{field:negative} ``-x'' and $z$ in \labelcref{field:reciprocal} ``$\frac{1}{x}$'' but $y, z$ haven't been proven to be unique yet. \textcolor{red!85!black}{Prove this}. \textcolor{green!30!black}{Proved as \cref{lem:unique inverses}} \\
    Once we have proven this, we can also define $a - b := a + (-b)$ and $a/b = a \cdot \frac{1}{b}$.
\end{rem}

\begin{thm} \label{thm:0x=0}
    $(F, +, \cdot)$ is a field. Then for all $x$, \[
        0 \cdot x = x \cdot 0 = 0
    \]
\end{thm}

\begin{defn} \label{defn:ordered set}
    A set $A$ with a relation $<$ is called an \emph{ordered set} if
    \begin{enumerate}[label=(O\arabic*)]
        \newcounter{temp}
        \item \label{order:trichotomy}
            (Trichotomy) For every $x, y \in A$, exactly one of the following holds. \[
                x < y, \quad x = y, \quad y < x
            \]
        \item \label{order:transitivity}
            (Transitivity) If $x < y$ and $y < z$, then $x < z$.
        \setcounter{temp}{\value{enumi}}
    \end{enumerate}
    \textbf{Notation:} $x < y$ is read as ``x is less than y'' \\
    $x \leq y$ means $x < y$ or $x = y$, read as ``x is less that or equal to y''. \\
    $x > y$ is read as ``$x$ is greater that $y$'' and equivalent to $y < x$.
\end{defn}

\begin{defn} \label{defn:ordered field}
    An \emph{ordered field} is a set that admits two operations $+$ and $\cdot$ and relation $<$ so that $(F, +, \cdot)$ is a field and $(F, <)$ is an ordered set and:
    \begin{enumerate}[label=(O\arabic*)]
        \setcounter{enumi}{\value{temp}}
        \item \label{order:addn preserves}
            For $x, y, z \in F$, if $x < y$ then $x + z < y + z$.
        \item \label{order:mult preserves}
            For $x, y \in F$, if $0 < x$ and $0 < y$ then $0 < x \cdot y$.
    \end{enumerate}
\end{defn}

\begin{lem} \label{lem:unique inverses}
    Given a field $(F, +, \cdot)$: For any element $a$ in a field $F$, there exists ony one $b$ such that $a + b = 0$. We will denote this $b$ as $-a$. Similarly for any $a$ in $F \setminus \set{0}$ there exists only one $b \in F$ such that $ab = 1$. We will denote this $b$ as $\frac{1}{a}$ or $a^{-1}$.
\end{lem}

\begin{lem} \label{lem:inverse involution}
    $-(-a) = a = (a^{-1})^{-1}$
\end{lem}

\begin{lem} \label{lem:(-a)b=-(ab)}
    For any field $(F, +, \cdot)$, $(-a)b = -(ab)$ and $(-a)(-b) = ab$.
\end{lem}

\begin{thm} \label{thm:0<1}
    For any field $(F, +, \cdot)$, $0 < 1$.
\end{thm}

\begin{rem}
    ``a contradiction'' is not necessary to state for the proof to be complete. See \href{https://teams.microsoft.com/l/message/19:5PNDOetYK3gbPZWX5Muk\_KnaEXgulRmRNwNmAHA8dZ81@thread.tacv2/1666960265828?tenantId=6f15cd97-f6a7-41e3-b2c5-ad4193976476&groupId=9cf683b7-9233-4d97-a7eb-1dc0051039a7&parentMessageId=1666960265828&teamName=UM\%20101\%20October\%202022&channelName=General&createdTime=1666960265828&allowXTenantAccess=false}{this discussion} at MS Teams.
\end{rem}

\subsection{Upper bounds \& least upper bounds}

\begin{defn} \label{defn:bounded above}
    A non-empty subset $S \subseteq F$ is said to be \emph{bounded above} in $F$ if there exists a $b \in F$ such that \[
        a \leq b \;\forall\; a \in S
    \]
    Here, $b$ is called an \emph{upper bound} of $S$. If $b \in S$, then $b$ is a \emph{maximum} of $S$.
\end{defn}

\begin{example}
    \begin{align*}
        S &= \set{ x \in F : 0 \leq x \leq 1 } \\
        T &= \set{ x \in F : 0 \leq x < 1 }
    \end{align*}
    Both $S$ and $T$ are bounded above as $1$ is an upper bound for both. \\
    $1$ is in fact, a maximum of S.
\end{example}

\begin{rem}
    If a maximum exists, it must be unique (\textcolor{red!85!black}{why?}).
\end{rem}

\begin{rem}
    Upper bounds may not be unique.
\end{rem}

\begin{defn} \label{defn:supremum}
    Let $S \subseteq F$ be bounded above. An element $b \in F$ is said to be a \emph{least upper bound} of $S$ or a \emph{supremum} of $S$ if:
    \begin{enumerate}[label=(\alph*)]
        \item $b$ is an upper bound of $S$.
        \item If for $c \in F$, $c < b$, then $c$ is not an upper bound of $S$. In other words, for any $c < b, \;\exists\; s_{c} \in S$ such that $c < s_{c}$. \\
        Contrapositive: If $c$ is an upper bound of $S$, then $c$ is not less than $b$ $\iff c \geq b$.
    \end{enumerate}
\end{defn}

\begin{rem}
    There is only one supremum of $S$.
\end{rem}

\begin{example}
    \[
        \sup \set{x \in F : 0 \leq x < 1 } = 1
    \]
\end{example}

\subsection{The Real Numbers}

\begin{thm}[Archimedean property of $\R$] \label{thm:archimedean}
    Let $x, y \in \R$ and $x > 0$, then $\exists\; n \in \N$ such that \[
        n \cdot x > y
    \]
\end{thm}

\section{Sequences \& Series}

\subsection{Sequences}

\begin{defn} \label{defn:sequence}
    A sequence in $\R$ is a function $f: \N \to \R$. We denote this sequence by $\set{a_{n}}_{n \in \N}$, where \[
        a_{n} = f(n) \quad \forall\; n \in \N
    \] and $a_{n}$ is called the $n^{th}$ term of $\set{a_{n}}_{n \in \N}$.
\end{defn}

\begin{rem}
    $\set{a_{n}} \subseteq \R$ will denote a sequence of real numbers. \\
    The numbering starts at 0 technically, but doesn't really matter. We will often omit the subscript $n \in \N$ and start indexing from some other point.
\end{rem}

\begin{defn} \label{defn:convergent seq}
    We say that a sequence $\set{a_{n}} \subseteq \R$ is \emph{convergent} (in $\R$) if $\exists\; L \in \R$ such that for each $\varepsilon > 0, \,\exists\; N_{\varepsilon,L} \in \N$ such that \[
        \abs{ a_{n} - L } < \varepsilon \quad \forall\; n \geq N_{\varepsilon, L}
    \] 
    We will call $L$ \emph{a} limit of $\set{a_{n}}$ and we write: \[
        a_{n} \to L \text{ as } n \to \infty
    \]
    A sequence $\set{a_{n}}$ is said to be \emph{divergent} if it is not convergent, \textit{i.e.}, $\forall\; L \in \R$ and $N_{L} \in \N$, $\exists\; \varepsilon > 0$ and $N \geq N_{L}$ such that \[
        \abs{ a_{N} - L } > \varepsilon
    \]
\end{defn}

\begin{thm}[Uniqueness of limits] \label{thm:unique limit}
    Suppose $L_{1}$ and $L_{2}$ are limits of a (convergent) sequence $\set{a_{n}} \in \R$. Then $L_{1} = L_{2}$.
\end{thm}

\begin{example}
    \begin{enumerate}[label=(\alph*)]
        \item Let $\set{a_{n}} = \frac{1}{n^{p}} \,\forall\; n \in \P$, where $p > 0$. \[
            \lim_{n \to \infty} a_{n} = 0
        \]
        \begin{proof}
            Let $\varepsilon > 0$. \\
            By the Archimedean property of $\R$ applied to $x = \varepsilon^{\frac{1}{p}}$ and $y = 1$, $\exists\; N \in \P$ such that: \[
                N \varepsilon^{\frac{1}{p}} > 1 \implies \varepsilon^{\frac{1}{p}} > \frac{1}{N} \implies \varepsilon > \frac{1}{N^{p}}
            \]
            Let $n \geq N$. Then
            \begin{align*}
                \abs{ \frac{1}{n^{p}} - 0 } &= \frac{1}{n^{p}} \\
                                     &\leq \frac{1}{N^{p}} \\
                                     &< \varepsilon \qedhere
            \end{align*}
        \end{proof}

        \item $\set{(-1)^{n}}_{n \in \P}$ is divergent.
        \begin{proof}
            Suppose there exists a limit $L$. \\
            Let $\varepsilon = 1$. \\
            Then $\exists\; N \in \P$ such that $\abs{ a_{n} - L } < \varepsilon$ for all $n \geq N$. \\
            $\abs{ a_{2N} - L } < 1 \implies \abs{ L - 1 } < 1$. \\
            $\abs{ a_{2N + 1} - L } < 1 \implies \abs{ L + 1 } < 1$. \\
            $\abs{ 1 - L + L + 1 } \leq \abs{ 1 - L } + \abs{ L + 1 } < 2$ \\
            $\implies 2 < 2$. Contradiction.
        \end{proof}
    \end{enumerate}
\end{example}

\begin{defn} \label{defn:bounded seq}
    A sequence $\set{a_{n}}_{n \in \N}$ is said to be \emph{bounded} if $\exists\; M > 0$ such that $\abs{ a_{n} } < M \,\forall\; n \in \N$.
\end{defn}

\begin{thm} \label{thm:convergent=>bounded}
    Every convergent sequence is bounded.
\end{thm}

\begin{defn} \label{defn:monotone seq}
    A sequence $\set{a_{n}} \subseteq \R$ is said to be \emph{monotonically increasing} if $a_{n} \leq a_{n+1} \,\forall\; n \in \N$. \\
    A sequence $\set{a_{n}} \subseteq \R$ is said to be \emph{monotonically decreasing} if $a_{n} \geq a_{n+1} \,\forall\; n \in \N$. \\
    A sequence $\set{a_{n}} \subseteq \R$ is said to be \emph{monotone} if it is either monotonically increasing or monotonically decreasing.
\end{defn}

\begin{thm} \label{thm:MCT}
    A monotone sequence is convergent iff it is bounded.
\end{thm}

\begin{rem}[Warning!]
    Divergent sequences may diverge for different reasons!
    \begin{itemize}
        \item $\set{(-1)^{n}}$ is bounded but divergent.
        \item $\set{n}$ is unbounded and divergent, to $+\infty$
        \item $\set{(-1)^{n}n}$ is unbounded and divergent, but not to $\pm \infty$.
    \end{itemize}
\end{rem}

\begin{defn} \label{defn:diverging to infinity}
    We say that a sequence diverges to $+\infty$ if $\forall\; R \in \R$, $\exists\; N_{R} \in \N$ such that $a_{n} > R \;\forall\; n \geq N_{R}$. \\
    We say that a sequence diverges to $-\infty$ if $\forall\; R \in \R$, $\exists\; N_{R} \in \N$ such that $a_{n} < R \;\forall\; n \geq N_{R}$. \\
    We write $\lim_{n \to \infty} a_{n} = +\infty$ or $\lim_{n \to \infty} a_{n} = -\infty$, but this is purely notational and does not mean ``$\set{a_{n}}$ has a limit''.
\end{defn}

\begin{thm}[Tao Theorem 6.1.19] \label{thm:}
    Suppose $\set{b_{n}}$ converges to $b \neq 0$ (and $\exists\; M \in \N$ such that $b_{n} \neq 0 \;\forall\; n \geq M$.) \\
    Then $\set{ \frac{1}{b} }_{n \geq M} \to \frac{1}{b}$ as $n \to \infty$.
\end{thm}

\subsection{Infinite series}

\begin{defn} \label{defn:infinite series}
    An infinite series is a \emph{formal expression} of the form \[
        a_{0} + a_{1} + a_{2} + \dots, \text{ or }, \sum_{n=0}^{\infty} a_{n}
    \]
    Given $\sum_{n=0}^{\infty} a_{n}$, its sequence of partial sums (sops) is $\set{s_{n}}_{n=0}^{\infty}$ where
    \begin{align*}
        s_{0} &= a_{0} \\
        s_{1} &= a_{0} + a_{1} \\
        .\\
        .\\
        s_{n} &= a_{0} + a_{1} + \dots a_{n}
    \end{align*}
    We say that $\sum a_{n}$ is \emph{convergent} with sum $s$ if $\lim_{n \to \infty} s_{n} = s$. Otherwise, we say that $\sum a_{n}$ is divergent.
\end{defn}

\begin{example}
    \begin{enumerate}[label=(\alph*)]
        \item (Harmonic series) $\sum_{n=1}^{\infty} \frac{1}{n}$ is divergent.
        \begin{proof}
            $\set{s_{n}}$ is a monotonically increasing sequence.
            \begin{align*}
                s_{1} &= 1 \\
                s_{2} &= 1 + \frac{1}{2} \\
                s_{4} &= 1 + \frac{1}{2} + \frac{1}{3} + \frac{1}{4} > 1 + \frac{1}{2} + \frac{1}{4} + \frac{1}{4} \\
                s_{8} &= 1 + \frac{1}{2} + \frac{1}{3} + \dots + \frac{1}{8} \\
                   &> 1 + \frac{1}{2} + 2 \cdot \frac{1}{4} + 4 \cdot \frac{1}{8} \\
                s_{2^{k}} &= 1 + \frac{1}{2} + \frac{1}{3} + \dots + \frac{1}{2^{k}} \\
                   &> 1 + \frac{1}{2} + 2 \cdot \frac{1}{4} + \dots + 2^{k-1} \cdot \frac{1}{2^{k}} \\
                   &= 1 + \frac{k}{2}
            \end{align*}
            Thus, given any $R \in \R, \;\exists\; k \in \N$ such that $s_{2^{k}} > R$. \\
            $\implies \set{s_{n}}$ is divergent by MCT.
        \end{proof}
        \item $\sum_{n=1}^{\infty} \frac{1}{n^{2}}$ is convergent.
        \begin{proof}
            \begin{align*}
                s_{1} &= 1 \\
                s_{n} &= 1 + \sum_{k=2}^{n} \frac{1}{k^{2}} \\
                &< 1 + \sum_{k=2}^{n} \frac{1}{k(k-1)} \\
                &= 1 + \sum_{k=2}^{n} \paren{\frac{1}{k-1} - \frac{1}{k}} \\
                &= 1 + 1 - \frac{1}{n} \\
                &< 2 \;\forall\; n \in \N
            \end{align*}
            So $\set{s_{n}}$ is a monotonically increasing sequence that is bounded above. \\
            $\implies \set{s_{n}}$ is convergent.
        \end{proof}
    \end{enumerate}
\end{example}

\begin{rem}
    (Telescoping sum)
\end{rem}

\begin{thm} \label{thm:divergence test}
    Suppose $\sum a_{n}$ is convergent. Then \[
        \lim_{n \to \infty} a_{n} = 0
    \]
\end{thm}

\begin{example}[Geometric Series]
    Let $x \in \R$. Then \[
        \sum_{n=0}^{\infty} x^{n} = \begin{cases}
            \frac{1}{1-x} & \abs{ x} < 1 \\
            \text{diverges} & \abs{ x} \geq 1
        \end{cases}
    \]
\end{example}

\begin{thm}[Comparison test] \label{thm:comparison}
    Suppose there exist constants $M \in \N$ and $0 < C$ such that \[
        0 \leq a_{n} \leq C b_{n} \quad\forall\; n \geq M
    \] If $\sum b_{n}$ converges, then $\sum a_{n}$ converges. In other words, If $\sum a_{n}$ diverges, $\sum b_{n}$ diverges.
\end{thm}

\begin{example}
    Let $p \in \R$. Claim: \[
        \sum_{n=1}^{\infty} \frac{1}{n^{p}}
        \begin{cases}
            \text{converges} & p > 1 \\
            \text{diverges} & p \leq 1
        \end{cases}
    \]
\end{example}

\begin{thm}[Ratio test] \label{thm:ratio test}
    Let $\sum a_{n}$ be a series of positive terms. Suppose \[
        \lim_{n \to \infty} \frac{a_{n+1}}{a_{n}} = L \in \R
    \] Then,
    \begin{enumerate}[label=(\alph*)]
        \item If $L < 1$, the series converges.
        \item If $L > 1$, the series diverges.
        \item If $L = 1$, the test is inconclusive.
    \end{enumerate}
\end{thm}

\begin{thm} \label{thm:series limit laws}
    Suppose $\sum a_{n}$ and $\sum b_{n}$ converge with sums $a$ and $b$ respectively. Then, for constants $l$ and $m$, $\sum la_{n} + mb_{n}$ converges to $la + mb$. Suppose $\sum \abs{ a_{n}}$ and $\sum \abs{ b_{n}}$ converge. Then, so does $\sum \abs{ la_{n} + mb_{n} }$ for any choice of $l$ and $m$ in $\R$.
\end{thm}

\begin{cor} \label{cor:}
    Suppose $\sum a_{n}$ converges and $\sum b_{n}$ diverges. Let $m \in \R \setminus \set{0}$. Then, $\sum(a_{n} + b_{n})$ diverges, and $\sum mb_{n}$ diverges.
\end{cor}

\begin{defn} \label{defn:absolute convergence}
    A series $\sum a_{n}$ of real numbers is said to \emph{converge absolutely} if $\sum \abs{ a_{n}}$ converges. A series $\sum a_{n}$ of real numbers is said to \emph{converge conditionally} if $\sum \abs{ a_{n}}$ diverges but $\sum a_{n}$ converges.
\end{defn}

\begin{thm} \label{thm:}
    If $\sum a_{n}$ converges absolutely, it must converge. Moreover, $\abs{ \sum a_{n}} \leq \sum \abs{ a_{n}}$.
\end{thm}

\begin{example}
    $\sum \frac{(-1)^{n}}{n}$ is convergent.
\end{example}

\begin{thm}[Alternating Series Test/Leibniz Test] \label{thm:leibniz test}
    Suppose $\set{a_{n}}$ is a decreasing sequence of positive numbers going to 0. Then, $\sum (-1)^{n} a_{n}$ converges. Denoting the sum by $S$, we have that \[
        0 < (-1)^{n}(S - s_{n}) < a_{n+1}
    \]
\end{thm}

\begin{rem}
    The estimate is AST allows us to estimate sums of alternating series within any prescribed error. For instance, to know $\sum_{n=1}^{\infty} \frac{(-1)^{n}}{n}$ up to an error of $0.01$. I need to find $n$ so that \[
        \abs{ S - s_{n} } < \frac{1}{100}.
    \]
    Take $n = 99$, or the sum of the first $99$ terms.
\end{rem}

\section{Limits \& Continuity}

\subsection{Limit of a function}

\begin{example}
    \begin{enumerate}[label=(\alph*)]
        \item For $f(x) = c$, $c \in \R$, \[
            \lim_{x \to p} f(x) = c
        \] Choose $\delta = 1$. $0 < \abs{ x - p } < \delta \implies \abs{ f(x) - c } = 0 < \varepsilon \;\forall\; \varepsilon > 0$.

        \item For $f(x) = x$, \[
            \lim_{x \to p} f(x) = p
        \] Choose $\delta = \varepsilon$. $ 0 < \abs{ x - p } < \delta \implies \abs{ f(x) - p} < \varepsilon$.

        \item For $f(x) = \sqrt{x}$ and $p > 0$, \[
            \lim_{x \to p} f(x) = \sqrt{p}
        \]
        \begin{align*}
            \abs{ \sqrt{x} - \sqrt{p} } &< \varepsilon \\
            \iff \frac{\abs{ x - p }}{\abs{ \sqrt{x} + \sqrt{p} }} &< \varepsilon \\
        \end{align*}
        Take $\delta = \min \set{p, \sqrt{p} \varepsilon}$ (this is to make sure $f$ is defined for all points in $N_{\delta}(p) \setminus \set{p}$). Now $\abs{ x - p} < \delta \implies$
        \begin{align*}
            \frac{\abs{ x - p }}{\abs{ \sqrt{x} + \sqrt{p} }} &< \frac{\delta}{\abs{ \sqrt{x} + \sqrt{p} }} \\
            &= \frac{\sqrt{p} \varepsilon}{\abs{ \sqrt{x} + \sqrt{p} }} \\
            &< \varepsilon \qedhere
        \end{align*}

        \item For $f(x) = \frac{1}{x}, x \neq 0$, \[
            \lim_{x \to 0} f(x) \text{ does not exist}
        \]
        \begin{proof}
            Suppose $\exists\; L \in \R$ such that \[
                \lim_{x \to 0} f(x) = L
            \] Choose $\varepsilon = \frac{1}{L}$. Then $\exists\; \delta > 0$ such that $ 0 < \abs{ x - 0 } < \delta \implies \abs{ f(x) - L } < 1$. By the Archimedean property, $\exists\; N \in \N$ such that $\frac{1}{N} < \delta$.

            Now $0 < \frac{1}{N+2} < \frac{1}{N} < \delta$. Thus by our hypothesis, $\abs{ N + 2 - L } < 1$ and $\abs{ N - L } < 1 \implies \abs{ 2 } < \abs{ N + 2 - L } + \abs{ N - L } < 1 + 1 = 2$. Contradiction.
        \end{proof}
    \end{enumerate}
\end{example}

\begin{thm}[Limit laws for functions] \label{thm:func limit laws}
    Suppose $f$ and $g$ are functions such that \[
        \lim_{x \to p} f(x) = a, \qquad \lim_{x \to p} g(x) = b.
    \] Then,
    \begin{align}
        \lim_{x \to p} (f \pm g)(x) &= a \pm b \\
        \lim_{x \to p} (f \cdot g)(x) &= a \cdot b \\
        \lim_{x \to p} (f/g)(x) &= a/b
    \end{align}
\end{thm}

\subsection{Continuity}

\begin{defn} \label{defn:continuous}
    Let $S \subseteq \R$ be a (nonempty) subset, $f: S \to \R$ and $p \in S$. We say that $f$ is continuous at $p$ iff: \\
    for every $\varepsilon > 0, \;\exists \delta_{\varepsilon} > 0$ such that \[
        \abs{ x - p } < \delta_{\varepsilon} \;\land\; x \in S \implies \abs{ f(x) - f(p) } < \varepsilon
    \] We say that $f$ is continuous on $S$ iff $f$ is continuous at each $p \in S$.
\end{defn}

\begin{rem}
    It is possible that $\exists\; \delta$ such that $N_{\delta}(p) \cup S = \set{p}$. \textit{E.g.}, $S = \N, p = 0, \delta \leq 1$
\end{rem}

\begin{rem}
    If $f$ is defined on some interval $(a, b)$ containing $p$, then this definition is equivalent to \[
        \lim_{x \to p} f(x) = f(p)
    \] \textcolor{red!85!black}{How?} For any $\varepsilon > 0$, choose $\delta = \min \set{\delta_{\varepsilon}, b - p, p - a}$. Then $f$ is defined on all points in $N_{\delta}(p)$, and for all $x \in N_{\delta}(p)$, we have $f(x) \in N_{\varepsilon}(f(p))$. Thus \[
        \lim_{x \to p} f(x) = f(p)
    \]
\end{rem}

\begin{thm}[Algebraic laws for continuity] \label{thm:continuity laws}
    Suppose $f$ and $g$ are continuous at $p \in S$. Then so are $f \pm g, fg$ and if $g(p) \neq 0, f/g$.
\end{thm}

\begin{thm} \label{thm:compositions}
    Let $f: A \to \R$ and $g: B \to \R$ be continuous functions such that $f(A) = \mathop{range}(f) \subseteq B$. Then, \[
        g \circ f(x) = g(f(x)) : A \to \R
    \] is continuous.
\end{thm}

\begin{thm}[Intermediate Value Theorem] \label{thm:intermediate value}
    Let $f: [a, b] \to \R$ be a continuous function. Suppose $y \in \R$ is a number between $f(a)$ and $f(b)$, \textit{i.e.}, $y \in [f(a), f(b)]$. Then $\exists\; c \in [a, b]$ such that \[
        f(c) = y
    \]
\end{thm}

\begin{cor}[Bolzano] \label{cor:bolzano}
    Let $f : [a, b] \to \R$ be a continuous function such that $f(a)$ and $f(b)$ take opposite signs. Then $\exists\; c \in (a, b)$ such that $f(c) = 0$.
\end{cor}

\begin{rem}
    Bolzano's statement is equivalent to the IVT (let $g = f - y$).
\end{rem}

\begin{thm}[The Borsuk-Ulam Theorem : baby version] \label{thm:borsuk-ulam}
    Let $S$ be the set $\set{(x, y) \in \R \times \R : x^{2} + y^{2} = 1}$. Let $f: S \to \R$ be a continuous function. Then there exists a pair of antipodal points on the circle which have the same value of $f$.
\end{thm}

\begin{lem} \label{lem:comparison of seq}
    Let $a_{n}, b_{n}$ be convergent sequences such that $a_{n} \leq b_{n}$ for all $n$ (large enough). Then \[
        \lim_{n \to \infty} a_{n} \leq \lim_{n \to \infty} b_{n}
    \]
\end{lem}

\begin{defn} \label{defn:bounded fn}
    A function $f : S \to \R$ is said to be \emph{bounded above} on $S$ if $\exists\; U \in \R$ such that $f(x) \leq U \;\forall\; x \in S$.

    $f$ is said to be \emph{bounded} if $\exists\; M > 0$ such that $\abs{ f(x) } < M \;\forall\; x \in S$.
\end{defn}

\begin{thm}[Continuous functions on closed, bounded intervals are bounded.] \label{thm:closed bounded}
    Let $f : [a, b] \to \R$ be continuous on $[a, b]$. Then $f$ is a bounded function.
\end{thm}

\begin{defn} \label{defn:global extrema}
    A function $f : S \to \R$ is said to have a \emph{global maximum} on $S$ at a point $p \in S$ if $f(x) \leq f(p) \;\forall\; x \in S$.
\end{defn}

\begin{thm}[Extreme value theorem] \label{thm:extreme value}
    Let $f : [a, b] \to \R$ be a continuous function. Then $f$ attains both a global maximum and a global minimum in $[a, b]$.
\end{thm}

\begin{cor} \label{cor:continuous fn range}
    Let $f : [a, b] \to \R$ be continuous. Then (using IVT), \[
        f([a, b]) = [\min_{[a, b]} f, \max_{[a, b]} f]
    \]
\end{cor}

\section{Differentiation}

\begin{defn} \label{defn:dv}
    Let $f : (a, b) \to \R$ be a function and $p \in (a, b)$. We say that $f$ is differentiable in $(a, b)$ if \[
        \lim_{h \to 0} \frac{f(p + h) - f(p)}{h} = \lim_{x \to p} \frac{f(x) - f(p)}{x - p}
    \] exists, and the limit is called the derivative of $f$ at $p$, denoted $f'(p)$.

    If $f$ is differentiable on each $p$ in $(a, b)$, it is said to be differentiable on $(a, b)$ and $f' : (a, b) \to \R$ is called the derivative of $f$ on $(a, b)$.
\end{defn}

\begin{thm}[Differentiability $\implies$ continuity] \label{thm:diff=>cont}
    Let $f : (a, b) \to \R$ be differentiable at $p \in (a, b)$. Then $f$ is continuous at $p$.
\end{thm}

\begin{thm} \label{thm:dv algebra}
    
\end{thm}

\end{document}





