\documentclass[12pt]{article}
\usepackage[utf8]{inputenc}
\usepackage[a4paper, total={6in, 9in}]{geometry}
\usepackage{amsmath}
\usepackage{amssymb}
\usepackage{amsthm}
\usepackage{enumitem}
\usepackage{outlines}

\usepackage{paracol}

\usepackage[dvipsnames]{xcolor}
\colorlet{exercise}{red!80!black}
\colorlet{solved}{green!30!black}
\colorlet{self_proof}{blue!30!black}

\usepackage{mathrsfs}
\usepackage{parskip}

\usepackage{accents}
\newcommand{\ubar}[1]{\underaccent{\bar}{#1}}

\usepackage{hyperref}
\usepackage{cleveref}

\providecommand{\dd}{\,\mathrm{d}}

\usepackage{mathtools} % also contains \coloneqq
\DeclarePairedDelimiter{\oldparen}{(}{)}
\DeclarePairedDelimiter{\oldset}{\{}{\}}
\DeclarePairedDelimiter{\oldabs}{\lvert}{\rvert}
\DeclarePairedDelimiter{\oldnorm}{\lVert}{\rVert}
\DeclarePairedDelimiter{\oldfloor}{\lfloor}{\rfloor}
\DeclarePairedDelimiter{\oldceil}{\lceil}{\rceil}

\makeatletter
\def\paren{\@ifstar{\oldparen}{\oldparen*}}
\def\set{\@ifstar{\oldset}{\oldset*}}
\def\abs{\@ifstar{\oldabs}{\oldabs*}}
\def\norm{\@ifstar{\oldnorm}{\oldnorm*}}
\def\floor{\@ifstar{\oldfloor}{\oldfloor*}}
\def\ceil{\@ifstar{\oldceil}{\oldceil*}}

\g@addto@macro\normalsize{%
  \setlength\abovedisplayskip{7pt}%
  \setlength\belowdisplayskip{7pt}%
  \setlength\abovedisplayshortskip{1pt}%
  \setlength\belowdisplayshortskip{1pt}%
}
\makeatother

\newcommand\N{\ensuremath{\mathbb{N}}}
\newcommand\R{\ensuremath{\mathbb{R}}}
\newcommand\Z{\ensuremath{\mathbb{Z}}}
\renewcommand\O{\ensuremath{\varnothing}}
\renewcommand\P{\ensuremath{\mathbb{P}}}
\newcommand\Q{\ensuremath{\mathbb{Q}}}
\newcommand\C{\ensuremath{\mathbb{C}}}

% make implied by and implies arrows shorter
\let\implies\Rightarrow
\let\impliedby\Leftarrow
\let\iff\Leftrightarrow

% make \epsilon and \varepsilon the same
\let\epsilon\varepsilon

% <theorems>
% \theoremstyle{plain}
% \newtheorem{thm}{Theorem}[section]
% \newtheorem*{thm*}{Theorem}
% \newtheorem{prop}[thm]{Proposition}
% \newtheorem{lem}[thm]{Lemma}
% \newtheorem{cor}[thm]{Corollary}
% \newtheorem{axiom}{Axiom}[section]
% \newtheorem*{exer}{Exercise}
% 
% \theoremstyle{definition}
% \newtheorem{defn}[thm]{Definition}
% 
% \theoremstyle{remark}
% \newtheorem*{rem}{Remark} 
% \newtheorem*{example}{Example}
% </theorems>

% Gilles Castel's theorems
\usepackage[framemethod=Tikz]{mdframed}
\mdfsetup{skipabove=1em,skipbelow=0em}
\mdfdefinestyle{axiomstyle}{
    outerlinewidth = 1.5,
    roundcorner = 10,
    leftmargin = 15,
    rightmargin = 15,
    backgroundcolor = blue!10
}
\mdfdefinestyle{defnstyle}{
    outerlinewidth = 1.5,
    % roundcorner = 10,
    leftmargin = 7,
    rightmargin = 7,
    backgroundcolor = green!10
}
\mdfdefinestyle{thmstyle}{
    outerlinewidth = 1.5,
    roundcorner = 10,
    leftmargin = 7,
    rightmargin = 7,
    backgroundcolor = yellow!10
}
\mdfdefinestyle{lemmastyle}{
    outerlinewidth = 1.5,
    roundcorner = 10,
    leftmargin = 7,
    rightmargin = 7,
    backgroundcolor = yellow!10
}
\theoremstyle{definition}
\newmdtheoremenv[nobreak=true, style=axiomstyle]{axiom}{Axiom}[section]
\newmdtheoremenv[nobreak=true, style=thmstyle]{thm}{Theorem}[section]
\newmdtheoremenv[nobreak=true]{prop}[thm]{Proposition}
\newmdtheoremenv[nobreak=true]{lem}[thm]{Lemma}
\newmdtheoremenv[nobreak=true]{cor}[thm]{Corollary}
\newmdtheoremenv[nobreak=true, style=defnstyle]{defn}[thm]{Definition}

\newcounter{assignment}
\newmdtheoremenv[nobreak=true]{problem}{Problem}[assignment]

\theoremstyle{remark}
\newtheorem*{rem}{Remarks}
\newtheorem*{example}{Example}
\newtheorem*{examplex}{Examples}

\newenvironment{examples}
{\begin{examplex}\leavevmode\begin{itemize}}{\end{itemize}\end{examplex}}
% <cref>
\crefname{thm}{theorem}{theorems}
\crefname{prop}{proposition}{propositions}
\crefname{lem}{lemma}{lemmas}
\crefname{cor}{corollary}{corollaries}
\crefname{axiom}{axiom}{axioms}
\crefname{defn}{definition}{definitions}
\crefname{problem}{problem}{problems}
% </cref>

% <hyperlinks>
\hypersetup{colorlinks,
    linkcolor={red!50!black},
    citecolor={blue!50!black},
    urlcolor={blue!80!black}}
% </hyperlinks>

\usepackage{xifthen}
\def\testdateparts#1{\dateparts#1\relax}
\def\dateparts#1 #2 #3 #4 #5\relax{
    \marginpar{\small\textsf{\mbox{#1 #2 #3 #5}}}
}

\def\@lecture{}%
\newcommand*{\lecture}[3]{
    \ifthenelse{\isempty{#3}}{%
        \def\@lecture{Lecture #1}%
    }{%
        \def\@lecture{Lecture #1: #3}%
    }%
    \subsection*{\@lecture}
    \marginpar{\raggedright\small\textsf{#2}}
    \vskip 6pt
}
\newcommand*{\assignment}[2]{%
    \stepcounter{assignment}%
    \subsection{Assignment #1}
    \marginpar{\raggedright\small{due #2}}
    \vskip 6pt
}
\makeatletter
\newcommand*{\refifdef}[3]{%label,command,fallback
    \@ifundefined{r@#1}{#3}{#2{#1}}%
}
\makeatother

\DeclareMathOperator{\dom}{dom}
\DeclareMathOperator{\ran}{ran}
\DeclareMathOperator{\spann}{span}
\DeclareMathOperator{\dimn}{dim}

\newcommand{\psum}[2]{\ensuremath{\mathrm{sum}_{#1}({#2})}}
\newcommand{\pprod}[2]{\ensuremath{\mathrm{prod}_{#1}({#2})}}


\title{Assignment 02}
\author{Naman Mishra}
\date{31 October 2022}

\begin{document}
\maketitle

\begin{problem} \leavevmode
    \begin{enumerate}[label=(\alph*)]
        \item Prove that for any $m, n \in \N$, exactly one of the following statements holds.
        \begin{enumerate}[label=(\textit{\roman*})]
            \item $m = n$;
            \item there is a $k \in \N \setminus \set{0}$ such that $m + k = n$;
            \item there is a $k \in \N \setminus \set{0}$ such that $n + k = m$.
        \end{enumerate}
        You may use: induction, the definition of $\mathrm{sum}_{m}$ any of its six properties stated in class (as Theorem 1.12), and the fact that the range of the function $f(x) = x + 1$ on $\N$ is $\N \setminus \set{0}$ (Problem 1 in HW1).
        \item Show that $\N$ is an ordered set if we define $<$ as follows: $m < n$ if there is a $k \in \N \setminus \set{0}$ such that $m + k = n$.  
    \end{enumerate}
\end{problem}

\begin{proof}
    Unless otherwise stated, any lowercase variable denotes a natural number.
    \begin{enumerate}[label=(\alph*)]
        \item Let $R \subseteq \N \times \N$ be a relation such that $a\mathrel{R}b \iff \exists\; k \neq 0$ such that $a + k = b$. Let \[
            B = \set{m \in \N : m = n, m\mathrel{R}n, \text{ or } n\mathrel{R}m}
        \]
        \textbf{Note:} If $\exists\; k$ such that $m + k = n$, then $m \in B$ as $k = 0$ gives $m = n$ and $k \neq 0$ gives $m \mathrel{R} n$. Similarly $n + k = m$ also implies $m \in B$. \\
        $0 \in B$ as $0 + n = n$. \\
        If $b \in B$, then:
        \begin{enumerate}
            \item[$(b = n)$] $S(b) = S(n) = n + 1 \implies S(b) \in B$.
            \item[$(b \mathrel{R} n)$] $\exists\; k \neq 0$ such that $b + k = n$. Since $k \in \operatorname{ran}(S)$ (HW 1.1), $\exists\; k'$ such that $S(k') = k$. Thus $b + S(k') = n \implies S(b) + k' = n \implies S(b) \in B$.
            \item[$(n \mathrel{R} b)$] $\exists\; k \neq 0$ such that $n + k = b$. Then $S(n + k) = S(b) \implies n + S(k) = S(b) \implies S(b) \in B$.
        \end{enumerate}
        Thus $b \in B \implies S(b) \in B \implies B = \N$.
        Since $n$ was arbitrary, one of the three statements holds for each $m, n$.

        Suppose $m = n$.
        Then if $m + k = n$, then $m + k = m + 0 \implies k = 0$ by the cancellation law.
        Similarly $n + k = m$ also implies $k = 0$.
        Thus $m = n$ cannot hold simultaneously with $m \mathrel{R} n$ or $n \mathrel{R} m$.
        Now if $m + k = n$ and $n + k' = m$, then $(n + k') + k = n \implies n + (k + k') = n + 0 \implies k + k' = 0 \implies k = k' = 0$.
        Thus $m \mathrel{R} n$ and $n \mathrel{R} n$ cannot hold simultaneously.

        Therefore exactly one of the three statements holds for all $m, n \in \N$.

        \item If we define $m < n$ as $m \mathrel{R} n$ above, from part (a) it is clear that exactly one of $m = n, m < n, n < m$ holds for all $m, n \in \N$.
        Moreover, if $a < b$ and $b < c$, then there exist natural numbers $k, k' \neq 0$ such that $a + k = b$ and $b + k' = c$.
        This implies $(a + k) + k' = a + (k + k') = c$.
        Since $x + y = 0 \implies x = y = 0$, $x \neq 0$ or $y \neq 0 \implies x + y \neq 0$.
        Thus $k + k' \neq 0 \implies a < c$.

        We have shown that $<$ obeys trichotomy and is transitive. Thus $(\N, <)$ is an ordered set. \qedhere
    \end{enumerate}
\end{proof}

\begin{problem}
    Let $(F, +, \cdot)$ be a field.
    According to axiom (F5), given $x \in F$, there is a $y \in F$ such that $x + y = 0$.
    Show that $y$ is unique, i.e., if there is a $z \in F$ such that if $x + y = x + z = 0$, then $y = z$.
    Use only the field axioms to justify your answer.
\end{problem}
\begin{proof}
    \begin{align*}
        x + y &= x + z \\
        (y + x) + y &= (y + x) + z \\
        y &= z \qedhere
    \end{align*}
\end{proof}

\begin{problem}
    Let $+$ and $\cdot$ be the usual addition and multiplication on $\N$.
    You are free to use their well-known properties.
    \begin{enumerate}[label=(\alph*)]
        \item Let $F = \set{0, 1, 2, 3}$. We endow $F$ with addition and multiplication as follows:
        \begin{align*}
            & a \oplus b = c, && \text{where $c$ is the remainder that $a + b$ leaves when divided by 4} \\
            & a \odot b = c, && \text{where $c$ is the remainder that $a \cdot b$ leaves when divided by 4}
        \end{align*}
        Is $(F, \oplus, \odot)$ a field? Please justify your answer.

        \item Let $F = \set{0, 1}$. We endow $F$ with addition and multiplication as follows:
        \begin{align*}
            & a \oplus b = c, && \text{where $c$ is the remainder that $a + b$ leaves when divided by 2} \\
            & a \odot b = c, && \text{where $c$ is the remainder that $a \cdot b$ leaves when divided by 2}
        \end{align*}
        You may assume that $(F, \oplus, \odot)$ is a field. Is it possible to give $F$ a relation $<$ so that $(F, \oplus, \odot, <)$ is an ordered field? Please justify your answer.
    \end{enumerate}
\end{problem}
\begin{proof}
    \begin{enumerate}[label=(\alph*)]
        \item Clearly 1 is the multiplicative identity.
            \begin{align*}
                2 \cdot 0 &= 0 & 2 \cdot 1 &= 2 & 2 \cdot 2 &= 4 & 2 \cdot 3 &= 6 \\
                2 \odot 0 &= 0 & 2 \odot 1 &= 2 & 2 \odot 2 &= 0 & 2 \odot 3 &= 2
            \end{align*}
            Thus there is no multiplicative inverse of 2 in $F$. So $(F, \oplus, \odot)$ is not a field.
        \item
            If $(F, \oplus, \odot, <)$ is an ordered field and $0 < 1$, then by the field axioms, $0 \oplus 1 < 1 \oplus 1 \iff 1 < 0$ which is a contradiction as it disobeys trichotomy of order. If $1 < 0$ then $1 \oplus 1 < 0 \oplus 1 \iff 0 < 1$, which cannot be true. \qedhere
    \end{enumerate}
\end{proof}

\begin{problem}
    Let $(F, +, \cdot, <)$ be an ordered field.
    \begin{enumerate}[label=(\textit{\roman*})]
        \item Using only the field axioms, and the uniqueness of the additive inverse, show that for all $a, b, c \in F, a(b - c) = ab - ac$.
        \item Using the field axioms, the order axioms, and Part (i), show that for all $a, b, c \in F$, if $a < b$ and $c < 0$, then $bc < ac$.
    \end{enumerate}
\end{problem}
\begin{proof}
    \begin{enumerate}[label=(\textit{\roman*})]
        \item $a(b + (-c)) = ab + a(-c)$
        \begin{align*}
            a(c + (-c)) &= ac + a(-c) \\
            0 &= ac + a(-c) \\
            a(-c) &= -(ac)
        \end{align*}
        Thus $a(b + (-c)) = ab - ac$.

        \item $c < 0 \implies c + (-c) < -c \implies 0 < -c$. \\
        $a < b \implies a + (-a) < b + (-a) \implies 0 < b - a$. \\
        Thus
        \begin{align*}
             0 &< (b + (-a))(-c) \tag{O4} \\
             0 &< b(-c) + (-a)(-c) \\
             0 &< -bc + ac \\
            bc &< ac \qedhere
        \end{align*}
    \end{enumerate}
\end{proof}

\begin{problem}
    Apostol defines an ordered field as a field $(F, +, \cdot)$ together with a set $P \subseteq F$ satisfying the following axioms.
    \begin{enumerate}[label=(O'\arabic*)]
        \item If $x, y \in P$, then $x + y \in P$ and $x \cdot y \in P$.
        \item For every $x \in F$ such that $x \neq 0$, $x \in P$ or $-x \in P$, but not both.
        \item $0 \notin P$
    \end{enumerate}
    Show that our definition of an ordered field is equivalent to that of Apostol’s. That is, show that for a field$(F, +, \cdot)$:
    \begin{enumerate}[label=(\textit{\roman*})]
        \item If there is a relation $<$ satisfying (O1)-(O4), then there is a $P \subseteq F$ satisfying (O'1)-(O'3), and
        \item if there is a $P \subseteq F$ satisfying (O’1)-(O’3), then there is a relation $<$ satisfying (O1)-(O4).
    \end{enumerate}
\end{problem}
\begin{proof}
    Suppose there is a relation $<$ on $(F, +, \cdot)$ satisfying (O1)-(O4). Define \[
        P = \set{x \in F : 0 < x}.
    \]
    Suppose $x, y \in P \iff 0 < x, y$.
    Then  $-x < x + (-x) \implies -x < 0 < y \implies -x < y \implies 0 < x + y \implies x + y \in P$ by (O2) and (O3).

    If $x, y \in P$, then by (O4), $x \cdot y \in P$.
    Thus (O'1) holds.

    If $0 < x$, $x \in P$.
    If $x < 0$, then by (O3) $x + (-x) < -x \implies 0 < -x$, \emph{i.e.}, $-x \in P$. Thus (O'2) is holds.

    $0 \not< 0$, so (O'3) holds.

    Now suppose there is a subset $P \subseteq F$ which satisfies (O'1)-(O'3).
    Define relation $<$ on $F$ as $a < b \iff b - a \in P $.

    Note that $-(b - a) = a - b$.
    \begin{enumerate}[label=(O\arabic*)]
        \item For any $a, b \in F$, exactly one of $b - a = 0$, $b - a \in P$, and $-(b - a) \in P$ holds (by (O'2) and (O'3), as $-0 = 0$).
        $b - a = 0 \iff a = b, b - a \in P \iff a < b,$ and $-(b - a) \in P \iff a - b \in P \iff b < a$.
        Thus exactly one of $a = b, a < b,$ and $b < a$ holds.
        \item If $a < b$ and $b < c$, then $b - a \in P$ and $c - b \in P$.
        So by (O'1), $c - b + b - a \in P \iff c - a \in P \iff a < c$.
        \item If $a < b$ and $c \in F$, then $(b + c) - (a + c) = b + c + (-a) + (-c) = b - a \in P \implies a + c < b + c$.
        \item $0 < a \iff a - 0 \in P \iff a \in P$.
        So $0 < a$ and $0 < b$ implies $0 < a \cdot b$ by (O'1). \qedhere
    \end{enumerate}
\end{proof}

\end{document}
