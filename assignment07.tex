\documentclass[12pt]{article}
\usepackage[utf8]{inputenc}
\usepackage[a4paper, total={6in, 9in}]{geometry}
\usepackage{amsmath}
\usepackage{amssymb}
\usepackage{amsthm}
\usepackage{enumitem}
\usepackage{outlines}

\usepackage{paracol}

\usepackage[dvipsnames]{xcolor}
\colorlet{exercise}{red!80!black}
\colorlet{solved}{green!30!black}
\colorlet{self_proof}{blue!30!black}

\usepackage{mathrsfs}
\usepackage{parskip}

\usepackage{accents}
\newcommand{\ubar}[1]{\underaccent{\bar}{#1}}

\usepackage{hyperref}
\usepackage{cleveref}

\providecommand{\dd}{\,\mathrm{d}}

\usepackage{mathtools} % also contains \coloneqq
\DeclarePairedDelimiter{\oldparen}{(}{)}
\DeclarePairedDelimiter{\oldset}{\{}{\}}
\DeclarePairedDelimiter{\oldabs}{\lvert}{\rvert}
\DeclarePairedDelimiter{\oldnorm}{\lVert}{\rVert}
\DeclarePairedDelimiter{\oldfloor}{\lfloor}{\rfloor}
\DeclarePairedDelimiter{\oldceil}{\lceil}{\rceil}

\makeatletter
\def\paren{\@ifstar{\oldparen}{\oldparen*}}
\def\set{\@ifstar{\oldset}{\oldset*}}
\def\abs{\@ifstar{\oldabs}{\oldabs*}}
\def\norm{\@ifstar{\oldnorm}{\oldnorm*}}
\def\floor{\@ifstar{\oldfloor}{\oldfloor*}}
\def\ceil{\@ifstar{\oldceil}{\oldceil*}}

\g@addto@macro\normalsize{%
  \setlength\abovedisplayskip{7pt}%
  \setlength\belowdisplayskip{7pt}%
  \setlength\abovedisplayshortskip{1pt}%
  \setlength\belowdisplayshortskip{1pt}%
}
\makeatother

\newcommand\N{\ensuremath{\mathbb{N}}}
\newcommand\R{\ensuremath{\mathbb{R}}}
\newcommand\Z{\ensuremath{\mathbb{Z}}}
\renewcommand\O{\ensuremath{\varnothing}}
\renewcommand\P{\ensuremath{\mathbb{P}}}
\newcommand\Q{\ensuremath{\mathbb{Q}}}
\newcommand\C{\ensuremath{\mathbb{C}}}

% make implied by and implies arrows shorter
\let\implies\Rightarrow
\let\impliedby\Leftarrow
\let\iff\Leftrightarrow

% make \epsilon and \varepsilon the same
\let\epsilon\varepsilon

% <theorems>
% \theoremstyle{plain}
% \newtheorem{thm}{Theorem}[section]
% \newtheorem*{thm*}{Theorem}
% \newtheorem{prop}[thm]{Proposition}
% \newtheorem{lem}[thm]{Lemma}
% \newtheorem{cor}[thm]{Corollary}
% \newtheorem{axiom}{Axiom}[section]
% \newtheorem*{exer}{Exercise}
% 
% \theoremstyle{definition}
% \newtheorem{defn}[thm]{Definition}
% 
% \theoremstyle{remark}
% \newtheorem*{rem}{Remark} 
% \newtheorem*{example}{Example}
% </theorems>

% Gilles Castel's theorems
\usepackage[framemethod=Tikz]{mdframed}
\mdfsetup{skipabove=1em,skipbelow=0em}
\mdfdefinestyle{axiomstyle}{
    outerlinewidth = 1.5,
    roundcorner = 10,
    leftmargin = 15,
    rightmargin = 15,
    backgroundcolor = blue!10
}
\mdfdefinestyle{defnstyle}{
    outerlinewidth = 1.5,
    % roundcorner = 10,
    leftmargin = 7,
    rightmargin = 7,
    backgroundcolor = green!10
}
\mdfdefinestyle{thmstyle}{
    outerlinewidth = 1.5,
    roundcorner = 10,
    leftmargin = 7,
    rightmargin = 7,
    backgroundcolor = yellow!10
}
\mdfdefinestyle{lemmastyle}{
    outerlinewidth = 1.5,
    roundcorner = 10,
    leftmargin = 7,
    rightmargin = 7,
    backgroundcolor = yellow!10
}
\theoremstyle{definition}
\newmdtheoremenv[nobreak=true, style=axiomstyle]{axiom}{Axiom}[section]
\newmdtheoremenv[nobreak=true, style=thmstyle]{thm}{Theorem}[section]
\newmdtheoremenv[nobreak=true]{prop}[thm]{Proposition}
\newmdtheoremenv[nobreak=true]{lem}[thm]{Lemma}
\newmdtheoremenv[nobreak=true]{cor}[thm]{Corollary}
\newmdtheoremenv[nobreak=true, style=defnstyle]{defn}[thm]{Definition}

\newcounter{assignment}
\newmdtheoremenv[nobreak=true]{problem}{Problem}[assignment]

\theoremstyle{remark}
\newtheorem*{rem}{Remarks}
\newtheorem*{example}{Example}
\newtheorem*{examplex}{Examples}

\newenvironment{examples}
{\begin{examplex}\leavevmode\begin{itemize}}{\end{itemize}\end{examplex}}
% <cref>
\crefname{thm}{theorem}{theorems}
\crefname{prop}{proposition}{propositions}
\crefname{lem}{lemma}{lemmas}
\crefname{cor}{corollary}{corollaries}
\crefname{axiom}{axiom}{axioms}
\crefname{defn}{definition}{definitions}
\crefname{problem}{problem}{problems}
% </cref>

% <hyperlinks>
\hypersetup{colorlinks,
    linkcolor={red!50!black},
    citecolor={blue!50!black},
    urlcolor={blue!80!black}}
% </hyperlinks>

\usepackage{xifthen}
\def\testdateparts#1{\dateparts#1\relax}
\def\dateparts#1 #2 #3 #4 #5\relax{
    \marginpar{\small\textsf{\mbox{#1 #2 #3 #5}}}
}

\def\@lecture{}%
\newcommand*{\lecture}[3]{
    \ifthenelse{\isempty{#3}}{%
        \def\@lecture{Lecture #1}%
    }{%
        \def\@lecture{Lecture #1: #3}%
    }%
    \subsection*{\@lecture}
    \marginpar{\raggedright\small\textsf{#2}}
    \vskip 6pt
}
\newcommand*{\assignment}[2]{%
    \stepcounter{assignment}%
    \subsection{Assignment #1}
    \marginpar{\raggedright\small{due #2}}
    \vskip 6pt
}
\makeatletter
\newcommand*{\refifdef}[3]{%label,command,fallback
    \@ifundefined{r@#1}{#3}{#2{#1}}%
}
\makeatother

\DeclareMathOperator{\dom}{dom}
\DeclareMathOperator{\ran}{ran}
\DeclareMathOperator{\spann}{span}
\DeclareMathOperator{\dimn}{dim}

\newcommand{\psum}[2]{\ensuremath{\mathrm{sum}_{#1}({#2})}}
\newcommand{\pprod}[2]{\ensuremath{\mathrm{prod}_{#1}({#2})}}


\title{Assignment 07}
\author{Naman Mishra}
\date{15 Dec 2022}

\begin{document}
\maketitle

% Problem 1
\begin{problem}
    Let $f : (a, b) \to \R$ be a function and let $c \in (a, b)$.
    Which of the following statements are true, and which are false?
    \begin{enumerate}[label=(\alph*)]
        \item If $\lim\limits_{h \to 0} \dfrac{f(c) - f(c - h)}{h}$ exists then $f$ is differentiable at $c$ and $f'(c) = \lim\limits_{h \to 0} \dfrac{f(c) - f(c - h)}{h}$.
        \item If $\lim\limits_{\substack{n \to \infty \\ n \in \N}} n (f( c + \frac{1}{n}) - f(c))$ exists, then $f$ is differentiable at $c$.
    \end{enumerate}
\end{problem}
(a) is just $h$ replaced with $-h$.
Since $h$ could take any real values, including negative ones, it doesn't make any difference to negate it.
Since we are replacing $h$ with $-h$, what we have is a composition.
The usual difference quotient ($\Delta y / \Delta x$) is composed with $f(x) = -x$.
We have two properties which let us compute the limit of $g \circ f$ given the limits of $g$ and $f$.

If $\lim_{x \to a} f(x) = L$ and $\lim_{y \to L} g(y) = M$, then $\lim_{x \to a} g(f(x)) = M$ if:
\begin{enumerate}[label=(\roman*)]
    \item $g$ is continuous.
    \item $f$ does not equal $L$ anywhere `around' $a$ (in a sufficiently small neighbourhood).
\end{enumerate}
Here, $f(x) = -x$, which clearly does not equal 0 anywhere except at 0. We will use the second condition to conclude that the limit exists (and is equal to the derivative).

In part (b), however, the points that we are checking are limited in two ways: only very few fixed points around $c$, as well as only points to the right.
We can construct a counterexample exploiting either of these limitations.
\begin{proof}
    \begin{enumerate}[label=(\alph*)]
        \item True. Let $g(x) = -x$ and $G(x) = \dfrac{f(c) - f(c - x)}{x} \;\forall\; x \in (a - c, 0) \cup (0, b - c)$. We have
        \begin{align*}
            \lim_{x \to 0} g(x) &= 0 & & & \lim_{x \to 0} G(x) &:= L \\
            && g(x) \neq 0 \;\forall\; x \neq 0
        \end{align*}
        Thus by limit of compositions, we have
        \begin{align*}
            L &= \lim_{x \to 0} G(g(x)) \\
            &= \lim_{x \to 0} \frac{f(c) - f(c - (-x))}{-x} \\
            &= \lim_{x \to 0} \frac{f(c + x) - f(c)}{x} \\
            &= f'(c)
        \end{align*}
        Thus $f'(c)$ exists and is equal to $L$.
    
        \item False. Let $f: (-1, 1) \to \R$ such that \[
            f(x) = 
            \begin{cases}
                0 & x \geq 0 \\
                1 & x < 0
            \end{cases}
        \] Then the given limit exists for $c = 0$ as $n \paren{f\paren{\frac{1}{n}} - f(0)} = n - n = 0 \;\forall\; n \in \N$.
        However, $f$ is not differentiable at 0 as it is not continuous at 0.
    
        Alternatively, consider $f: (-1, 1) \to \R$ such that \[
            f(x) = 
            \begin{cases}
                0 & x \in \Q \\
                1 & x \in \R \setminus \Q
            \end{cases}
        \]
    \end{enumerate}
\end{proof}

% Problem 2
\begin{problem}
    Let $f : \R \to \R$ be a differentiable function.
    In each of the following cases, argue that $g$ is differentiable on its domain (you may use any theorems stated in class),
    and determine the derivative of $g$ in terms of $f'$.
    \begin{enumerate}[label=(\alph*)]
        \item $g(x) = f(x^{3}) + \sin(f(x))$.
        \item $g(x) = (f \circ f)(x)$.
    \end{enumerate}
\end{problem}
\begin{proof}
    Since $\sin, x^{3}$ and $f$ are differentiable, we have by the chain rule and addition rule:
    \begin{enumerate}[label=(\alph*)]
        \item 
        \begin{align*}
            g'(x) &= f'(x^{3}) (x^{3})' + \sin'(f(x)) f'(x) \\
            &= 3x^{2} f'(x^{3}) + \cos(f(x)) f'(x)
        \end{align*}
    
        \item 
        \begin{align*}
            g'(x) = f'(f(x)) f'(x)
        \end{align*}
    \end{enumerate}
\end{proof}

% Problem 3
\begin{problem}
    Show that $f(x) = x^{1/3}, x \in \R$, is not differentiable at $x = 0$.
\end{problem}
\[
    \frac{f(0 + h) - f(0)}{h} = h^{-2/3} \;\forall\; h \neq 0
\]
We need to prove that this does not have a limit at 0.
This is simple Archimedean, OR direct $\varepsilon-\delta$ contradiction similar to $(-1)^{n}$.
I have given the Archimedean proof, you try $\varepsilon-\delta$.
(Hint: Let $\varepsilon = 1$.
Consider $x = \pm\min \set{\dfrac{\delta}{2}, 1}$.)

\begin{proof}
    Consider the sequence $\set{\frac{1}{n}}_{n \in \P}$.
    For any $R \in \R$, there exists $N \in \N$ such that $N \cdot 1 > R^{3/2} \implies N^{2/3} > R \implies \paren{\frac{1}{n}}^{-2/3} > R \;\forall\; n \geq N$.
    Thus, $\set{\frac{1}{n}}$ converges to 0 as $n \to \infty$ but $\set{\paren{\frac{1}{n}}^{-2/3}}$ is unbounded $\implies$ divergent.
    Thus limit of $h^{-2/3}$ does not exist at 0.
\end{proof}

% Problem 4
\begin{problem}
    Suppose $f : \R \to \R$ is differentiable at $c$ and $f(c) = 0$.
    Show that $g(x) = \abs{f(x)}$ is differentiable at $c$ if and only if $f'(c) = 0$.
\end{problem}
We want two things:
\begin{enumerate}[label=(\roman*)]
    \item $f'(c) = 0 \implies g$ is differentiable at $c$.
    \item $g$ is differentiable at $c \implies f'(c) = 0$.
\end{enumerate}
Let us think about modulus.
If $\lim_{x \to a} f(x)$ is known, can we say anything about $\lim_{x \to a} \abs{f(x)}$?
The answer is yes, we have $\lim_{x \to a} \abs{f(x)} = \abs{\lim_{x \to a} f(x)}$.
Here is a short proof.
\begin{proof}
    $\lim_{x \to a} f(x) = L \implies \;\forall\; \varepsilon > 0 \;\exists\; \delta > 0$ such that $\abs{f(x) - L} < \varepsilon$. But by the triangle inequality, $\abs{\abs{f(x)} - \abs{L}} < \abs{f(x) - L} < \varepsilon$. Thus the limit of $\abs{f(x)} = \abs{L}$ as $x \to a$.
\end{proof}

\emph{However}, can we determine $\lim_{x \to a} f(x)$ from $\lim_{x \to a} \abs{f(x)}$? Let's attempt a similar proof.
\begin{proof}[Attempt]
    $\lim_{x \to a} \abs{f(x)} = \abs{L} \implies \;\forall\; \varepsilon > 0 \;\exists\; \delta > 0$ such that $\abs{\abs{f(x)} - \abs{L}} < \varepsilon$.
    But by the triangle inequality, $\abs{f(x) - L} \;\textcolor{red}{>}\; \abs{\abs{f(x)} - \abs{L}} \;\textcolor{red}{<}\; \varepsilon$.
    The signs of inequality are not helpful to us anymore.
\end{proof}
Your first thought might be that $f$ can have limit either $L$ or $-L$, but it may not have a limit at all.
As an example, consider signum, with limit of $\abs{\mathrm{signum}} = 1$ at 0.

The \emph{essence} of why this is happening is that modulus is not an invertible function.
We lose \emph{information} on passing a number through modulus.
There is only one special case where we do NOT lose information.
What is it? \[
    \vdots
\] $\abs{x} = 0 \implies x = 0$ with full certainty!
This is why the given statement is true, and this is how we will prove it. 
We will use the fact that $\lim_{x \to a} f(x) = 0 \iff \lim_{x \to a} \abs{f(x)} = 0$, which we will prove right now.
\begin{proof}
    \begin{align*}
        \lim_{x \to a} f(x) &= 0 \\
        \iff &\;\forall\; \varepsilon > 0 \;\exists\; \delta > 0 \text{ such that} \abs{x - a} < \delta \implies \abs{f(x) - 0} < \varepsilon \\
        \iff &\;\forall\; \varepsilon > 0 \;\exists\; \delta > 0 \text{ such that} \abs{x - a} < \delta \implies \abs{\abs{f(x)} - 0} < \varepsilon \\
        &\iff \lim_{x \to a} \abs{f(x)} = 0. \qedhere
    \end{align*}
\end{proof}

Note that the difference quotients of $f$ and $g$ have the same absolute value. Thus if the limit of either is zero, the limit of the other must also be zero.
Thus,
\begin{enumerate}[label=(\roman*)]
    \item $f'(c) = 0 \implies$ \textcolor{exercise}{$g'(c) = 0$}.
    \item \textcolor{exercise}{$g'(c) = 0$} $\implies f'(c) = 0$.
\end{enumerate}
Can you argue why $g'(c) = 0 \iff g$ is differentiable at $c$ to complete the proof?

% Problem 5
\begin{problem}
    
\end{problem}
We know $f(x) = x^{n} \implies f'(x) = n x^{n-1}$.

Let $g = f^{-1} \implies g(x) = x^{\frac{1}{n}}$.

Then $f(p) = q \implies g'(q) = \frac{1}{f'(p)}$. \[
    g'(q) = \frac{1}{f'(q^{\frac{1}{n}})} = \frac{1}{n q^{1 - \frac{1}{n}}} = \frac{1}{n} q^{\frac{1}{n} - 1}.
\] Now let $F(x) = x^{\frac{1}{n}} - 1 - (x - 1)^{\frac{1}{n}}, x \in [1, a/b]$.
Clearly, $F(1) = 0$. \[
    F'(x) = \frac{1}{n} x^{\frac{1}{n} - 1} - 0 - \frac{1}{n} (x - 1)^{\frac{1}{n} - 1} = \frac{1}{n} \paren{x^{\frac{1}{n}} - (x - 1)^{\frac{1}{n}}}.
\] $x \geq 1 \implies x > x - 1 > 0$.
$n \geq 2 \implies \frac{1}{n} \leq \frac{1}{2} \implies \frac{1}{n} - 1 < 0$.
Thus $x^{\frac{1}{n} - 1} < (x - 1)^{\frac{1}{n} - 1} \implies F'(x) < 0$.
Now suppose $F(a / b) \geq 0$.
By MVT, there exists a $c \in (1, a/b)$ such that $F'(c) = \dfrac{F(x_{0}) - F(1)}{x_{0} - 1} \geq 0$, a contradiction. Thus,
\begin{align*}
    F(a / b) &< 0 \\
    \implies \paren{\frac{a}{b}}^{\frac{1}{n}} - 1 &< \paren{\frac{a}{b} - 1}^{\frac{1}{n}} \\
    \implies a^{\frac{1}{n}} - b^{\frac{1}{n}} &< (a - b)^{\frac{1}{n}}
\end{align*}

\section*{Problem 6}
\[
    f(x) = 
    \begin{cases}
        - nx + \sum_{j = 1}^{n} a_{j} & x < a_{1} \\
        (2k - n) x - \sum_{j = 1}^{k} a_{j} + \sum_{j = k + 1}^{n} a_{j} & a_{k} \leq x < a_{k+1} \\
        nx - \sum_{j = 1}^{n} a_{j} & a_{n} \leq x
    \end{cases}
\]
We first prove that a global minimum exists.

Consider $f$ restricted to $[a, b]$.
By the extreme value theorem, there exists a $c \in [a, b]$ such that $f(c) \leq f(x) \;\forall\; x \in [a, b]$. Moreover, \begin{align*}
    x < a_{1} &\implies -nx > -n a_{1} \implies f(x) > f(a_{1}) \\
    x > a_{n} &\implies nx > na_{n} \implies f(x) > f(a_{n})
\end{align*}
Thus, $f(c) \leq f(x) \;\forall\; x \in \R$.

For $x \neq a_{j}$ for some $j \in [1 .. n]$, $f(x)$ is differentiable with \[
    f'(x) =
    \begin{cases}
        -n & x < a_{1} \\
        2k - n & a_{k} < x < a_{k+1} \\
        n & a_{n} < x
    \end{cases}
\] Thus the potential points of global minimum are $\set{a_{1}, a_{2}, \dots, a_{n}} \cup (a_{k}, a_{k+1})_{2k = n}$.

Note that $f$ is continuous everywhere.

Thus for $2k < n$, we have $f'(x) < 0 \;\forall\; x \in (a_{k}, a_{k + 1}) \implies f(a_{k + 1}) < f(a_{k}) \implies a_{k}$ is not a global minimum.

For $2k > n$, we have $f'(x) > 0 \;\forall\; x \in (a_{k}, a_{k+1}) \implies f(a_{k}) < f(a_{k + 1}) \implies a_{k+1}$ is not a global minimum.

If $n$ is odd, the only potential point of global minimum is thus $x = a_{\frac{n + 1}{2}}$.

If $n$ is even the potential points of global minimum are $\set{a_{\frac{n}{2}}, a_{\frac{n}{2} + 1}} \cup (a_{\frac{n}{2}}, a_{\frac{n}{2} + 1}) = [a_{\frac{n}{2}}, a_{\frac{n}{2} + 1}]$.
Note that the value of $f$ is the same at all these points.

\section*{Problem 7}
We consider: \[
    \frac{f^{-1}(q + k) - f^{-1}(q)}{k}.
\] Let $h(k) = f^{-1}(q + k) - f^{-1}(q) = f^{-1}(q + k) - p$ wherever defined. Thus
\begin{align*}
    h(k) + p &= f^{-1}(q + k) \\
    f(h(k) + p) &= q + k \\
    k &= f(h(k) + p) - q \\
    &= f(h(k) + p) - f(p)
\end{align*}
Note that since $f$ is continuous, $f^{-1}$ is continuous.
Thus as $k \to 0$, $h(k) \to 0$.

If $k \neq 0$, then $h(k) \neq 0$. By the composition theorem,
\begin{align*}
    \lim_{k \to 0} \frac{f^{-1}(q + k) - f^{-1}(q)}{k} &= \lim_{k \to 0} \frac{h}{f(h(k) + p) - f(p)} \\
    &= \lim_{k \to 0} F(h(k))
\end{align*}
where $F(x) = \frac{x}{f(p + x) - f(p)}$. Since $k \neq 0 \implies h(k) \neq 0$,


\end{document}





