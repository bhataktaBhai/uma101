\lecture{3}{fri 21 oct 2022}{Relations and Infinity}
\begin{defn} \label{defn:relation}
    A relation from set $A$ to set $B$ is a subset $R$ of $A \times B$. For any $a \in A, b \in B$ we say $a\mathrel{R}b$ iff $(a, b) \in R$.
    \begin{outline}
        \1 The \emph{domain} of $R$ is the set
        \[
            \text{dom}(R) = \set{ a \in A : (a, b) \in R \text{ for some } b \in B}
        \]
        \1 The \emph{range} of $R$ is the set
        \[
            \text{ran}(R) = \set{ b \in B : (a, b) \in R \text{ for some } a \in A}
        \]
        \1 $R$ is called a \emph{function} from $A$ to $B$, denoted as $R: A \to B$ iff
            \2 dom$(R) = A$
            \2 for each $a \in A$ there is (at most) one $b \in B$ such that $(a, b) \in R$.
    \end{outline}
\end{defn}

\textbf{Exercise:} Write definitions for \emph{injective} and \emph{surjective} functions.
\begin{rem}
    A \emph{bijective} function from $A$ to $B$ is an injective as well as surjective fuction from $A$ to $B$.
\end{rem}

\begin{axiom}[axiom of regularity] \label{zfc:regularity}
    Read up
\end{axiom}
\begin{axiom}[axiom of replacement] \label{zfc:replacement}
    Read up
\end{axiom}
\begin{axiom}[axiom of choice] \label{zfc:choice}
    Read up
\end{axiom}

\begin{defn} \label{defn:zfc:inductive}
    Given a set $A$, its \emph{successor} is the set \[
        A^{+} = A \cup \set{A}.
    \] A set $A$ is said to be \emph{inductive} if $\varnothing \in A$ and for every $a \in A$, we have $a^{+} \in A$.
\end{defn}
\begin{rem}
    The successor of $A$ is guaranteed to exist by \nameref{zfc:pairing} and \nameref{zfc:unions}. \\
    $\set{A}$ exists by \nameref{zfc:pairing} by letting $B = A$. \\
    $A \cup \set{A}$ exists by applying \nameref{zfc:unions} on the set $\set{A, \{A}\}$ formed using \nameref{zfc:pairing} again.

    $\set{A}$ can also be created as a subset (\nameref{zfc:specification}) of the power set (\nameref{zfc:powers}) of $A$.
\end{rem}
\begin{rem}
    The definition of an inductive set is very similar to the principle of mathematical induction in the Peano axioms.
\end{rem}

\begin{axiom}[axiom of infinity] \label{zfc:infinity}
    There exists an inductive set.    
\end{axiom}

\begin{lem} \label{thm:zfc:inductive_intersection}
    Let $\mathscr{F}$ be a nonempty set of inductive sets.
    Then \[
        \bigcap_{A \in \mathscr{F}}{A} \text{ is inductive.}
    \]
\end{lem}
\begin{proof}
    \cref{prob:zfc:inductive_intersection}
\end{proof}

\begin{thm} \label{thm:zfc:omega}
    There exists a \emph{unique}, \emph{minimal} inductive set $\omega$, \textit{i.e.}, for any inductive set $S$, \[
        \omega \subseteq S
    \] and if $\omega'$ is any other inductive set satisfying this property, \[
        \omega = \omega'
    \]
\end{thm}
\begin{proof}
    By \nameref{zfc:infinity}, there exists an inductive set $I$.
    Let $\iota$ be the set of inductive subsets of $I$.
    That is \[
        \iota = \set{X \in \mathscr{P}(I) \mid
        X \text{ is inductive}}
    \] and let $\omega = \bigcap_{X \in \iota}{X}$.

    Let $J$ be an inductive set.
    We have $\omega \cap J \subseteq \omega \subseteq I$.
    Moreover, by \cref{thm:zfc:inductive_intersection}, $\omega \cap J$
    is inductive.
    Thus $\omega \cap J \in \iota$ and so $\omega \subseteq \omega \cap J$.

    Thus, $\omega = \omega \cap J$ and therefore $\omega \subseteq J$.
\end{proof}

\begin{thm}
    The minimal inductive set is a Peano set under the successor function
    $a \mapsto a^{+}$.
\end{thm}
\begin{proof} \leavevmode
    \begin{enumerate}[label=(P\arabic*)]
        \item $\O \in \omega$.
        \item If $a \in \omega$, then $a^{+} \in \omega$.
        \item Suppose there exists $a \in \omega$ such that $\O = a^{+}$.
            Then $\O = a \cup \set{a}$ and so $a \in \O$, a contradiction.
        \item Suppose $a, b \in \omega$ such that $a^{+} = b^{+}$.
            Then $a \cup \set{a} = b \cup \set{b}$ and so
            $a \in b$ or $a = b$.
            Similarly, $b \in a$ or $b = a$.

            If $a \neq b$ then $a \in b$ and $b \in a$.
            This contradicts \nameref{zfc:regularity},
            by applying it on the set $\set{a, b}$.

            Thus $a = b$.
        \item For any set $o \subseteq \omega$, if $\phi \in o$ and
            $a^{+} \in o$ for all $a \in o$, then $o$ is an inductive set.
            By \cref{thm:zfc:omega}, $\omega \subseteq o$ and so $o = \omega$.
            \qedhere
    \end{enumerate}
\end{proof}

\begin{thm}[principle of recursion] \label{thm:zfc:recursion}
    Let $A$ be a set, and $f: A \to A$ be a function. Let $a \in A$. Then, there exists a function $F: \omega \to A$ such that
    \begin{enumerate}[label=(\alph*)]
        \item $F(\varnothing) = a$
        \item For some $b \in \omega$, we have $F(b^{+}) = f(F(b))$
    \end{enumerate}
\end{thm}
\begin{proof}
    Define $F$ as \[
        F \coloneqq \set{(x, y) \in \omega \times A \mid
            (x = \O \land y = a) \lor
            (\exists\; x_{0} \in \omega (x = x_{0}^{+} \land y = f(F(x_{0}))))}.
            \qedhere
    \]
\end{proof}

\textbf{Back to natural numbers!}
