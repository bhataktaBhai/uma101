\lecture{8}{fri 4 nov 2022}{Sequences Continued}
        \item $\set{(-1)^{n}}_{n \in \P}$ is divergent.
        \begin{proof}
            Suppose there exists a limit $L$.

            Let $\varepsilon = 1$.
            Then $\exists\; N \in \P$ such that $\abs{a_{n} - L} < \varepsilon$ for all $n \geq N$.
            \begin{align*}
                \abs{a_{2N} - L} < 1 \quad\implies\quad \abs{L - 1} &< 1 \\
                \abs{a_{2N + 1} - L} < 1 \quad\implies\quad \abs{L + 1} &< 1
            \end{align*}
            Thus,
            \begin{align*}
                \abs{1 - L + L + 1} &\leq \abs{L - 1} + \abs{L + 1} \\
                2 &< 2
            \end{align*}
            Contradiction.
        \end{proof}
    \end{enumerate}
\end{example}

\begin{defn} \label{defn:sequence:bounded}
    A sequence $\set{a_{n}}_{n \in \N}$ is said to be \emph{bounded} if $\exists\; M > 0$ such that $\abs{a_{n}} < M \,\forall\; n \in \N$.
\end{defn}

\begin{thm} \label{thm:sequence:convergent=>bounded}
    Every convergent sequence is bounded.
\end{thm}
\begin{proof}
    Suppose the sequence $\set{a_{n}}$ converges to a limit $L$.
    Thus there exists an $n_{0} \in \N$ such that $\abs{a_{n} - L} < 1 \;\forall\; n \geq n_{0} \implies \abs{a_{n}} < \abs{L} + 1 \;\forall\; n \geq n_{0}$.
    We know that the maximum of a finite number of numbers exists.
    Let $M = \max\set{\abs{L}, \abs{a_{1}}, \abs{a_{2}}, \dots, \abs{a_{n_{0}}}} + 1$.

    Let $n \in \N$.
    If $n < n_{0}$, then $\abs{a_{n}} < M$ by the definition of $M$.
    If $n \geq n_{0}$, then $\abs{a_{n}} < \abs{L} + 1 \leq M$ by the definition of $M$.
    Thus $\abs{a_{n}} < M \;\forall\; n \in \N$ and so every convergent sequence is bounded.
\end{proof}

\begin{defn} \label{defn:sequence:monotone}
    A sequence $\set{a_{n}} \subseteq \R$ is said to be \emph{monotonically increasing} iff $a_{n} \leq a_{n+1} \,\forall\; n \in \N$.

    A sequence $\set{a_{n}} \subseteq \R$ is said to be \emph{monotonically decreasing} iff $a_{n} \geq a_{n+1} \,\forall\; n \in \N$.

    A sequence $\set{a_{n}} \subseteq \R$ is said to be \emph{monotone} iff it is either monotonically increasing or monotonically decreasing.
\end{defn}

\begin{thm}[monotone convergence theorem] \label{thm:sequence:MCT}
    A monotone sequence is convergent iff it is bounded.
\end{thm}
\begin{proof}
    Assume $\set{a_{n}}$ is increasing and $\exists\; M > 0$ such that $\abs{a_{n}} < M \,\forall\; n \in \N$.

    Let $S = \set{a_{n} : n \in \N}$.
    $S$ is nonempty. $S$ is bounded above.
    Thus, by the LUB property, there exists \[
        b \coloneqq \sup S
    \] Let $\varepsilon > 0$.
    Since $a_{n} \leq b \,\forall\; n \in \N$, we have $a_{n} < b + \varepsilon$.

    Since $b$ is the lowest upper bound, $\exists\; N \in \N : a_{N} > b - \varepsilon$.
    Since $\set{a_{n}}$ is monotonically increasing, $b + \varepsilon > a_{n} \geq a_{N} > b - \varepsilon \;\forall\; n \geq N$.
    $\abs{a_{n} - b} < \varepsilon \;\forall\; n \geq N$. Thus the given sequence is convergent, with limit $b$.

    A monotone sequence which is unbouded is divergent, as all convergent sequences are bounded (\cref{thm:sequence:convergent=>bounded}).
\end{proof}

\begin{rem}[Warning!]
    Divergent sequences may diverge for different reasons!
    \begin{itemize}
        \item $\set{(-1)^{n}}$ is bounded but divergent.
        \item $\set{n}$ is unbounded and divergent, to $+\infty$
        \item $\set{(-1)^{n}n}$ is unbounded and divergent, but not to $\infty$ or $-\infty$.
    \end{itemize}
\end{rem}
\begin{defn} \label{defn:sequence:diverging_to_infinity}
    We say that a sequence diverges to $+\infty$ if $\forall\; R \in \R$, $\exists\; N_{R} \in \N$ such that $a_{n} > R \;\forall\; n \geq N_{R}$. \\
    We say that a sequence diverges to $-\infty$ if $\forall\; R \in \R$, $\exists\; N_{R} \in \N$ such that $a_{n} < R \;\forall\; n \geq N_{R}$. \\
    We write $\lim_{n \to \infty} a_{n} = +\infty$ or $\lim_{n \to \infty} a_{n} = -\infty$, but this is purely notational and does not mean ``$\set{a_{n}}$ has a limit''.
\end{defn}
