\lecture{38}{Wed 25 Jan '23}{}

\begin{defn}[linear independence] \label{defn:vector:linear_independence}
    \begin{enumerate}[label=(\alph*)]
        \item A finite sequence of vectors $v_{1}, \dots, v_{m}$ is said to be \emph{linearly dependent} if there exist $c_{1}, \dots, c_{m} \in F$, not all zero, such that \[
            c_{1} v_{1} + \dots + c_{m} v_{m} = 0.
        \] We refer to such a relation as a \emph{nontrivial linear relation} between the vectors $v_{1}, \dots, v_{m}$.
        The vectors are said to be \emph{linearly independent} if they are not linearly dependent.
        That is, \[
            c_{1} v_{1} + \dots + c_{m} v_{m} = 0 \iff c_{1} = \dots = c_{m} = 0.
        \]
        \item A finite set $\set{v_{1}, \dots, v_{m}}$ is said to be linearly independent if the finite sequence $v_{1}, \dots, v_{m}$.
        \item A set $S \subseteq V$ is said to be linearly dependent if there exists a finite subset $L \subseteq S$ such that $L$ is linearly dependent.
        Similarly, $S$ is said to be linearly independent if $L$ is linearly independent for all finite subsets $L \subseteq S$.
    \end{enumerate}

    The empty set is declared to be linearly independent.
\end{defn}

\begin{prop}[] \label{thm:vector:redundancy}
    Let $S = \set{v_{1}, \dots, v_{m}} \subseteq V$ be a linearly dependent set.
    Then there is some $j \in [1 .. m]$ such that \[
        \spann S = \spann(v_{1}, \dots, v_{j-1}, v_{j+1}, \dots, v_{m}),
    \]
    In particular (if $S \neq \set{0}$), $v_{j}$ can be expressed as a linear combination of the other vectors in $S$.
\end{prop}
