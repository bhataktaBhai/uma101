\lecture{9}{mon 7 nov 2022}{Series and Convergence}
\vskip 5pt
\begin{thm}[Tao Theorem 6.1.19] \label{thm:sequence:reciprocal_limit}
    Suppose $\set{b_{n}}$ converges to $b \neq 0$ (and $\exists\; M \in \N$ such that $b_{n} \neq 0 \;\forall\; n \geq M$.) \\
    Then $\set{\frac{1}{b}}_{n \geq M} \to \frac{1}{b}$ as $n \to \infty$.
\end{thm}
\begin{proof}[``Proof'']
    Given $\varepsilon > 0$, find $N \in \N$ such that \[
        \abs{\frac{1}{b_{n}} - \frac{1}{b}} < \varepsilon \;\forall\; n \geq N
    \] 
    This is equivalent to \[
        \frac{\abs{b - b_{n}}}{\abs{b \cdot b_{n}}} < \varepsilon
    \]
    Since $b_{n} \to b$, there exists $M$ such that $\abs{b_{n} - b} < b/2$ for all $n \geq M$. Thus for all $n \geq M$,
    \begin{alignat*}{2}
        \frac{b}{2} &<& b_{n} &< \frac{3b}{2} \\
        \frac{1}{2}\abs{b} &<& \abs{b_{n}} &< \frac{3}{2}\abs{b} \\
        \frac{2}{3\abs{b}} &<& \frac{1}{\abs{b_{n}}} &< \frac{2}{\abs{b}}
    \end{alignat*}
    Now, for any $\varepsilon > 0$, there exists $M'$ such that $\abs{b_{n} - b} <  \dfrac{2}{\abs{b}^{2}} \varepsilon$ for all $n \geq M'$. Thus, \[
        \abs{\frac{1}{b_{n}} - \frac{1}{b}} \frac{\abs{b_{n} - b}}{\abs{b \cdot b_{n}}} < \frac{2 \abs{b_{n} - b}}{\abs{b}^{2}} < \varepsilon
    \] for all $n \geq M.$
\end{proof}

\subsection{Infinite series}
\begin{defn} \label{defn:series}
    An infinite series is a \emph{formal expression} of the form \[
        a_{0} + a_{1} + a_{2} + \dots, \text{ or}, \sum_{n=0}^{\infty} a_{n}
    \]
    Given $\sum_{n=0}^{\infty} a_{n}$, its sequence of partial sums (sops) is $\set{s_{n}}_{n=0}^{\infty}$ where
    \begin{align*}
        s_{0} &= a_{0} \\
        s_{1} &= a_{0} + a_{1} \\
        &\;\;\vdots \\
        s_{n} &= a_{0} + a_{1} + \dots a_{n}
    \end{align*}
    We say that $\sum a_{n}$ is \emph{convergent} with sum $s$ if $\lim_{n \to \infty} s_{n} = s$. Otherwise, we say that $\sum a_{n}$ is divergent.
\end{defn}

\begin{examples}
    \item (Harmonic series) $\sum_{n=1}^{\infty} \frac{1}{n}$ is divergent.
    \begin{proof}
        $\set{s_{n}}$ is a monotonically increasing sequence.
        \begin{align*}
            s_{1} &= 1 \\
            s_{2} &= 1 + \frac{1}{2} \\
            s_{4} &= 1 + \frac{1}{2} + \frac{1}{3} + \frac{1}{4} > 1 + \frac{1}{2} + \frac{1}{4} + \frac{1}{4} \\
            s_{8} &= 1 + \frac{1}{2} + \frac{1}{3} + \dots + \frac{1}{8} \\
               &> 1 + \frac{1}{2} + 2 \cdot \frac{1}{4} + 4 \cdot \frac{1}{8} \\
            s_{2^{k}} &= 1 + \frac{1}{2} + \frac{1}{3} + \dots + \frac{1}{2^{k}} \\
               &> 1 + \frac{1}{2} + 2 \cdot \frac{1}{4} + \dots + 2^{k-1} \cdot \frac{1}{2^{k}} \\
               &= 1 + \frac{k}{2}
        \end{align*}
        Thus, given any $R \in \R, \;\exists\; k \in \N$ such that $s_{2^{k}} > R$. \\
        $\implies \set{s_{n}}$ is divergent as it is unbounded (\cref{thm:sequence:bounded_if_convergent}).
    \end{proof}
    \item $\sum_{n=1}^{\infty} \frac{1}{n^{2}}$ is convergent.
    \begin{proof}
        \begin{align*}
            s_{1} &= 1 \\
            s_{n} &= 1 + \sum_{k=2}^{n} \frac{1}{k^{2}} \\
            &< 1 + \sum_{k=2}^{n} \frac{1}{k(k-1)} \\
            &= 1 + \sum_{k=2}^{n} \paren{\frac{1}{k-1} - \frac{1}{k}} \\
            &= 1 + 1 - \frac{1}{n} \\
            &< 2 \;\forall\; n \in \N
        \end{align*}
        So $\set{s_{n}}$ is a monotonically increasing sequence that is bounded above. \\
        $\implies \set{s_{n}}$ is convergent.
    \end{proof}
\end{examples}

\begin{rem}
    (Telescoping sum)
\end{rem}

\begin{thm} \label{thm:series:vanishing_test}
    Suppose $\sum a_{n}$ is convergent. Then \[
        \lim_{n \to \infty} a_{n} = 0
    \]
\end{thm}
\begin{proof}
    Suppose $\sum a_{n}$ converges to limit $L$.

    Then $\forall\; \varepsilon > 0 \;\exists\; N \in \N$ such that for all $n \geq N$, $\abs{s_{n} - L} < \frac{\varepsilon}{2}$. Now
    \begin{align*}
        \abs{a_{n}} &= \abs{s_{n+1} - s_{n}} \\
        &= \abs{s_{n+1} - L + L - s_{n}} \\
        &\leq \abs{s_{n+1} - L} + \abs{s_{n} - L} \\
        &< \varepsilon
    \end{align*}
    for all $n \geq N$, which implies $a_{n} \to 0$.
\end{proof}
