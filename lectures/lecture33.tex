\lecture{33}{11 jan '23}{Logarithms \& Exponentiation}

\subsection{Logarithms \& Exponentiation}
Goal: for any $b > 0$, we wish to define ``$\log_{b}(x)$'' as \emph{the} real number $z$ which satisfies: $b^z = x$. \\
There are multiple caveats:

\begin{enumerate}[label=(\alph*)]
    \item However, we don't even know what $b^{z}$ means for irrational $z$.
    \item Why should there exist exactly one $z$, and for what values of $x$?
\end{enumerate}

Heuristics: Suppose $z = ``\log_{b}(x)''$ and $w = ``\log_{b}(y)''$, \textit{i.e.}, \[
    b^{z} = x \qquad b^{w} = y.
\] Then $b^{z + w} = x + y$. Thus we expect \begin{equation}
    \log_{b}(x) + \log_{b}(y) = \log_{b}(xy) \;\forall\; x, y \in \dom(\log)
\end{equation}
We will call this function $l$ for the moment.
Suppose $0 \in \dom(\log)$. Then taking $x = 0$ in the above equation, we get \[
    l(0) + l(y) = l(0y) = l(0) \;\forall\; y \in \dom(\log).
\] Thus $l(y) = 0$ for all $y \in \dom(\log)$.

Thus we will exclude $0$ from the domain. Let's calculate $l(1)$. \[
    l(1) + l(1) = l(1) = 0.
\] Since $(-x)(-x) = (x)(x)$, we get $l(-x) + l(-x) = l(x) + l(x) \implies$ \[
    l(-x) = l(x).
\]

Now, differentiating the functional equation with respect to $x$,
\begin{align*}
    l(xy) &= l(x) + l(y) \\
    yl'(xy) &= l'(x) \\
    l'(y) &= \frac{l'(1)}{y}
\end{align*}

By the second fundamental theorem of calculus, \[
    l(y) = \begin{cases}
        l(y) - l(1) = \int_{1}^{y} \dfrac{l'(1)}{x} \dd x & y > 0 \\
        l(y) - l(-1) = \int_{-1}^{y} \dfrac{l'(1)}{x} \dd x
    \end{cases}
\]

\begin{defn}[Natural Logarithm] \label{defn:ln}
    Let $x > 0$. The \emph{natural logarithm} of $x$ is the quantity \[
        \ln(x) = \int_{1}^{x} \frac{1}{t} \dd t.
    \]
\end{defn}

\begin{thm}[]
    The function $\ln : \R^{+} \to \R$ has the following properties:
    \begin{enumerate}[label=(\alph*)]
        \item $\ln(1) = 0$.
        \item $\ln(x) + \ln(y) = \ln(xy) \;\forall\; x, y \in \R^{+}$.
        \item $\ln$ is continuous and strictly increasing.
        \item $\ln$ is differentiable and \[
            \ln'(x) = \frac{1}{x}.
        \]
        \item (Leibniz) \[
            \int \frac{1}{t} \dd t = \ln \abs{t} + C.
        \]
        \item (Leibniz) \[
            \int \ln x \dd x = x \ln x - x + C.
        \]
        \item $\ln$ is bijective.
    \end{enumerate}
\end{thm}
\begin{proof} \leavevmode
    \begin{enumerate}[label=(\alph*)]
        \item $\ln(1) = \int_{1}^{1} \frac{1}{t} \dd t = 0$.
        \item Suppose $x, y > 0$. Then
        \begin{align*}
            \ln(\frac{1}{x}) &= \int_{1}^{\frac{1}{x}} \frac{1}{t} \dd t \\
            &= \frac{1}{x} \int_{x}^{1} \frac{x}{t} \dd t \\
            &= - \int_{1}^{x} \frac{1}{t} \dd t \\
            &= - \ln(x). \qedhere
        \end{align*}
    \end{enumerate}
\end{proof}

\begin{defn}[e \& Exponentiation] \label{defn:e}
    Let $e$ be the unique number that satisfies \[
        \ln(e) = 1.
    \] Given any $x \in \R$, let $\exp(x)$ be the unique positive $y$ such that \[
        \ln(y) = x.
    \] That is, $\exp$ is the inverse function of $\ln$.
\end{defn}

Hard exercise: Prove that this definition of $e$ is identical to the infinite series definiton.

\begin{thm}[]
    $\exp : \R \to \R^{+}$ has the following properties:
    \begin{enumerate}[label=(\alph*)]
        \item $\exp(0) = 1$.
        \item $\exp(x + y) = \exp(x) \exp(y)$.
        \item $\exp$ is continuous and strictly increasing.
        \item $\exp$ is differentiable and $\exp'(x) = \exp(x) \;\forall\; x \in \R$.
        \item $\int \exp(x) \dd x = \exp(x) + C.$
        \item $\exp$ is bijective.
        \item $\exp(r) = e^{r} \;\forall\; r \in \Q$.
    \end{enumerate}
\end{thm}
