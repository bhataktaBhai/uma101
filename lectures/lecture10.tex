\lecture{10}{wed 9 nov 2022}{Geometric Series; Comparison Test}

\begin{example}[Geometric Series]
    Let $x \in \R$. Then \[
        \sum_{n=0}^{\infty} x^{n} = \begin{cases}
            \frac{1}{1-x} & \abs{ x} < 1 \\
            \text{diverges} & \abs{ x} \geq 1
        \end{cases}
    \]
\end{example}
\begin{proof}
    \begin{align*}
        s_{n} &= \paren{1 + x + x^{2} + \cdot x^{n} } \cdot \frac{1-x}{1-x} \\
        &= \frac{1-x^{n+1}}{1-x} \\
        &= \frac{1}{1-x} - \frac{x^{n+1}}{1-x}
    \end{align*}
    Claim: $x^{n}$ tends to 0 if $\abs{ x} < 1$ and diverges if $\abs{ x}  > 1$
    \begin{enumerate}[label=(Case \arabic*)]
        \item $\abs{ x} < 1$. Suffices to prove for $x > 0$. \\
        Note that $(1 + y)^{n} = 1 + ny + \dots > ny$. \\
        If $x < 1$, then $\frac{1}{x} > 1$. Say $\frac{1}{x} = 1 + y$ for some $y > 0$. \\
        Then \[
            \paren{\frac{1}{x}}^{n} = \paren{1 + y}^{n} > ny = n\paren{\frac{1}{x} - 1}
        \] So \[
            x^{n} < \frac{1}{n} c, \quad c = \frac{1}{\frac{1}{x} - 1}
        \] Since $\frac{1}{n}$ tends to $0$, $x^{n}$ tends to $0$ by the squeeze theorem (HW). By limit laws,
        \begin{align*}
            \lim_{n \to \infty} s_{n} &= \lim_{n \to \infty} \paren{\frac{1}{1-x} - \frac{x}{1-x}\cdot x^{n}} \\
            &= \frac{1}{1-x}
        \end{align*}
        
        \item $\abs{ x} \geq 1, x \neq 1$. $x = -1$ diverges as seen before. \\
        For $\abs{ x} > 1$, $\abs{ x} ^{n} = \paren{1 + (\abs{ x} - 1)}^{n} > n(\abs{ x} - 1)$. \\
        Once again, using the ``limit laws'', \[
            s_{n} = \frac{1}{1-x} - \frac{x}{1-x} \cdot x^{n} \text{ diverges}
        \]

        \item x = 1. $\sum (1)^{n}$ diverges since $a_{n} = 1^{n} \to 1 \neq 0$ as $n \to \infty$ \qedhere
    \end{enumerate}
\end{proof}

\begin{thm}[Comparison test] \label{thm:series:comparison}
    Suppose there exist constants $M \in \N$ and $0 < C$ such that \[
        0 \leq a_{n} \leq C b_{n} \quad\forall\; n \geq M
    \] If $\sum b_{n}$ converges, then $\sum a_{n}$ converges. In other words, If $\sum a_{n}$ diverges, $\sum b_{n}$ diverges.
\end{thm}
\begin{proof}
Let $\set{s_{n}}$ and $\set{t_{n}}$ be the suquence of partial sums of $\sum_{n=M}^{\infty} a_{n}$ and $\sum_{n=M}^{\infty} b_{n}$, respectively. Note that $\set{s_{n}}$ and $\set{t_{n}}$ are increasing sequences. By convergent of $\set{t_{n}}$, there exists $N \in\N$ and $L > 0$ such that \[
        t_{n} < L \quad \forall\; n \geq N
    \] Thus, $0 \leq s_{n} \leq Ct_{n} < CL \;\forall\; n \geq \max\set{M, N}$. \\
    Thus, $\set{s_{n}}$ is bounded and by monotone convergence theorem, $s_{n}$ converges. \\
    $\implies \sum a_{n}$ converges.
\end{proof}
\begin{example}
    Let $p \in \R$. Claim: \[
        \sum_{n=1}^{\infty} \frac{1}{n^{p}}
        \begin{cases}
            \text{converges} & p > 1 \\
            \text{diverges} & p \leq 1
        \end{cases}
    \]
\end{example}
\begin{proof}
    If $p \leq 0,$ check \[
        \paren{\frac{1}{n}} \not\to 0
    \] So $\sum \frac{1}{n^{p}}$ diverges.

    If $0 < p \leq 1$, then $n^{p} \leq n \;\forall\; n \geq 1$. Thus $\frac{1}{n^{p}} \geq \frac{1}{n}$. By the comparison test theorem, since $\sum \frac{1}{n}$ diverges, so does $\sum \frac{1}{n^{p}}$

    If $p \geq 2$, we have $\frac{1}{n^{p}} < \frac{1}{n^{2}}$. By the comparison test, this sum converges.

    Finally, if $1 < p < 2$. Note that
    \begin{align*}
        s_{1} &= 1 \\
        s_{3} &= 1 + \frac{1}{2^{p}} + \frac{1}{3^{p}} \leq 1 + 2 \cdot \frac{1}{2^{p}} \\
        s_{7} &= 1 + \frac{1}{2^{p}} + \frac{1}{3^{p}} + \dots \frac{1}{7^{p}} \leq 1 + 2\cdot \frac{1}{2^{p}} + 4 \cdot \frac{1}{4^{p}} \\
        s_{2^{k} - 1} &\leq 1 + \frac{2}{2^{p}} + \dots + \frac{2^{k-1}}{2^{p(k-1)}} \\
        &= 1 + \frac{1}{2^{p-1}} + \frac{1}{2^{2(p-1)}} + \dots + \frac{1}{2^{(k-1)(p-1)}}
    \end{align*}
    The RHS are sops of \[
        \sum_{n=0}^{\infty} \paren{\frac{1}{2^{p-1}}}^{n} = \frac{1}{1-\paren{\frac{1}{2}}^{p-1}}
    \] and so $s_{2^{k}-1} < \dfrac{1}{1-\paren{\frac{1}{2}}^{p-1}} \;\forall\; k \in \P$.
    We also know that $\set{s_{n}}$ is increasing.

    Thus by \nameref{thm:sequence:MCT} $\set{s_{n}}$ converges (\textcolor{exercise}{prove $\set{s_{n}}$ is bounded}).
\end{proof}
