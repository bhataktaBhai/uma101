\lecture{12}{mon 14 nov 2022}{Limit of a Function}

\section{Limits \& Continuity}
\subsection{Limit of a function}
\vskip 5pt
\begin{defn}[Neighborhood] \label{defn:limit:neighborhood}
    Given a real number $p$ and an $\varepsilon > 0$, the \emph{$\varepsilon$-neighborhood of $p$} is the open interval \[
        N_{\varepsilon}(p) = \paren{p - \varepsilon, p + \varepsilon} = \set{x \in \R : \abs{ x - p} < \varepsilon}.
    \] 
\end{defn}

\begin{defn}[Limit of a function] \label{defn:limit:function}
    Given a function $f$ that is defined on some $I = (a, p) \cup (p, b)$ with $a < b$, we say that $f$ has a \emph{limit} $L$ as it approaches $p$ iff for every $\varepsilon > 0 \;\exists\; \delta > 0$ such that
    \begin{enumerate}[label=(\alph*)]
        \item $0 < \abs{ x - p } < \delta \implies \abs{ f(x) - L } < \varepsilon$ OR
        \item $x \in N_{\delta}(p) \setminus \set{p} \implies f(x) \in N_{\varepsilon}(L)$.
    \end{enumerate}
    This is denoted as \[
        \lim_{x \to p} f(x) = L.
    \]
\end{defn}

\begin{example} \leavevmode
    \begin{enumerate}[label=(\alph*)]
        \item For $f(x) = c$, $c \in \R$, \[
            \lim_{x \to p} f(x) = c.
        \] Choose $\delta = 1$.
        $0 < \abs{ x - p } < \delta \implies \abs{ f(x) - c } = 0 < \varepsilon \;\forall\; \varepsilon > 0$.
        \item For $f(x) = x$, \[
            \lim_{x \to p} f(x) = p.
        \] Choose $\delta = \varepsilon$.
        $ 0 < \abs{ x - p } < \delta \implies \abs{ f(x) - p} < \varepsilon$.
        \item For $f(x) = \sqrt{x}$ and $p > 0$, \[
            \lim_{x \to p} f(x) = \sqrt{p}.
        \]
        \begin{align*}
            \abs{ \sqrt{x} - \sqrt{p} } &< \varepsilon \\
            \iff \frac{\abs{ x - p }}{\abs{ \sqrt{x} + \sqrt{p} }} &< \varepsilon \\
        \end{align*}
        Take $\delta = \min \set{p, \sqrt{p} \varepsilon}$ (this is to make sure $f$ is defined for all points in $N_{\delta}(p) \setminus \set{p}$).
        Now $\abs{ x - p} < \delta \implies$
        \begin{align*}
            \frac{\abs{ x - p }}{\abs{ \sqrt{x} + \sqrt{p} }} &< \frac{\delta}{\abs{ \sqrt{x} + \sqrt{p} }} \\
            &= \frac{\sqrt{p} \varepsilon}{\abs{ \sqrt{x} + \sqrt{p} }} \\
            &< \varepsilon \qedhere
        \end{align*}
