\lecture{5}{fri 28 oct 2022}{Fields, Ordered Sets and Ordered Fields}
\subsection{Fields, Ordered Sets and Ordered Fields}
We cannot solve
\begin{align*}
    3 + x &= 2 \\
    3 \cdot x &= 2
\end{align*}
\hfill in \N.

\begin{defn} \label{defn:field}
    A field is a set $F$ with 2 operations $+ : F \times F \to F$ and $\cdot : F \times F \to F$ such that
    \begin{enumerate}[label=(F\arabic*)]
        \item \label{defn:field:commutativity}
            $+$ \& $\cdot$ are commutative on $F$.
        \item \label{defn:field:associativity}
            $+$ \& $\cdot$ are associative on $F$.
        \item \label{defn:field:distributivity}
            $+$ \& $\cdot$ satisfy distributivity on $F$, \textit{i.e.}, $a \cdot (b + c) = a \cdot b + a \cdot c$ for all $a, b, c \in F$. 
        \item \label{defn:field:identity}
            There exist 2 \emph{distinct} elements, called 0 (additive identity) and 1 (multiplicative identity) such that
            \begin{align*}
                x + 0 &= x \\
                x \cdot 1 &= x
            \end{align*}
            for all $x \in F$
        \item \label{defn:field:negative}
            For every $x \in F, \,\exists\; y \in F$ such that \[
                x + y = 0
            \]
        \item \label{defn:field:reciprocal}
            For every $x \in F \setminus \set{0}, \,\exists\; z \in F$ such that \[
                x \cdot z = 1
            \]
    \end{enumerate}
\end{defn}
\begin{rem}
    We are tempted to call $y$ in \labelcref{defn:field:negative} ``$-x$'' and $z$ in \labelcref{defn:field:reciprocal} ``$\frac{1}{x}$'' but $y, z$ haven't been proven to be unique yet.
    \textcolor{exercise}{Prove this}.
    \textcolor{solved}{Proved as \cref{thm:field:unique_inverses}}.

    Once we have proven this, we can also define $a - b := a + (-b)$ and $a/b = a \cdot \frac{1}{b}$.
\end{rem}

\begin{thm} \label{thm:field:zero_product}
    $(F, +, \cdot)$ is a field. Then for all $x$, \[
        0 \cdot x = x \cdot 0 = 0
    \]
\end{thm}
\begin{proof}
    By \labelcref{defn:field:commutativity}, the first equality holds. \\
    Now by \labelcref{defn:field:identity}, $1 + 0 = 1$. \\
    By \labelcref{defn:field:distributivity}, $x \cdot (1 + 0) = x \cdot 1 + x \cdot 0$ \\
    So $x \cdot 1 = x \cdot 1 + x \cdot 0$ or $x = x + x \cdot 0$ \labelcref{defn:field:identity}\\
    Adding $y$ to both sides where $x + y = 0$ \labelcref{defn:field:negative} and using associativity and commutativity,
    \begin{align*}
        x + y &= x + x \cdot 0 + y \\ 
        0 &= x + y + x \cdot 0 \\
        &= 0 + x \cdot 0 \\
        &= x \cdot 0
    \end{align*}
    By commutativity, $0 \cdot x = 0$.
\end{proof}

\begin{defn} \label{defn:order}
    A set $A$ with a relation $<$ is called an \emph{ordered set} if
    \begin{enumerate}[label=(O\arabic*)]
        \newcounter{temp}
        \item \label{defn:order:trichotomy}
            (Trichotomy) For every $x, y \in A$, exactly one of the following holds. \[
                x < y, \quad x = y, \quad y < x
            \]
        \item \label{defn:order:transitivity}
            (Transitivity) If $x < y$ and $y < z$, then $x < z$.
        \setcounter{temp}{\value{enumi}}
    \end{enumerate}
    \textbf{Notation:} $x < y$ is read as ``x is less than y'' \\
    $x \leq y$ means $x < y$ or $x = y$, read as ``x is less that or equal to y''. \\
    $x > y$ is read as ``$x$ is greater that $y$'' and equivalent to $y < x$.
\end{defn}
\begin{example}
    $\N$ with $a < b$ iff $b = a + k$ for some non-zero $k$.
\end{example}

\begin{defn} \label{defn:ordered_field}
    An \emph{ordered field} is a set that admits two operations $+$ and $\cdot$ and relation $<$ so that $(F, +, \cdot)$ is a field and $(F, <)$ is an ordered set and:
    \begin{enumerate}[label=(O\arabic*)]
        \setcounter{enumi}{\value{temp}}
        \item \label{defn:order:sum}
            For $x, y, z \in F$, if $x < y$ then $x + z < y + z$.
        \item \label{defn:order:product}
            For $x, y \in F$, if $0 < x$ and $0 < y$ then $0 < x \cdot y$.
    \end{enumerate}
\end{defn}

\begin{lem} \label{thm:field:unique_inverses}
    Given a field $(F, +, \cdot)$: For any element $a$ in a field $F$, there exists ony one $b$ such that $a + b = 0$. We will denote this $b$ as $-a$. Similarly for any $a$ in $F \setminus \set{0}$ there exists only one $b \in F$ such that $ab = 1$. We will denote this $b$ as $\frac{1}{a}$ or $a^{-1}$.
\end{lem}
\begin{proof}
    Suppose $a + b_{1} = 0$ and $a + b_{2} = 0$. Adding $b_{1}$ to both sides, we get $a + b_{2} + b_{1} = 0 + b_{1} \implies (a + b_{1}) + b_{2} = b_{1} \implies b_{2} = b_{1}$. \\
    The second part of the proof is analogous.
\end{proof}

\begin{lem} \label{thm:field:inverse_involution}
    $-(-a) = a = (a^{-1})^{-1}$
\end{lem}
\begin{proof}
    $a + (-a) = 0$, so the additive inverse of $-a$ is $a$.

    $aa^{-1} = 1$, so the multiplicative inverse of $a^{-1}$ is $a$.
\end{proof}

\begin{lem} \label{thm:field:negative_product}
    For any field $(F, +, \cdot)$, $(-a)b = -(ab)$ and $(-a)(-b) = ab$.
\end{lem}
\begin{proof}
    By the distributive law, we have \[
        (a + (-a))b = ab + (-a)b \implies 0 = ab + (-a)b \implies (-a)b = -(ab).
    \]
    It follows that $(-a)(-b) = -(a(-b)) = -(-(ab)) = ab$.
\end{proof}

\begin{thm} \label{thm:field:0<1}
    For any field $(F, +, \cdot)$, $0 < 1$.
\end{thm}
\begin{proof}
    By \cref{defn:order:trichotomy} and \labelcref{defn:field:identity}, either $0 < 1$ or $0 > 1$.

    If $0 < 1$, we are done.

    If $0 > 1$, then on adding $-1$ on both sides, $0 < -1$ by \cref{defn:order:sum}.
    So $0 < (-1)(-1) \iff 0 < 1$, a contradiction.
\end{proof}
\begin{rem}
    ``a contradiction'' is not necessary to state for the proof to be complete. See \href{https://teams.microsoft.com/l/message/19:5PNDOetYK3gbPZWX5Muk\_KnaEXgulRmRNwNmAHA8dZ81@thread.tacv2/1666960265828?tenantId=6f15cd97-f6a7-41e3-b2c5-ad4193976476&groupId=9cf683b7-9233-4d97-a7eb-1dc0051039a7&parentMessageId=1666960265828&teamName=UM\%20101\%20October\%202022&channelName=General&createdTime=1666960265828&allowXTenantAccess=false}{this discussion} at MS Teams.
\end{rem}
