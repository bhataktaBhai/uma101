\lecture{7}{wed 2 nov 2022}{Sequences and Convergence}
\section{Sequences \& Series}
We will now assume \emph{everything} about real numbers: multiplication, division, exponentiation, etc.

\subsection{Sequences} \vskip 5pt
\begin{defn} \label{defn:sequence}
    A \emph{sequence} in $\R$ is a function $f: \N \to \R$. We denote this sequence by $\set{a_{n}}_{n \in \N}$, where \[
        a_{n} = f(n) \quad \forall\; n \in \N
    \] and $a_{n}$ is called the $n^{th}$ term of $\set{a_{n}}_{n \in \N}$.
\end{defn}
\begin{rem}
    $\set{a_{n}} \subseteq \R$ will denote a sequence of real numbers. \\
    The numbering starts at 0 technically, but doesn't really matter. We will often omit the subscript $n \in \N$ and start indexing from some other point.
\end{rem}

\begin{defn} \label{defn:sequence:convergence}
    We say that a sequence $\set{a_{n}} \subseteq \R$ is \emph{convergent} (in $\R$) if $\exists\; L \in \R$ such that for each $\varepsilon > 0, \,\exists\; N_{\varepsilon,L} \in \N$ such that \[
        \abs{a_{n} - L} < \varepsilon \quad \forall\; n \geq N_{\varepsilon, L}
    \] 
    We will call $L$ \emph{a} limit of $\set{a_{n}}$ and we write: \[
        a_{n} \to L \text{ as } n \to \infty
    \]
    A sequence $\set{a_{n}}$ is said to be \emph{divergent} if it is not convergent, \textit{i.e.}, $\forall\; L \in \R$ and $N_{L} \in \N$, $\exists\; \varepsilon > 0$ and $N \geq N_{L}$ such that \[
        \abs{a_{N} - L} > \varepsilon
    \]
\end{defn}

\begin{thm}[uniqueness of limits] \label{thm:sequence:unique_limit}
    Suppose $L_{1}$ and $L_{2}$ are limits of a (convergent) sequence $\set{a_{n}} \in \R$. Then $L_{1} = L_{2}$.
\end{thm}
\begin{proof}[Proof \textcolor{self_proof}{(self)}]
    Suppose $L_{1} \neq L_{2}$. Then define $\varepsilon = \frac{\abs{L_{1} - L_{2}}}{2}$.
    There exists $N_{1}$ such that $\abs{a_{n} - L_{1}} < \varepsilon \;\forall\; n \geq N_{1}$.
    So for all $n \geq N_{1}$,
    \begin{align*}
        \abs{L_{1} - L_{2}}  &= \abs{L_{1} - a_{n} + a_{n} - L_{2}}  \\
        &\leq \abs{a_{n} - L_{1}} + \abs{a_{n} - L_{2}} \\
        &\leq \varepsilon + \abs{a_{n} - L_{2}}  \\
        \implies 2 \varepsilon  &\leq \varepsilon + \abs{a_{n} - L_{2}} \\
        \varepsilon &\leq \abs{a_{n} - L_{2}} \qedhere
    \end{align*}
    % $\abs{ a_{n} - L_{1} } = \abs{ a_{n} - L_{2} + L_{2} - L_{1} } \leq \abs{ a_{n} - L_{2} } + \abs{ L_{2} - L_{1} }$
\end{proof}

\begin{examples}
    \item Let $\set{a_{n}} = \frac{1}{n^{p}} \,\forall\; n \in \P$, where $p > 0$. \[
        \lim_{n \to \infty} a_{n} = 0
    \]
    \begin{proof}
        Let $\varepsilon > 0$. \\
        By the \nameref{thm:R:archimedean} applied to $x = \varepsilon^{\frac{1}{p}}$ and $y = 1$, $\exists\; N \in \P$ such that: \[
            N \varepsilon^{\frac{1}{p}} > 1 \implies \varepsilon^{\frac{1}{p}} > \frac{1}{N} \implies \varepsilon > \frac{1}{N^{p}}
        \]
        Let $n \geq N$. Then
        \begin{align*}
            \abs{\frac{1}{n^{p}} - 0} &= \frac{1}{n^{p}} \\
            &\leq \frac{1}{N^{p}} \\
            &< \varepsilon \qedhere
        \end{align*}
    \end{proof}
