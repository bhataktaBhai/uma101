\documentclass[12pt]{article}
\usepackage[utf8]{inputenc}
\usepackage[a4paper, total={6in, 9in}]{geometry}
\usepackage{amsmath}
\usepackage{amssymb}
\usepackage{amsthm}
\usepackage{enumitem}
\usepackage{outlines}

\usepackage{paracol}

\usepackage[dvipsnames]{xcolor}
\colorlet{exercise}{red!80!black}
\colorlet{solved}{green!30!black}
\colorlet{self_proof}{blue!30!black}

\usepackage{mathrsfs}
\usepackage{parskip}

\usepackage{accents}
\newcommand{\ubar}[1]{\underaccent{\bar}{#1}}

\usepackage{hyperref}
\usepackage{cleveref}

\providecommand{\dd}{\,\mathrm{d}}

\usepackage{mathtools} % also contains \coloneqq
\DeclarePairedDelimiter{\oldparen}{(}{)}
\DeclarePairedDelimiter{\oldset}{\{}{\}}
\DeclarePairedDelimiter{\oldabs}{\lvert}{\rvert}
\DeclarePairedDelimiter{\oldnorm}{\lVert}{\rVert}
\DeclarePairedDelimiter{\oldfloor}{\lfloor}{\rfloor}
\DeclarePairedDelimiter{\oldceil}{\lceil}{\rceil}

\makeatletter
\def\paren{\@ifstar{\oldparen}{\oldparen*}}
\def\set{\@ifstar{\oldset}{\oldset*}}
\def\abs{\@ifstar{\oldabs}{\oldabs*}}
\def\norm{\@ifstar{\oldnorm}{\oldnorm*}}
\def\floor{\@ifstar{\oldfloor}{\oldfloor*}}
\def\ceil{\@ifstar{\oldceil}{\oldceil*}}

\g@addto@macro\normalsize{%
  \setlength\abovedisplayskip{7pt}%
  \setlength\belowdisplayskip{7pt}%
  \setlength\abovedisplayshortskip{1pt}%
  \setlength\belowdisplayshortskip{1pt}%
}
\makeatother

\newcommand\N{\ensuremath{\mathbb{N}}}
\newcommand\R{\ensuremath{\mathbb{R}}}
\newcommand\Z{\ensuremath{\mathbb{Z}}}
\renewcommand\O{\ensuremath{\varnothing}}
\renewcommand\P{\ensuremath{\mathbb{P}}}
\newcommand\Q{\ensuremath{\mathbb{Q}}}
\newcommand\C{\ensuremath{\mathbb{C}}}

% make implied by and implies arrows shorter
\let\implies\Rightarrow
\let\impliedby\Leftarrow
\let\iff\Leftrightarrow

% make \epsilon and \varepsilon the same
\let\epsilon\varepsilon

% <theorems>
% \theoremstyle{plain}
% \newtheorem{thm}{Theorem}[section]
% \newtheorem*{thm*}{Theorem}
% \newtheorem{prop}[thm]{Proposition}
% \newtheorem{lem}[thm]{Lemma}
% \newtheorem{cor}[thm]{Corollary}
% \newtheorem{axiom}{Axiom}[section]
% \newtheorem*{exer}{Exercise}
% 
% \theoremstyle{definition}
% \newtheorem{defn}[thm]{Definition}
% 
% \theoremstyle{remark}
% \newtheorem*{rem}{Remark} 
% \newtheorem*{example}{Example}
% </theorems>

% Gilles Castel's theorems
\usepackage[framemethod=Tikz]{mdframed}
\mdfsetup{skipabove=1em,skipbelow=0em}
\mdfdefinestyle{axiomstyle}{
    outerlinewidth = 1.5,
    roundcorner = 10,
    leftmargin = 15,
    rightmargin = 15,
    backgroundcolor = blue!10
}
\mdfdefinestyle{defnstyle}{
    outerlinewidth = 1.5,
    % roundcorner = 10,
    leftmargin = 7,
    rightmargin = 7,
    backgroundcolor = green!10
}
\mdfdefinestyle{thmstyle}{
    outerlinewidth = 1.5,
    roundcorner = 10,
    leftmargin = 7,
    rightmargin = 7,
    backgroundcolor = yellow!10
}
\mdfdefinestyle{lemmastyle}{
    outerlinewidth = 1.5,
    roundcorner = 10,
    leftmargin = 7,
    rightmargin = 7,
    backgroundcolor = yellow!10
}
\theoremstyle{definition}
\newmdtheoremenv[nobreak=true, style=axiomstyle]{axiom}{Axiom}[section]
\newmdtheoremenv[nobreak=true, style=thmstyle]{thm}{Theorem}[section]
\newmdtheoremenv[nobreak=true]{prop}[thm]{Proposition}
\newmdtheoremenv[nobreak=true]{lem}[thm]{Lemma}
\newmdtheoremenv[nobreak=true]{cor}[thm]{Corollary}
\newmdtheoremenv[nobreak=true, style=defnstyle]{defn}[thm]{Definition}

\newcounter{assignment}
\newmdtheoremenv[nobreak=true]{problem}{Problem}[assignment]

\theoremstyle{remark}
\newtheorem*{rem}{Remarks}
\newtheorem*{example}{Example}
\newtheorem*{examplex}{Examples}

\newenvironment{examples}
{\begin{examplex}\leavevmode\begin{itemize}}{\end{itemize}\end{examplex}}
% <cref>
\crefname{thm}{theorem}{theorems}
\crefname{prop}{proposition}{propositions}
\crefname{lem}{lemma}{lemmas}
\crefname{cor}{corollary}{corollaries}
\crefname{axiom}{axiom}{axioms}
\crefname{defn}{definition}{definitions}
\crefname{problem}{problem}{problems}
% </cref>

% <hyperlinks>
\hypersetup{colorlinks,
    linkcolor={red!50!black},
    citecolor={blue!50!black},
    urlcolor={blue!80!black}}
% </hyperlinks>

\usepackage{xifthen}
\def\testdateparts#1{\dateparts#1\relax}
\def\dateparts#1 #2 #3 #4 #5\relax{
    \marginpar{\small\textsf{\mbox{#1 #2 #3 #5}}}
}

\def\@lecture{}%
\newcommand*{\lecture}[3]{
    \ifthenelse{\isempty{#3}}{%
        \def\@lecture{Lecture #1}%
    }{%
        \def\@lecture{Lecture #1: #3}%
    }%
    \subsection*{\@lecture}
    \marginpar{\raggedright\small\textsf{#2}}
    \vskip 6pt
}
\newcommand*{\assignment}[2]{%
    \stepcounter{assignment}%
    \subsection{Assignment #1}
    \marginpar{\raggedright\small{due #2}}
    \vskip 6pt
}
\makeatletter
\newcommand*{\refifdef}[3]{%label,command,fallback
    \@ifundefined{r@#1}{#3}{#2{#1}}%
}
\makeatother

\DeclareMathOperator{\dom}{dom}
\DeclareMathOperator{\ran}{ran}
\DeclareMathOperator{\spann}{span}
\DeclareMathOperator{\dimn}{dim}

\newcommand{\psum}[2]{\ensuremath{\mathrm{sum}_{#1}({#2})}}
\newcommand{\pprod}[2]{\ensuremath{\mathrm{prod}_{#1}({#2})}}


\title{Assignment 8}
\author{Naman Mishra}
\date{23 Dec 2022}

\begin{document}
\maketitle

\textbf{Problem 1.}
Let $f : [a, b] \to \R$ be continuous on $[a, b]$ and differentiable on $(a, c) \cup (c, b)$. Show that if $\lim\limits_{x \to c} f'(x) = L$, then $f'(c)$ exists and equals $L$.

\begin{proof}
    For all $\varepsilon > 0$ there exists a $\delta > 0$ such that $x \in N_{\delta}(c) \setminus \set{c} \implies \abs{ f'(x) - L } < \varepsilon$. \[
        \lim_{x \to c} \frac{f(x) - f(c)}{x - c} = L
    \]
\end{proof}

\section*{Problem 7}
We define a function $f : \R \to \R$ to be `close' at a point $c \in \R$ if for every $\varepsilon > 0$, there exists an $L \in \R$ and a $\delta > 0$ such that for every $x \in N_{\delta}(c) \setminus \set{c}$, we have that \[
    \abs{ f(x) - L } < \varepsilon.
\]

Suppose that $f : \R \to \R$ is close at $c \in \R$.
Then for every $\varepsilon > 0$, there exists an $L \in \R$ and a $\delta_{\varepsilon} > 0$ such that for every $x \in N_{\delta_{\varepsilon}}(c) \setminus \set{c}$, we have \[
    \abs{ f(x) - L } < \varepsilon / 2.
\]
Consider a sequence $\set{a_{n}} \subset \R \setminus \set{c}$ converging to $c$.
Then there exists an $n_{0}$ such that for every $n \geq n_{0}$, we have \[
    \abs{ a_{n} - c } < \delta_{\varepsilon} \implies \abs{ f(a_{n}) - L } < \varepsilon / 2
\] By the triangle inequality, \[
    \abs{ f(a_{n}) - f(a_{m}) } \leq \abs{ f(a_{n}) - L } + \abs{ L - f(a_{m}) } < \varepsilon \;\forall\; m, n \geq n_{0}
\] Thus $\set{f(a_{n})}$ is a Cauchy sequence, and therefore convergent.

Suppose sequences $\set{a_{n}}$ and $\set{b_{n}}$ both converge to (but never equal) $c$, with $\set{f(a_{n})}$ and $\set{f(b_{n})}$ having different limits $L_{1}$ and $L_{2}$. Consider the sequence \[
    c_{n} = \begin{cases}
        a_{n} & \text{if } n \text{ is even} \\
        b_{n} & \text{if } n \text{ is odd}
    \end{cases}
\] Clearly $\set{c_{n}}$ converges to $c$, but $\set{f(c_{n})}$ diverges. This is a contradiction.

Thus there exists a unique $L_{0} \in \R$ such that given any sequence $\set{a_{n}}$ converging to $c$, we have that \[
    \lim_{n \to \infty} f(a_{n}) = L_{0}.
\]

By the sequential characterization of limits, the limit of $f$ at $c$ exists and equals $L_{0}$.

On the other hand, if it is known that $f$ has a limit $L_{0}$ at $c$, setting $L = L_{0} \;\forall\; \varepsilon$ proves that $f$ is close at $c$.

Thus the two definitions are equivalent.
\end{document}
